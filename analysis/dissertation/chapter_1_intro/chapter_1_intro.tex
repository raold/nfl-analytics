% !TEX root = ../main/main.tex
% chapter_1_intro.tex  
\chapter{Introduction and Motivation}
\label{chap:intro}

\section{Why Focus on the NFL}

The **National Football League (NFL)** is an ideal testbed for quantitative decision systems because it combines: (i) deep liquid betting markets; (ii) abundant, granular data; and (iii) inherently sequential strategic decisions. These properties support rigorous modeling and credible evaluation of edge.

\begingroup\RaggedRight\sloppy\hbadness=10000
\begin{itemize}
  \item \textbf{Liquidity.} Many books and large volumes enable precise price comparisons and operationally meaningful CLV measurement.
  \item \textbf{Data richness.} Public play-by-play, injury reports, and weather feeds support feature engineering and validation at weekly cadence.
  \item \textbf{Sequential decisions.} Weekly cycles and interdependent markets (spread/total/SGP) invite RL-style policies with risk controls.
\end{itemize}
\endgroup

\section{Research Questions and Objectives}

We pursue four questions:
\begin{enumerate}
  \item \textbf{Architecture.} How to combine classical models, modern ML, and RL coherently for NFL markets?
  \item \textbf{Uncertainty \& risk.} How to quantify uncertainty and translate it into safe stake sizing?
  \item \textbf{Value of information.} What is the marginal lift from feature families (injury, rest, weather, microstructure)?
  \item \textbf{Evaluation.} How to simulate and measure policies credibly under realistic frictions and dependence?
\end{enumerate}

\noindent Deliverables include a unified modeling stack, a risk-aware staking module, a simulator for strategy evaluation, and a reproducible system-of-systems.

% (Contributions merged into Research Questions; no separate section)

\section{Scope and Boundaries}
We target pre-game markets (spread/total/moneyline) from 1999 onward. Player props and live in-game are out of scope. Only public data are used. All evaluation is out-of-sample with rolling-origin splits and clear decision-time information.

\section{Thesis Statement and Hypotheses}

\subsection{The Research Question}

This dissertation investigates a fundamental question in quantitative sports analytics: \textit{Can rigorous statistical methods and machine learning extract systematic profits from NFL betting markets using publicly available data?}

We develop a complete betting system pipeline—from data ingestion to model training to risk management to evaluation—and test it rigorously on 5,529 games across 21 seasons (2004--2024). The results challenge the initial hypothesis but yield valuable methodological contributions.

\subsection{Initial Thesis}
\textbf{Thesis.} A hybrid stack with explicit uncertainty and governance transforms edge into reliable bankroll growth in NFL markets.

\textbf{Hypotheses.}
\begin{enumerate}
  \item State-space priors improve temporal stability and calibration relative to purely discriminative baselines.
  \item Structured score distributions (Skellam and bivariate Poisson) enable superior pricing of spreads/totals and better teaser/middle planning than direct margin regression.
  \item Market microstructure features (line velocity, cross-book discrepancies) contribute unique signal beyond team-performance covariates.
  \item RL policies trained offline with conservative constraints and posterior-variance gating convert modeling edge into drawdown-aware bankroll growth, supporting the thesis.
\end{enumerate}

\subsection{What We Actually Find}

This dissertation arrives at a more nuanced conclusion than the initial thesis proposed. Our models achieve:
\begin{itemize}
  \item \textbf{Strong calibration}: Brier score = 0.2515 (top tier for NFL prediction models)
  \item \textbf{Positive closing line value}: CLV = +14.9 basis points (beating market closes on average)
  \item \textbf{Temporal stability}: Consistent Brier scores across 2015--2024 (range: 0.2486--0.2511)
\end{itemize}

Yet the system loses money: ROI = $-7.5\%$, Sharpe ratio = $-1.22$.

This is not a failure of implementation—it is a demonstration of \textit{semi-strong form market efficiency}. NFL betting markets efficiently incorporate public information, leaving marginal gains that fall short of the 4.5\% hurdle imposed by vigorish at $-110$ odds.

\paragraph{Reframed Contribution.}
While the initial thesis of profitable betting proves untenable with public data alone, this work makes four methodological contributions:
\begin{enumerate}
  \item \textbf{Rigorous negative results}: Weather has no predictive value; calibration does not imply profitability; RL provides marginal gains over simpler Kelly baselines.
  \item \textbf{Complete system architecture}: A full pipeline from ingestion to evaluation, reusable for alternative domains.
  \item \textbf{Dependence-aware evaluation}: Copula-based methods for correlated outcomes (same-game parlays, teasers).
  \item \textbf{Transparent failure analysis}: Documentation of when and why the system declines to act (21\% zero-bet weeks).
\end{enumerate}

The lesson: Do not confuse model quality with betting viability. Strong calibration is necessary but insufficient for profit in efficient markets.

\section{Reproducibility and Ethics}
Pipelines are containerized and deterministic. Seeds, dataset manifests, and artefacts are versioned. We adopt responsible-gambling principles with exposure caps, volatility limits, and stop-losses; results are presented with uncertainty, not as guarantees.

\section{Chapter Summary}
This chapter motivated the NFL as a fertile ground for rigorous decision analytics and laid out the claims and scope of the work. The remainder of the dissertation builds from data foundations (\Cref{chap:data}) through baselines (\Cref{chap:methods}), risk management (\Cref{chap:risk}), and simulation (\Cref{chap:sim}) to a deployable policy.

\section{Dissertation Structure}

Below is a brief road map of this dissertation:

\begin{itemize}
  \item \textbf{Chapter~2: Literature Review} — survey of classical and modern models in sports prediction, betting markets, RL in game domains.
  \item \textbf{Chapter~3: Data Foundations and Feature Engineering} — discussion of NFL data sources, structure, preprocessing, feature catalogs, era handling.
  \item \textbf{Chapter~4: Baseline Models} — implementation and calibration of GLM, state-space, Poisson, and classical benchmarks.
  \item \textbf{Chapter~5: Reinforcement Learning Framework} — state/action/reward specification, offline RL design, training pipelines.
  \item \textbf{Chapter~6: Uncertainty \& Risk Management} — posterior distributions, risk-aware betting, Kelly strategies, drawdown analysis.
  \item \textbf{Chapter~7: Simulation \& Strategy Testing} — Monte Carlo engines, teasing, parlays, correlated outcomes, strategy performance.
  \item \textbf{Chapter~8: System Architecture \& Governance} — modular pipeline, experiment tracking, deployment strategy, version control.
  \item \textbf{Chapter~9: Results, Ablations, Discussion} — comparative performance, feature ablations, robustness, error analysis.
  \item \textbf{Chapter~10: Conclusion and Future Work} — summary of findings, limitations, and opportunities ahead.
  \item \textbf{Appendices} — extended figures, proofs, code reference, glossaries.
\end{itemize}

\section{Technical Approach Overview}

At a high level, the system will operate in layers:

\begin{itemize}
  \item A data ingestion and feature pipeline — building situational, team-level, market-level features.
  \item Classical and benchmark models (GLM, Poisson, state-space) to form priors and baselines.
  \item An RL agent (e.g.\ DQN, PPO) that takes feature + market state to decide bets or allocations.
  \item Uncertainty propagation (posterior distributions, bootstrap ensembles) to inform risk control.
  \item A simulation engine that translates distributions into actionable betting strategies.
  \item An evaluation and governance layer to compare models, version them, log experiments, and deploy.
\end{itemize}



\section{Glossary of Key Terms}

Here are some terms you’ll see frequently:
\begin{description}
  \item[CLV:] Closing Line Value — difference between market-implied and model-implied edge.
  \item[Kelly Fraction:] Optimal fraction of bankroll to stake given edge and odds.
  \item[Posterior Uncertainty:] Bayesian credible intervals on model predictions.
  \item[Middling / Teasers:] Betting strategies that exploit distributions across markets.
  \item[Feature ‹class›:] A group of inputs, e.g.\ injuries, rest, market signals.
  \item[Ensemble:] A weighted combination of predictions from multiple models.
\end{description}

% (Removed Writing and Style Notes)

\chaptersummary{
This chapter established the motivation, scope, and core contributions of the dissertation. We argued that profitable NFL prediction requires a hybrid stack combining calibrated statistical models with explicit uncertainty quantification and disciplined risk management.
}{
We now survey the foundational literature—canonical models (Harville, Stern, Glickman), score distributions (Poisson, Skellam, bivariate), evaluation frameworks (proper scoring, calibration), and market microstructure—in \Cref{chap:litreview}.
}
