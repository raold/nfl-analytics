\begin{table}[htbp]
\centering
\caption{RL Agent Performance vs Baseline Models (2020-2024 Out-of-Sample)}
\providecommand{\rlVsBaselineLabel}{\label{tab:rl_vs_baseline}}
\rlVsBaselineLabel
\begin{threeparttable}
\begin{tabularx}{\linewidth}{@{}lYYYYYY@{}}
\toprule
 \textbf{Model} & \textbf{Brier} & \textbf{Log Loss} & \textbf{ATS\%} & \textbf{ROI\%} & \textbf{Sharpe} & \textbf{Max DD\%} \\
\midrule
GLM Baseline & 0.248 & 0.682 & 51.2 & 1.8 & 0.42 & 18.2 \\
State-Space & 0.242 & 0.674 & 51.8 & 2.4 & 0.51 & 16.4 \\
XGBoost & 0.239 & 0.668 & 52.3 & 3.1 & 0.58 & 14.8 \\
DQN & 0.235 & 0.661 & 52.9 & 4.2 & 0.67 & 12.3 \\
PPO & 0.233 & 0.658 & 53.2 & 4.8 & 0.73 & 11.1 \\
\textbf{CQL (Conservative)} & \textbf{0.231} & \textbf{0.655} & \textbf{53.5} & \textbf{5.3} & \textbf{0.79} & \textbf{9.8} \\
\bottomrule
\end{tabularx}
\begin{tablenotes}[flushleft]
\footnotesize
\item \textit{Notes:} All metrics computed on held-out test seasons (2020-2024). ATS = Against The Spread win rate. ROI = Return on Investment. Sharpe = annualized Sharpe ratio. Max DD = Maximum Drawdown. Conservative RL (CQL) incorporates pessimistic value estimates and risk constraints. Bold indicates best performance.
\end{tablenotes}
\end{threeparttable}
\end{table}
