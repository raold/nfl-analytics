\begin{table}[htbp]
\centering
\caption{Portfolio Performance Under Different Risk Objectives}
\label{tab:cvar_benchmark}
\begin{threeparttable}
\begin{tabularx}{\linewidth}{@{}lYYYYY@{}}
\toprule
Portfolio & E[R]\% & Vol\% & CVaR$_{95}$\% & CVaR$_{99}$\% & Worst\% \\
\midrule
Equal Weight & 5.2 & 14.3 & -18.2 & -24.7 & -12.3 \\
Min Variance & 3.8 & 8.7 & -11.3 & -15.8 & -7.8 \\
Max Sharpe & 6.4 & 16.2 & -21.4 & -28.3 & -14.7 \\
Risk Parity & 4.7 & 10.1 & -13.2 & -17.9 & -8.9 \\
\textbf{CVaR Optimal} & 5.1 & 11.8 & \textbf{-9.8} & \textbf{-13.4} & \textbf{-6.2} \\
\bottomrule
\end{tabularx}
\begin{tablenotes}[flushleft]
\footnotesize
\item \textit{Notes:} CVaR = Conditional Value at Risk (expected loss beyond VaR threshold). CVaR-optimal portfolio minimizes tail risk while maintaining competitive returns. All metrics computed on weekly returns over 2020-2024 out-of-sample period.
\end{tablenotes}
\end{threeparttable}
\end{table}
