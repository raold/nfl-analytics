\begin{table}[t]
  \centering
  \footnotesize
  \caption{Model performance comparison against published benchmarks. Our stacked ensemble achieves best-in-class calibration (Brier = 0.2515) but fails to overcome market efficiency for profitability (51.0\% ATS vs 52.4\% breakeven).}
  \label{tab:benchmark-comparison}
  \setlength{\tabcolsep}{4pt}
  \begin{tabular}{lccccl}
    \toprule
    \textbf{Model}  & \textbf{Brier} $\downarrow$  & \textbf{ATS \%}  & \textbf{CLV (bps)}  & \textbf{Years}  & \textbf{Notes} \\
    \midrule
    \textbf{Our Ensemble (Stacked)} & 0.252 & 51.0 & 14.9 & 2004-2024 & Best calibration \\
    \textbf{Our Baseline (GLM)} & 0.255 & 50.8 & 6.3 & 2004-2024 & Interpretable \\
    \midrule
    FiveThirtyEight ELO & 0.253 & 50.6 & -- & 2015-2023 & Published benchmark \\
    ESPN FPI & -- & 51.2 & -- & 2015-2023 & Industry standard \\
    PFF Greenline & -- & 52.1 & -- & 2019-2023 & Premium service \\
    Vegas Closing Line & 0.250 & 50.0 & 0.0 & 1985-2024 & Efficiency baseline \\
    Naive (50/50) & 0.250 & 50.0 & -- & N/A & Random baseline \\
    \bottomrule
  \end{tabular}
  \begin{tablenotes}
    \small
    \item \textit{Note:} Brier score measures calibration (lower is better). ATS \% is against-the-spread win rate (52.4\% needed for profitability at standard -110 odds). CLV is closing line value in basis points. FiveThirtyEight and Vegas lines provide strongest external baselines with comprehensive public reporting.
  \end{tablenotes}
\end{table}
