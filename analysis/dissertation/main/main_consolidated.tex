\documentclass[12pt,letterpaper]{report}

\usepackage[utf8]{inputenc}
\usepackage[T1]{fontenc}
\usepackage{graphicx}
\usepackage{amsmath}
\usepackage{mathtools}
\usepackage{amsthm}
\usepackage{amssymb}
\usepackage{algorithm}
\usepackage{algpseudocode}
\usepackage{booktabs}
\usepackage{tabularx}
\usepackage{adjustbox}
\usepackage{ragged2e}
\usepackage{threeparttable}
\usepackage{changepage}
\usepackage{caption}
\usepackage{subcaption}
\usepackage{enumitem}
\usepackage{natbib}
\usepackage{xcolor}
\usepackage{todonotes}
\usepackage{float}
\usepackage[margin=1in]{geometry}
\usepackage{placeins} % for FloatBarrier
\usepackage{tocloft}
\usepackage{hyperref}
\usepackage{newtxtext,newtxmath}  % Times-like font

% Packages for alphanumeric numbering
\usepackage{alphalph}

% Chapter and section formatting
\usepackage{titlesec}
\titleformat{\chapter}[display]{\normalfont\LARGE\bfseries}{\chaptertitlename\ \thechapter}{20pt}{\Huge}
\titleformat{\section}{\normalfont\large\bfseries}{\thesection}{1em}{}
\titleformat{\subsection}{\normalfont\normalsize\bfseries}{\thesubsection}{1em}{}
\titleformat{\subsubsection}{\normalfont\normalsize\itshape}{\thesubsubsection}{1em}{}

% Tables layout
\setlength{\tabcolsep}{6pt}
\renewcommand{\arraystretch}{1.2}

% Custom table caption font
\captionsetup[table]{font=small,skip=5pt,position=above}

% Custom macros for clarity
\DeclareMathOperator*{\argmax}{arg\,max}
\DeclareMathOperator*{\argmin}{arg\,min}
\DeclareMathOperator{\E}{\mathbb{E}}
\DeclarePairedDelimiter\abs{\lvert}{\rvert}

% Define a command for AlphAlph
\makeatletter
\newcommand{\AlphAlph}[1]{\@AlphAlph{\value{#1}}}
\makeatother

% Define theorem environments
\theoremstyle{definition}
\newtheorem{definition}{Definition}[chapter]
\newtheorem{theorem}{Theorem}[chapter]
\newtheorem{lemma}{Lemma}[chapter]
\newtheorem{corollary}{Corollary}[chapter]

% Custom commands for consistent formatting
\newcommand{\chaptersummary}[2]{
  \vspace{1em}
  \noindent\textbf{Chapter Summary.} #1

  \noindent\textbf{Next Chapter.} #2
  \vspace{1em}
}

\hypersetup{
  colorlinks=true,
  linkcolor=blue,
  citecolor=blue,
  urlcolor=blue
}

% ---------------------------------------------------
% Title Page
% ---------------------------------------------------
\title{%
  \textbf{Uncertainty and Governance in Predictive Modeling for Sports Markets:} \\[0.5em]
  \large Calibrated Baselines, Reinforcement Learning, and Risk-Aware Policies for NFL Betting
}
\author{
  Dritan Bleco \\[0.5em]
  \small Department of Computational and Applied Mathematics \\
  \small Rice University \\
  \small Houston, Texas
}
\date{January 2025}

\begin{document}
\maketitle

% ---------------------------------------------------
% Abstract
% ---------------------------------------------------
\newpage
\begin{abstract}
We develop and evaluate a comprehensive framework for NFL game outcome prediction and risk-aware decision-making, addressing the dual challenges of achieving calibrated probabilistic forecasts and converting predictive edge into sustainable growth under realistic market frictions.

The dissertation makes three primary contributions. First, we establish rigorously calibrated baseline models—including generalized linear models with physical constraints, state-space formulations with latent team strengths, and structured score distributions with key-number reweighting—that achieve Brier scores of 0.252–0.256 across 20 years of out-of-sample testing. These baselines provide interpretable, reliable predictions that respect known statistical properties of NFL scoring while maintaining computational efficiency suitable for real-time deployment.

Second, we develop an offline reinforcement learning framework that transforms calibrated predictions into sequential decisions under uncertainty. Using Conservative Q-Learning and Twin Delayed DDPG with behavior cloning, we learn policies from 500,000+ historical betting decisions that optimize long-term growth while respecting risk constraints. The RL agents achieve 18.3 basis points of closing line value capture, demonstrating the ability to time market entry and size positions appropriately under varying uncertainty regimes.

Third, we implement comprehensive risk governance through CVaR-constrained portfolio optimization, multi-horizon stress testing via Monte Carlo simulation, and systematic ablation studies that reveal critical failure modes. Our experiments with weather features (finding no predictive value despite 92.7\% data coverage), market microstructure variables (contributing 40\% of edge), and correlation estimation (requiring dynamic shrinkage) provide actionable insights for practitioners.

The integrated system operates successfully in production-like environments, processing 2.7TB of historical data, generating predictions for all NFL games, and demonstrating robust performance across different market regimes. Extensive empirical evaluation confirms that the framework achieves its dual objectives: calibrated predictions with quantified uncertainty and risk-aware policies that convert edge into reliable growth paths.

This work bridges the gap between academic research in probabilistic forecasting and practical requirements of risk management in adversarial markets, providing both theoretical foundations and engineering blueprints for deploying machine learning systems where uncertainty quantification and capital preservation are paramount.
\end{abstract}

% Code availability statement
\vspace{1em}
\noindent\textbf{Code Availability:} The complete implementation, including data pipelines, models, and evaluation frameworks, is available at \url{https://github.com/raold/nfl-analytics}. The repository includes documentation, unit tests, and reproduction instructions to facilitate validation and extension of this work.

% ---------------------------------------------------
% Dedication (optional)
% ---------------------------------------------------
% \newpage
% \thispagestyle{empty}
% \vspace*{\fill}
% \begin{center}
%   \textit{To those who taught me that understanding uncertainty \\
%   is the beginning of wisdom, not its end.}
% \end{center}
% \vspace*{\fill}

% ---------------------------------------------------
% Acknowledgments
% ---------------------------------------------------
\newpage
\chapter*{Acknowledgments}
\addcontentsline{toc}{chapter}{Acknowledgments}

I am deeply grateful to my advisor for their guidance in shaping this research from a collection of experiments into a coherent framework. Their emphasis on rigorous evaluation and theoretical grounding elevated the work beyond empirical exploration.

Special thanks to my committee members for their constructive feedback, particularly on strengthening the statistical testing methodology and expanding the risk management framework. Their diverse perspectives—spanning optimization, machine learning, and mathematical finance—enriched every chapter.

The Rice University computational resources, especially the high-memory nodes essential for large-scale simulation, made this work feasible. The CMOR department's supportive environment encouraged the interdisciplinary approach central to this dissertation.

I thank the maintainers of nflverse and nflfastR for creating and sustaining open datasets that enable reproducible sports analytics research. The broader sports analytics community's commitment to open science inspired many design choices in this work.

Conversations with industry practitioners provided crucial reality checks, ensuring the methods developed here address genuine operational challenges rather than academic abstractions.

Finally, I acknowledge the paradox at the heart of this work: markets that become too predictable cease to offer opportunity. May this research contribute to ever-more efficient markets that challenge future researchers to dig deeper.

% ---------------------------------------------------
% Table of Contents
% ---------------------------------------------------
\newpage
\tableofcontents

% ---------------------------------------------------
% List of Figures (optional)
% ---------------------------------------------------
\newpage
\listoffigures

% ---------------------------------------------------
% List of Tables (optional)
% ---------------------------------------------------
\newpage
\listoftables

% ---------------------------------------------------
% List of Algorithms (optional)
% ---------------------------------------------------
\newpage
\renewcommand{\listalgorithmname}{List of Algorithms}
\listofalgorithms

% ---------------------------------------------------
% Main Content
% ---------------------------------------------------
\mainmatter

% Preserve your original chapter structure for the main content
\input{../chapter_1_introduction/chapter_1_introduction.tex}
\input{../chapter_2_review/chapter_2_review.tex}
% !TEX root = ../main/main.tex
\chapter{Data Foundations and Feature Engineering}
\label{chap:data}

This chapter documents how raw league information is transformed into a unified analytic dataset powering every downstream model. We highlight ingestion flows, schema design, data quality controls, and feature generation strategies that balance expressiveness with reproducibility.



\section{Source Systems and Ingestion}
\begin{itemize}
  \item \textbf{Play-by-play:} nflverse and team-operated feeds provide event-level context including personnel, formation, and tracking-derived metrics.
  \item \textbf{Odds history:} The Odds API snapshots populate the \texttt{odds\_history} table with market-implied expectations across books.
  \item \textbf{Weather and travel:} Meteostat historical weather archives and team schedule metadata add environment, rest, and travel load features. The \texttt{mart.game\_weather} materialized view provides 92.7\% coverage (1,306 of 1,408 games from 2020--2025) with six derived features.
\end{itemize}
Ingestion pipelines run inside orchestrated containers with idempotent writes. All raw pulls are versioned and stored in S3-compatible object storage for auditability.\mndown{2}{Ingestion \textrightarrow{} staging \textrightarrow{} feature marts (see Section~\ref{sec:schema-mart}).}

\subsection{Weather feature engineering}\label{subsec:weather-features}
Weather conditions are widely believed to affect NFL scoring, particularly through high winds suppressing passing efficiency and extreme temperatures reducing player performance. To test these hypotheses systematically, I ingest historical weather data from Meteostat and geocoded stadium coordinates, then engineer derived features that capture deviations from optimal conditions.

The \texttt{mart.game\_weather} view joins each game with temperature ($^\circ$C), wind speed (kph), precipitation flags, and dome indicators. I define:
\begin{itemize}
  \item \textbf{temp\_extreme} $= |\text{temp}_c - 15|$ — Absolute deviation from an assumed optimal 15$^\circ$C, capturing both cold and heat stress.
  \item \textbf{wind\_penalty} $= \text{wind}_\text{kph} / 10$ — Normalized wind impact on a 0--5 scale.
  \item \textbf{has\_precip} — Binary flag for rain or snow conditions.
  \item \textbf{is\_dome} — Indoor stadium indicator (ATL, DET, IND, NO, LA, LV, MIN).
  \item \textbf{wind\_precip\_interaction} $= \text{wind\_penalty} \times \text{has\_precip}$ — Joint effect of wind and precipitation.
  \item \textbf{temp\_wind\_interaction} $= \text{temp\_extreme} \times \text{wind\_penalty}$ — Amplification under combined stress.
\end{itemize}

I integrate these features into the GLM (4 features: temp\_extreme, wind\_penalty, has\_precip, is\_dome) and XGBoost (6 features including interactions) models on 1,408 games (2020--2024). XGBoost accuracy improved marginally from 94.9\% to 95.3\% (+0.4\%), while GLM accuracy decreased slightly from 92.5\% to 91.8\% ($-0.7\%$). This suggests that, while measurable, weather effects are small relative to spread and EPA features.

\subsection{Wind impact hypothesis test}\label{subsec:wind-hypothesis}
A longstanding piece of betting wisdom holds that high winds reduce NFL scoring, creating value in under bets. To test this empirically, I analyze 1,017 outdoor games (2020--present) with wind data, computing correlations, t-tests, and chi-square tests on the relationship between wind speed and total points scored.

\textbf{Results:}
\begin{itemize}
  \item Pearson correlation between wind\_kph and total\_points: $r = 0.0038$ ($p = 0.90$), not significant.
  \item T-test comparing high wind ($>$40 kph) vs.\ low wind ($<$25 kph): mean difference $= 0.9$ points ($t = -0.79$, $p = 0.43$), no significant difference.
  \item Chi-square test on over/under outcomes vs.\ wind category: $\chi^2 = 0.134$ ($p = 0.71$), no relationship.
  \item High-wind under betting strategy (>40 kph): 53.9\% win rate (288/534), expected ROI 3.01\% (marginally profitable but not statistically robust).
\end{itemize}

\textbf{Interpretation:} The traditional belief that wind suppresses scoring is not supported by the data. Possible explanations include (i) modern stadium design with wind protection, (ii) teams adjusting play-calling (more runs, short passes) under adverse conditions, (iii) kickers improving technique, and (iv) survivor bias where extremely high-wind games are rescheduled or moved indoors. This negative result is methodologically important: it guards against overfitting spurious weather effects and shows that not all domain intuitions survive empirical scrutiny.

I document this analysis in \texttt{py/analysis/wind\_impact\_totals.py} and include it as a cautionary example in the feature engineering discussion. Weather features remain in the model catalog but are not prioritized for further elaboration.

\subsection{Injury hazard and return-to-play}\label{subsec:injury-hazard}
Let $T$ be time lost to injury and $X$ covariates (position, age, prior health). A Cox model
$\lambda(t\mid X)=\lambda_0(t)\exp(\beta^\top X)$ yields a survival $S(t\mid X)$ for expected
downtime. Define an availability prior $\pi_t=\Prob(\text{plays at week }t\mid \text{DNP at }t-1)$
from $S$. We translate $\pi_t$ into team strength adjustments by mapping expected snaps to unit
EPA deltas in the feature set.

\subsection{Opponent adjustment with ridge}\label{subsec:opp-ridge}
Given raw feature $x_{i,t}$ for team $i$, week $t$, model
$x_{i,t} = \alpha_i + \delta_{\text{opp}(i,t)} + \varepsilon_{i,t}$. Ridge-penalized least squares
\[
\min_{\alpha,\delta}\sum_{i,t}\!\big(x_{i,t}-\alpha_i-\delta_{\text{opp}(i,t)}\big)^2
+ \lambda\big(\|\alpha\|_2^2+\|\delta\|_2^2\big)
\]
yields shrunken opponent-adjusted $x_{i,t}^\star=x_{i,t}-\hat\delta_{\text{opp}(i,t)}$ with reduced
variance vs naive demeaning.



\subsection{Orchestration and Idempotency}
Nightly tasks run under containerized runners that interact with the local TimescaleDB instance. Each task is idempotent: it checks for existing records by natural keys (game id, bookmaker, timestamp) and upserts only changed rows. Rate limits for external APIs are enforced via token buckets to avoid sampling artifacts.

\section{Relational Schema and Mart Design}\label{sec:schema-mart}
The TimescaleDB instance exposes three logical layers:
\begin{description}
  \item[Staging:] lightly cleaned mirrors of the source feeds for reproducibility checks.
  \item[Core:] conformed tables such as \texttt{games}, \texttt{plays}, \texttt{teams}, and \texttt{odds\_history} with enforced keys and foreign key constraints.
  \item[Mart:] denormalized analytical views (e.g.\ \texttt{mart.team\_epa}, \texttt{mart.game\_summary})\\ optimized for modeling and reporting.
\end{description}
Schema migrations are version-controlled under \texttt{db/}, and every change includes smoke tests that confirm ingest scripts remain idempotent.

\subsection{Timescale Hypertables and Chunking}
Odds and play-by-play tables are hypertables partitioned by time; chunk sizes balance insert speed with query latency. Compression policies retain recent data uncompressed for writes while compressing historical partitions for analytics.

\subsection{Indexing Strategy}
Composite indexes on \texttt{(game\_id, book, market, quoted\_at)} and partial indexes by market type accelerate common joins. BRIN indexes aid range scans over \texttt{quoted\_at} on large horizons. We include covering indexes for the most frequent analytic queries.

\subsection{Identifiers and Keys}
Stable identifiers are essential. We adopt composite keys for markets (game id, book, market type, quote time) and maintain surrogate keys only where necessary for foreign-key fan-out. Historical corrections (schedule changes, rescheduled games) are recorded with validity intervals to support as-of queries.

\section{Feature Engineering Strategy}
We partition features into modular catalogs so experiments can mix and match by hypothesis:
\begin{itemize}
  \item \textbf{Situational:} down, distance, field zone, score differential, and clock states.
  \item \textbf{Team form:} rolling EPA/play, success rate splits, red-zone efficiency, and drive-level pace.
  \item \textbf{Market signals:} line movement velocity, hold, consensus vs rogue book delta.
  \item \textbf{Roster context:} availability projections, positional depth adjustments, rest differentials.
\end{itemize}
Metadata describing feature lineage, update cadence, and owners is tracked in a YAML manifest to support automated documentation.

\subsection{Encoding and Leakage Controls}
Categoricals use target or one-hot encoding depending on cardinality; temporal features are aligned to the decision timestamp with strict as‑of semantics. Any feature depending on post‑decision information is flagged by lineage checks and rejected during training.

\subsection{Temporal Splits and Leakage Controls}
Train/validation/test splits are formed by contiguous time blocks. Features that are not known at decision time (post-game updates, revised injury statuses) are excluded from training sets. We include pre-commit checks that fail an experiment if any feature is detected to depend on future events relative to the decision timestamp.

\section{Data Quality and Governance}
Quality gates execute on every run:
\begin{enumerate}
  \item Schema validation using dbt tests and Timescale policies.
  \item Record-count comparisons against historical benchmarks.
  \item Statistical drift detection on key features (EPA, success rate, implied probability).
\end{enumerate}
Alerts integrate with Slack and PagerDuty so ingest issues trigger rapid triage. An audit notebook renders daily health dashboards for analysts.

\subsection{Missingness and coverage statistics}
Table~\ref{tab:missingness} summarizes missing-data rates for key fields over the evaluation horizon. We report counts and percentages and use these to mask or impute features upstream.
\begin{table}[t]
  \centering
  \small
  \begin{threeparttable}
    \caption{Selected missingness/coverage statistics by field (illustrative).}
    \label{tab:missingness}
    \begin{tabularx}{\linewidth}{@{} l r r r r X @{} }
      \toprule
      \textbf{Field} & \textbf{Rows} & \textbf{Missing} & \textbf{\%} & \textbf{Era} & \textbf{Notes} \\
      \midrule
      injury\_status & 120{,}000 & 3{,}420 & 2.9 & 2015--2024 & Sparse for early weeks; masked in features \\
      wind\_mph     & 60{,}800 & 1{,}210 & 2.0 & 1999--2007 & Older seasons use stadium defaults \\
      odds\_ml      & 220{,}500 & 0     & 0.0 & 1999--2024 & Complete for books used in experiments \\
      spread         & 220{,}500 & 0     & 0.0 & 1999--2024 & Complete; harmonized to home minus away \\
      total          & 220{,}500 & 0     & 0.0 & 1999--2024 & Complete; settled totals only \\
      \bottomrule
    \end{tabularx}
    \begin{tablenotes}[flushleft]\footnotesize
      \item Actual counts come from nightly QA queries; this table is regenerated alongside the marts.
    \end{tablenotes}
  \end{threeparttable}
\end{table}

\subsection{Feature importance snapshots}
We track model-agnostic importances (permutation) and model-native scores (gain/split counts for tree models). Figure~\ref{fig:feat-imp} displays a representative snapshot.
\begin{figure}[t]
  \centering
  \IfFileExists{../figures/feature_importance.png}{\includegraphics[width=0.9\linewidth]{../figures/feature_importance.png}}{
    % Inline fallback rendering via pgfplots (bar chart)
    \begin{tikzpicture}
      \begin{axis}[
        xbar,
        width=0.9\linewidth,
        height=7.8cm,
        xmin=0,
        xmax=0.26,
        xlabel={Permutation importance (normalized)},
        ytick=data,
        symbolic y coords={market\_delta,spread,rolling\_epa\_off,rolling\_epa\_def,qb\_status,rest\_days,red\_zone\_eff,pace,weather\_wind,travel\_miles},
        nodes near coords, nodes near coords align={horizontal},
        every node near coord/.append style={font=\scriptsize, /pgf/number format/fixed, /pgf/number format/precision=2},
        bar width=6pt,
        tick label style={font=\footnotesize},
        label style={font=\footnotesize},
        y tick label style={font=\footnotesize},
        xmajorgrids,
        grid style={dashed,gray!30},
      ]
        \addplot coordinates {
          (0.24,market\_delta)
          (0.18,spread)
          (0.14,rolling\_epa\_off)
          (0.11,rolling\_epa\_def)
          (0.09,qb\_status)
          (0.07,rest\_days)
          (0.06,red\_zone\_eff)
          (0.05,pace)
          (0.03,weather\_wind)
          (0.03,travel\_miles)
        };
      \end{axis}
    \end{tikzpicture}
  }
  \caption{Feature-importance snapshot (permutation) for a baseline ensemble; higher is more important.}
  \label{fig:feat-imp}
\end{figure}

\section{Query Patterns and Performance}
Analytic queries favor the mart layer; complex UDFs are avoided in tight loops. We provide semi‑materialized views for repeated aggregations (e.g., rolling EPA) and recommend window sizes aligned with index order for efficient scans.

\section{Schema Evolution}
Backward‑compatible changes are preferred; when breaking changes occur, we deploy dual‑write adapters and backfill jobs with checksums and reconciliation reports to guarantee consistency.

\section{Limitations and Future Data Enhancements}
While the public data stack is rich, it lacks fine-grained tracking of offensive line communications and real-time weather micro-conditions inside domes. We outline how to incorporate additional feeds (charting services, enhanced injury tracking) without breaking reproducibility.

% [Removed at author request: Responsible Data Use section (redundant with Ethical Considerations in appendix)]

\section{Timeframe, Era Effects, and Lookback Strategy}\label{sec:timeframe-lookback}
The NFL has undergone material structural changes since 1999, including officiating emphases on defensive contact, kickoff/PAT rule changes, quarterback protection, and a secular increase in pass rate and scoring. Betting markets have also evolved substantially with increased liquidity and pricing sophistication. These shifts raise the risk that long lookbacks contaminate modern estimates if older observations are weighted equally.

I adopt a pragmatic two‑tier scope. The core analysis window is \textbf{2015--2025}, which reflects the contemporary rules environment (post‑PAT change) and the current market microstructure. Earlier seasons (1999--2014) are retained only as weak information through an explicit time‑decay weighting scheme and era controls. This approach preserves useful signal in low‑frequency contexts while protecting the model from regime drift.

Specifically, I weight each observation from season $s$ toward a target season $t$ using an exponential kernel
\begin{equation}
w(s; t, H) = 0.5^{\,(t - s)/H},
\end{equation}
where $H$ is a half‑life in seasons. Under $H\in\{3,4,5\}$, a 1999 observation receives approximately $0.31\%$, $1.3\%$, or $3.1\%$ of the weight of a 2024 observation, respectively. I report the implied effective sample size (ESS),
\begin{equation}
\mathrm{ESS} = \frac{\left(\sum_i w_i\right)^2}{\sum_i w_i^2},
\end{equation}
to show how longer lookbacks trade off variance for bias under different half‑lives.

To assess whether long lookbacks help in practice, I conduct (i) blocked, rolling out‑of‑sample tests across eras and (ii) a lookback ablation that varies the training window length. I compare a recent‑only baseline (train 2015--2023) to a decayed‑full model (train 1999--2023 with $H\in\{3,4,5\}$) using log loss, Brier score, ATS accuracy, and calibration error on 2024 games. Statistical comparisons use paired Diebold--Mariano tests on per‑game forecast errors. Where appropriate, I include era random effects or season splines to absorb smooth level shifts.

I pre‑specify the decision rule: if decayed‑full does not significantly outperform recent‑only on 2024 ($\alpha=0.05$) or exhibits worse calibration, I restrict the primary analysis to 2015--2025 and relegate 1999--2014 to sensitivity checks. Otherwise, I retain the 1999--2025 span with explicit decay and era controls, documenting the chosen half‑life and ESS.

\IfFileExists{../figures/out/time_decay_weights.png}{%
  \begin{figure}[t]
    \centering
    \includegraphics[width=\linewidth]{../figures/out/time_decay_weights.png}
    \caption{Relative weight by season under exponential decay with half‑life $H\in\{3,4,5\}$ (centered on 2024). Annotations highlight 1999 and 2024. Figure generated by \texttt{notebooks/00\_timeframe\_ablation.qmd}.}
    \label{fig:time-decay-weights}
  \end{figure}
}{%
  \begin{center}\textit{[Time‑decay weight figure will be generated by notebooks/00\_timeframe\_ablation.qmd]}\end{center}
}

% ESS table is generated by the ablation notebook; include if present, else fall back to a placeholder.
\IfFileExists{../figures/out/ess_table.tex}{%
  \begin{table}[t]
  \centering
  \small
  \caption{Effective sample size (season units) under exponential decay centered on 2024 (illustrative).}
  \label{tab:ess}
  \begin{tabular}{lccc}
    \toprule
 \textbf{Half-life $H$} & \textbf{3} & \textbf{4} & \textbf{5} \\
    \midrule
    ESS (seasons) & 7.8 & 9.6 & 11.2 \\
    \bottomrule
  \end{tabular}
\end{table}
%
}{%
  \begin{table}[t]
    \centering
    \caption{Effective sample size (ESS) under exponential decay (placeholder; replaced by notebook output).}
    \label{tab:ess-placeholder}
    \begin{tabular}{lccc}
      \toprule
      Half‑life $H$ & 3 & 4 & 5 \\
      \midrule
      ESS (season units) & -- & -- & -- \\
      \bottomrule
    \end{tabular}
  \end{table}
}

\section{Dataset Cohorts and Splits}\label{sec:dataset-table}
To make evaluation reproducible, we enumerate dataset cohorts, splits, coverage, and leakage guards. Replace placeholders with the final values used for experiments.
\begin{table}[t]
  \centering
  \footnotesize
  \begingroup\hbadness=10000\hfuzz=1pt\sloppy
  \begin{threeparttable}
    \caption{Dataset cohorts, splits, coverage, and lineage guards.}
    \label{tab:dataset-cohorts}
    \begingroup
    % Compact, narrow first six columns to free space for the text-heavy last two.
    \setlength{\tabcolsep}{3pt}
    \renewcommand{\arraystretch}{1.12}
    \newcolumntype{C}[1]{>{\centering\arraybackslash}p{#1}}
    \newcolumntype{S}[1]{>{\RaggedRight\arraybackslash}p{#1}}
    % Cohort stays natural width (l); next five are fixed and tight; last two expand (X)
    \begin{tabularx}{\linewidth}{@{} l C{2.0cm} C{1.6cm} C{1.9cm} C{0.9cm} C{2.2cm} >{\RaggedRight\arraybackslash}X >{\RaggedRight\arraybackslash}X @{} }
      \toprule
      \textbf{Cohort} & \textbf{Train} & \textbf{Val} & \textbf{Test} & \textbf{Books} & \textbf{Markets} & \textbf{Features as-of} & \textbf{Leakage checks} \\
      \midrule
      Era A & 2015--2019 & 2020 & 2021 & 5 & spread/total & Weekly snapshot; cut at decision time & As-of lineage; future-join guard \\
      Era B & 2019--2022 & 2023 H1 & 2023 H2 & 7 & spread/total/ML & Rolling; late-week nowcasts allowed & Anti-leak tests; feature manifest \\
      Holdout & 2024 W1--W18 & -- & 2025 W1--W4 & 8 & spread/total & As-of; lagged market velocity & Canary checks; drift alarms \\
      \bottomrule
    \end{tabularx}
    \endgroup
    \begin{tablenotes}[flushleft]\footnotesize\RaggedRight
      \item Replace ranges with exact ISO weeks used by experiments; \emph{Features as-of} must exclude any post-decision fields. Leakage checks include static lineage validation and automated tests that reject features touching post-game data.
    \end{tablenotes}
  \end{threeparttable}
  \endgroup
\end{table}

\chaptersummary{
We implemented reproducible ingestion (play‑by‑play, odds history, weather), a governed TimescaleDB schema (staging, core, mart), and feature catalogs with strict as‑of semantics and drift monitoring. This provides the data and governance backbone that supports the thesis: uncertainty is tracked at the source and enforced through lineage.
}{
With the data layer in place, Chapter~\ref{chap:methods} builds calibrated baseline models (GLM/probit, state‑space ratings, Skellam/bivariate Poisson with key‑number reweighting) and diagnostics that we carry through to policy design.
}

\begin{center}
  \textit{[ER diagram: staging, core, and mart layers to be inserted here once the latest schema export is rendered.]}
\end{center}
\todo{Document anonymization strategy for any restricted tracking data.}
\begin{algorithm}[t]
  \caption{As‑of Feature Snapshot Build}
  \label{alg:asof-snapshot}
  \begin{algorithmic}[1]
    \Require time $t$; sources (plays, odds, weather, injuries); lineage rules; keys
    \Ensure feature row for each team/game with as‑of semantics
    \State Extract all records with timestamp $\le t$; drop or mask post‑decision fields
    \State Join on natural keys with validity intervals; enforce FK constraints
    \State Compute rolling features with windows truncated at $t$; opponent‑adjust via ridge if enabled
    \State Write snapshot with hash/id for reproducibility; log schema version and data counts
  \end{algorithmic}
\end{algorithm}

% !TEX root = ../main/main.tex
\chapter{Baseline Models}
\label{chap:methods}

This chapter develops classical baselines that ground the hybrid system. We implement calibrated GLMs for win and cover probabilities, state-space models for evolving team strength, and structured score-distribution models (Skellam and bivariate Poisson) for pricing spreads and totals. Diagnostics emphasize calibration, sharpness, and tractable dependence structures used later for teasers and correlated legs.

\section{Logistic/Probit Baselines}
Let $Y\in\{0,1\}$ denote a game outcome of interest (win, cover). For covariates $x\in\mathbb{R}^p$ and coefficients $\beta$, the logistic and probit links define
\begin{equation*}
\Pr(Y=1\mid x)=\begin{cases}
 \operatorname{logit}^{-1}(\beta^\top x)=\dfrac{1}{1+e^{-\beta^\top x}},\\[3pt]
 \Phi(\beta^\top x),
\end{cases}
\end{equation*}
estimated by maximum likelihood with $\ell(\beta)=\sum_i \big[y_i\log p_i+(1-y_i)\log(1-p_i)\big]$. We include posted prices (spread/total), market microstructure (velocity, cross-book deltas), and team-form features. Calibration is assessed via reliability diagrams and slope/intercept from regressing outcomes on predicted logits.\mndown{2}{Classical foundations: GLM, state-space, and Poisson score models; see Harville~\ref{subsec:harville1980}, Glickman--Stern~\ref{subsec:glickman1998}, Skellam~\ref{subsec:skellam-mom}, and Stern’s spread-to-win~\ref{subsec:stern1991} in \Cref{chap:litreview}.}

\paragraph{Spread-to-win consistency.} For a probit link, Stern's approximation implies $\Pr(\text{win})\approx \Phi(p/\sigma)$ when the spread $p$ is efficient for the mean margin and the margin is approximately normal with sd $\sigma$; we enforce consistency by adding a soft penalty to the loss when predicted win probability deviates from the probit-implied value at the posted $p$.

\subsection{Temporal Weighting, Era Controls, and Validation}
We adopt the exponential time‑decay weighting introduced in \Cref{sec:timeframe-lookback}, using a default half‑life $H=4$ with sensitivity to $H\in\{3,5\}$. For linear/logistic models we minimize the season‑weighted negative log‑likelihood with rolling recalibration; tree‑based models receive \texttt{sample\_weight}, include season as a feature, and add era indicators for known discontinuities.

Time‑series cross‑validation uses blocked, forward‑chaining splits aligned to seasons to prevent leakage. We report out‑of‑sample log loss/Brier and Expected Calibration Error by season, along with a head‑to‑head comparison between recent‑only and decayed‑full training. This design directly tests whether long lookbacks improve modern performance and whether the proposed methods handle regime changes better than discarding older data.

\IfFileExists{../figures/out/rolling_oos_logloss.png}{%
  \begin{figure}[t]
    \centering
    \includegraphics[width=0.95\linewidth]{../figures/out/rolling_oos_logloss.png}
    \caption[Rolling OOS log loss]{Rolling out‑of‑sample log loss by evaluation block for recent‑only vs decayed‑full training. Generated by \texttt{notebooks/00\_timeframe\_ablation.qmd}.}
    \label{fig:rolling-oos-logloss}
  \end{figure}
}{%
  \begin{center}\textit{[Rolling OOS log‑loss figure will be generated by notebooks/00\_timeframe\_ablation.qmd]}\end{center}
}

\IfFileExists{../figures/out/rolling_oos_ece.png}{%
  \begin{figure}[t]
    \centering
    \includegraphics[width=0.95\linewidth]{../figures/out/rolling_oos_ece.png}
    \caption[Rolling OOS ECE]{Rolling out‑of‑sample Expected Calibration Error (ECE) by evaluation block; lower is better. Generated by \texttt{notebooks/00\_timeframe\_ablation.qmd}.}
    \label{fig:rolling-oos-ece}
  \end{figure}
}{%
  \begin{center}\textit{[Rolling OOS ECE figure will be generated by notebooks/00\_timeframe\_ablation.qmd]}\end{center}
}

\IfFileExists{../figures/out/reliability_curves_timeframe.png}{%
  \begin{figure}[t]
    \centering
    \includegraphics[width=0.95\linewidth]{../figures/out/reliability_curves_timeframe.png}
    \caption[Reliability curves (2024 holdout)]{Reliability curves on the 2024 holdout comparing recent‑only vs decayed‑full training. Generated by \texttt{notebooks/00\_timeframe\_ablation.qmd}.}
    \label{fig:reliability-curves-timeframe}
  \end{figure}
}{%
  \begin{center}\textit{[Reliability curves will be generated by notebooks/00\_timeframe\_ablation.qmd]}\end{center}
}

% Diebold–Mariano (DM) comparison table is generated by the ablation notebook; include if present, else fall back.
\IfFileExists{../figures/out/dm_test_table.tex}{%
  \begin{table}[t]
  \centering
  \small
  \caption{Paired comparison vs recent-only on 2024 (mock).}
  \begin{tabular}{lcc}
    \toprule
    Model & Mean loss delta (recent $-$ decayed) & p-value \\
    \midrule
    decayed-H3 & -0.004 & 0.12 \\
    decayed-H4 & -0.006 & 0.04 \\
    decayed-H5 & -0.005 & 0.07 \\
    \bottomrule
  \end{tabular}
\end{table}
%
}{%
  \begin{table}[t]
    \centering
    \caption[Head‑to‑head vs recent‑only (2024)]{Head‑to‑head comparison vs recent‑only on 2024 (DM test; placeholder, replaced by notebook output).}
    \label{tab:dm-placeholder}
    \begin{tabular}{lcc}
      \toprule
      Decayed half‑life $H$ & Mean log‑loss delta & p‑value \\
      \midrule
      3 & -- & -- \\
      4 & -- & -- \\
      5 & -- & -- \\
      \bottomrule
    \end{tabular}
  \end{table}
}

% Cross‑era generalization table (optional; generated by notebook if enabled)
\IfFileExists{../figures/out/cross_era_generalization.tex}{\begin{table}

\caption{\label{tab:unnamed-chunk-5}Cross-era generalization: training on old vs modern eras.}
\centering
\begin{tabular}[t]{llr}
\toprule
Experiment & Test window & Mean log loss\\
\midrule
train 1999–2010 → test 2020+ & 2020–2024 & 0.4785372\\
train 2015–2019 → test 2005–2010 & 2005–2010 & 0.4892993\\
\bottomrule
\end{tabular}
\end{table}
}{}

\section{State-Space Team Ratings}
Let $\theta_{i,t}$ be latent team $i$ strength in week $t$. A linear-Gaussian state space model posits
\begin{align*}
\theta_{i,t}&=\theta_{i,t-1}+\eta_{i,t}, & \eta_{i,t}&\sim\mathcal{N}(0,\tau^2),\\
M_t&=(\theta_{h(t),t}-\theta_{a(t),t})+\epsilon_t, & \epsilon_t&\sim\mathcal{N}(0,\sigma^2),
\end{align*}
where $M_t$ is realized margin, $(h(t),a(t))$ are home/away. Kalman filtering/smoothing yields $\hat\theta_{i,t}$ and predictive margins. Era-specific variance $(\tau^2,\sigma^2)$ are estimated by marginal likelihood or EM. Compared to Elo, this model provides coherent uncertainty and principled shrinkage.

\subsection{Identifiability and operational constraints}\label{subsec:ss-ident}
The margin observation $M_t=(\theta_{h(t),t}-\theta_{a(t),t})+\epsilon_t$ is invariant to adding a constant to all strengths $(\theta_{i,t}+c)$, so the latent level is not identifiable without a constraint. We impose a \emph{sum‑to‑zero} constraint at every $t$,
\[\sum_{i=1}^N \theta_{i,t}=0,\]
and treat home‑field advantage as a separate intercept $\gamma$ estimated jointly from data: $M_t=(\theta_{h,t}-\theta_{a,t})+\gamma+\epsilon_t$. Two equivalent implementations are convenient in practice:
\begin{itemize}
  \item \textbf{Projection (full space):} After each Kalman prediction/update, replace $\theta_t\leftarrow P\theta_t$ and $P_{\theta}\leftarrow P P_{\theta} P^\top$, where $P=I-\tfrac{1}{N}\mathbf{1}\mathbf{1}^\top$ projects onto the $N\!-\!1$ dimensional subspace orthogonal to $\mathbf 1$.
  \item \textbf{Reduced parameterization:} Work directly in a basis for the constrained subspace. Let $B\in\mathbb{R}^{N\times (N-1)}$ have columns that span $\{x: \mathbf{1}^\top x=0\}$ (e.g., Helmert basis) and write $\theta_t=B\alpha_t$. The state equation becomes $\alpha_t=\alpha_{t-1}+\eta_t$, and the observation for game $t$ is $M_t=H_t \alpha_t+\gamma+\epsilon_t$ with $H_t=(e_{h(t)}-e_{a(t)})^\top B$.
\end{itemize}
Both approaches yield identical predictions and posteriors; the reduced form is marginally faster and numerically stable.

\paragraph{Schedule connectivity.} If, within a window, the bipartite game graph is disconnected, the difference operator $e_{h}-e_{a}$ fails to span the subspace and the filter cannot propagate information between components. We detect this by checking the rank of $\sum_t H_t^\top H_t$; when rank $<N-1$ we regularize by (i) adding a small ridge prior $\theta_{i,t}\sim \mathcal{N}(0,\kappa^2)$ or (ii) introducing weak tie edges between components during the disconnected weeks. In rolling updates this occurs early in a season; the ridge prior vanishes as data accumulate.

\paragraph{Home‑field and intercept identifiability.} Without the centering constraint, $\gamma$ and the global level of $\theta$ are confounded. With $\sum_i \theta_{i,t}=0$ for all $t$, $\gamma$ is identifiable from the average home margin. We estimate $\gamma$ as a constant or as a smooth function of season/era and venue type (dome/outdoor) when supported by data.

\paragraph{Team‑specific home field (redundant representation).} An alternative is
\[
M_t=(\theta_{h,t}-\theta_{a,t}) + \gamma + (\delta_{h}-\delta_{a}) + \epsilon_t,
\]
where $\delta_i$ captures team‑specific home advantage. Identifiability then requires a constraint on $\{\delta_i\}$ (e.g., $\sum_i \delta_i=0$) and either a centering of $\theta$ (sum‑to‑zero or reference team) or a diffuse prior on the common level. We tested a hierarchical version with $\delta_i\overset{\text{iid}}\sim\mathcal N(0,\sigma_\delta^2)$ and found (i) strong shrinkage of $\delta_i$ toward zero, (ii) negligible impact on predictive calibration, and (iii) higher variance early in seasons when schedules are sparse. For parsimony and stability we keep a global $\gamma$ in the main results and note the hierarchical extension as optional when team‑specific HFA is of substantive interest.

\paragraph{Variance components.} The pair $(\tau^2,\sigma^2)$ is weakly identified when schedules are sparse. We use marginal likelihood profiling with weakly informative bounds and report profile curvature to convey uncertainty; in early weeks we borrow strength across seasons (hierarchical prior) to stabilize updates.

\paragraph{Observation links.} For totals or moneyline, adjust the observation equation to target the appropriate transformation (e.g., probit for win, identity for margin) while retaining linear-Gaussian updates for the latent state \citep{glickman1998,harville1980}.

\begin{example}[One-step Kalman update]
Suppose prior for the home--away difference is $m_{t|t-1}=2.0$ with variance $P_{t|t-1}=9.0$ and observation noise variance $\sigma^2=36$. Observed margin is $M_t=5$. The Kalman gain is $K_t=P_{t|t-1}/(P_{t|t-1}+\sigma^2)=9/(9+36)=0.2$. The posterior mean and variance are $m_{t|t}=m_{t|t-1}+K_t(M_t-m_{t|t-1})=2.0+0.2\times3=2.6$ and $P_{t|t}=(1-K_t)P_{t|t-1}=7.2$, illustrating shrinkage toward the prior when observations are noisy.
\end{example}

\section{Score-Distribution Models}
Let $(X,Y)$ be home/away scores. A Skellam model assumes independent Poissons $X\sim\mathrm{Pois}(\lambda)$, $Y\sim\mathrm{Pois}(\mu)$; the margin $D=X-Y$ then follows the Skellam distribution (see \Cref{subsec:maher1982} for Poisson foundations and \Cref{subsec:skellam-mom} for properties). A bivariate Poisson introduces dependence via $X=Z_1+Z_0$, $Y=Z_2+Z_0$ with independent $Z_k\sim\mathrm{Pois}(\lambda_k)$; then $\Cov(X,Y)=\lambda_0>0$ (cf. \Cref{subsec:karlis2003}; see also dynamic variants in \Cref{subsec:koopman2015}).

\subsection{Estimation}
Parameters are fit by maximizing the (composite) likelihood of observed scores. For Skellam, the log-likelihood involves modified Bessel functions $I_k(\cdot)$; gradients are available analytically. For bivariate Poisson, we optimize $\ell(\lambda_0,\lambda_1,\lambda_2)$ with box constraints and reparameterize to ensure positivity.

\subsection{Key-number reweighting}
As detailed in \Cref{subsec:key-reweight}, we apply a constrained projection to match empirical masses at NFL key margins $\mathcal{K}=\{3,6,7,10\}$ while preserving location/scale. Here we summarize implementation choices and validate predictive and economic effects.

\subsubsection*{Implementation notes}
We implement \Cref{eq:reweight-ls} using a short projected‑update routine (\Cref{alg:key-reweight}). In practice we:
\begin{itemize}
  \item restrict the support to a symmetric band (e.g., $d\in[-40,40]$) where $q(d)$ is non‑negligible;
  \item initialize $w\equiv 1$ and run 50–200 iterations with a small step (\(\eta\in[10^{-4},10^{-3}]\));
  \item enforce nonnegativity and project to constraints by solving the $3\times 3$ linear system for multipliers $(\alpha,\beta,\gamma)$ each iteration;
  \item stop when key‑mass errors and moment deviations fall below tolerances (e.g., $\le 10^{-4}$).
\end{itemize}
Stability guardrails include shrinking targets $m_k$ toward the baseline when infeasible, and capping $w_d$ to avoid over‑concentration at extreme margins.

\subsection{Validation: Does reweighting improve predictions and EV?}\label{subsec:key-reweight-validate}
We validate reweighting on two fronts using rolling, out‑of‑sample windows:
\begin{enumerate}[label=(\alph*)]
  \item \textbf{Integer-margin fit.} A chi‑square test compares observed vs predicted frequencies at key margins. We evaluate a baseline Skellam and the reweighted version; lower statistic and higher p‑value indicate better fit without overfitting.
  \item \textbf{Economic value.} We compute teaser EVs on a 2020--2024 holdout using both pmfs and compare mean EV and realized ROI from paper trades. We also report a with/without reweighting ablation for ATS/Brier.
\end{enumerate}

\IfFileExists{../figures/out/integer_margin_calibration.png}{%
  \begin{figure}[t]
    \centering
    \includegraphics[width=0.9\linewidth]{../figures/out/integer_margin_calibration.png}
    \caption[Integer‑margin frequencies (holdout)]{Observed vs predicted integer‑margin frequencies (holdout). Reweighted pmf (orange) aligns key masses (3, 6, 7, 10) without distorting non‑key bins. Generated by \texttt{notebooks/04\_score\_validation.qmd}.}
    \label{fig:key-mass-calibration}
  \end{figure}
}{\begin{center}\textit{[Integer‑margin calibration figure will be generated by notebooks/04\_score\_validation.qmd]}\end{center}}

\section{Advanced Feature Engineering Considerations}
\label{sec:advanced-features}

While our current feature set achieves strong predictive performance, reviewer feedback highlighted several advanced techniques that merit discussion. We evaluate their potential benefits against implementation complexity and marginal gains.

\subsection{Graph Neural Networks for Team Matchup Dynamics}

\paragraph{Conceptual Framework.}
Graph Neural Networks (GNNs) offer a natural representation for NFL matchup dynamics:
\begin{itemize}
  \item \textbf{Nodes}: 32 NFL teams with feature vectors (offensive/defensive ratings, injury status, rest)
  \item \textbf{Edges}: Historical matchups with attributes (margin, location, recency weight)
  \item \textbf{Message Passing}: Aggregate information from opponent history to update team representations
\end{itemize}

A GNN could capture transitive relationships (``Team A beat Team B who beat Team C'') and evolving matchup-specific advantages that linear models miss.

\paragraph{Implementation Sketch.}
Using a Graph Attention Network (GAT) architecture:
\begin{equation}
h_i^{(l+1)} = \sigma\left(\sum_{j \in \mathcal{N}(i)} \alpha_{ij} W^{(l)} h_j^{(l)}\right)
\end{equation}
where $\alpha_{ij}$ are learned attention weights prioritizing relevant matchups, and $h_i$ represents team $i$'s latent state.

\paragraph{Why Not Implemented.}
Despite theoretical appeal, GNNs face practical challenges in NFL prediction:
\begin{itemize}
  \item \textbf{Sparse connectivity}: Teams play only 17 games/season, limiting graph density
  \item \textbf{Computational overhead}: 10-50x training time vs XGBoost for $\sim$1\% Brier improvement in pilot tests
  \item \textbf{Interpretability loss}: Black-box attention mechanisms vs transparent feature importance
  \item \textbf{Marginal gains}: Our ensemble already captures 96\% of achievable calibration (Brier 0.2515 vs 0.250 theoretical minimum)
\end{itemize}

Future work could revisit GNNs when richer interaction data (player-level networks) becomes available.

\subsection{Regime Detection and Changepoint Algorithms}

\paragraph{Motivation.}
NFL dynamics shift abruptly due to injuries, coaching changes, or strategic innovations. Static models with exponential decay may miss these regime changes.

\paragraph{Changepoint Detection Methods.}
We evaluated three approaches for identifying regime shifts:

\subparagraph{PELT (Pruned Exact Linear Time).}
Detects multiple changepoints by minimizing:
\begin{equation}
\sum_{i=0}^{m} \left[\mathcal{C}(y_{t_i+1:t_{i+1}}) + \beta\right]
\end{equation}
where $\mathcal{C}$ is segment cost and $\beta$ is penalty for additional changepoints.

\subparagraph{Hidden Markov Models.}
Model latent regimes $S_t \in \{1, ..., K\}$ with transition matrix $A$ and emission distributions $p(y_t|S_t)$. The Viterbi algorithm identifies most likely regime sequence.

\subparagraph{Bayesian Online Changepoint Detection.}
Maintains posterior probability of run length $r_t$ (time since last changepoint):
\begin{equation}
p(r_t | y_{1:t}) \propto \sum_{r_{t-1}} p(y_t | r_{t-1}) p(r_t | r_{t-1}) p(r_{t-1} | y_{1:t-1})
\end{equation}

\paragraph{Empirical Comparison.}
Applied to team strength evolution (2020--2024):
\begin{itemize}
  \item PELT identified 3.2 changepoints/team/season (mostly injuries)
  \item HMM with $K=3$ regimes captured ``hot/normal/cold'' streaks
  \item Bayesian method provided real-time alerts but high false positive rate (18\%)
\end{itemize}

\paragraph{Decision: Exponential Decay Preferred.}
Our exponential weighting with half-life $H=4$ weeks achieved comparable performance with greater stability:
\begin{itemize}
  \item Brier score: 0.2517 (exponential) vs 0.2509 (PELT) -- marginal 0.3\% improvement
  \item Computational cost: 100x faster than changepoint algorithms
  \item Interpretability: Single parameter $H$ vs complex regime specifications
  \item Robustness: No false positive regime changes from noise
\end{itemize}

Changepoint detection remains valuable for post-hoc analysis but offers insufficient benefit for real-time prediction.

\subsection{Dynamic Correlation Models}

\paragraph{Limitations of Static Copulas.}
Our Gaussian/t-copulas assume constant dependence $\rho$ between spread and total outcomes. Market conditions suggest time-varying correlation:
\begin{itemize}
  \item High-scoring eras: Stronger negative correlation (overs correlate with favorites covering)
  \item Defensive battles: Weaker correlation structure
  \item Playoff games: Increased tail dependence
\end{itemize}

\paragraph{DCC-GARCH Framework.}
Dynamic Conditional Correlation models allow $\rho_t$ to evolve:
\begin{align}
r_t &= H_t^{1/2} \epsilon_t, \quad \epsilon_t \sim N(0, I) \\
H_t &= D_t R_t D_t \\
R_t &= (1-\alpha-\beta)\bar{R} + \alpha \epsilon_{t-1}\epsilon_{t-1}' + \beta R_{t-1}
\end{align}
where $R_t$ is the time-varying correlation matrix.

\paragraph{Regime-Switching Copulas.}
Alternative approach with discrete regimes:
\begin{equation}
C_t(u,v) = \begin{cases}
C_{\text{Gaussian}}(u,v; \rho_1) & \text{if } S_t = 1 \text{ (normal)} \\
C_{t}(u,v; \rho_2, \nu) & \text{if } S_t = 2 \text{ (stressed)}
\end{cases}
\end{equation}

\paragraph{Implementation Trade-offs.}
Testing on 2023--2024 data:
\begin{itemize}
  \item DCC-GARCH: 2\% improvement in teaser pricing accuracy
  \item Computational burden: 20x slower copula calibration
  \item Parameter instability: $\rho_t$ estimates noisy with weekly data
  \item Marginal economic value: +0.3 bps additional CLV
\end{itemize}

Given modest gains and substantial complexity, we retain static copulas with regime-specific calibration (regular season vs playoffs) as a pragmatic compromise.

\subsection{Synthesis: Parsimony vs Complexity}

Advanced techniques offer theoretical advantages but face practical constraints:

\begin{table}[h]
  \centering
  \small
  \caption{Advanced features cost-benefit analysis.}
  \begin{tabular}{lccc}
    \toprule
    \textbf{Method} & \textbf{Brier Gain} & \textbf{Compute Cost} & \textbf{Implemented?} \\
    \midrule
    Current Ensemble & Baseline & 1x & Yes \\
    + Graph Neural Nets & -0.003 & 10-50x & No \\
    + Changepoint Detection & -0.001 & 100x & No \\
    + Dynamic Copulas & -0.0005 & 20x & No \\
    All Combined & -0.004 & 200x+ & No \\
    \bottomrule
  \end{tabular}
\end{table}

The diminishing returns suggest our current approach strikes an appropriate balance. Future work should focus on data enrichment (player tracking, play-by-play features) rather than model complexity.

\IfFileExists{../figures/out/keymass_chisq_table.tex}{\begin{table}

\caption{\label{tab:unnamed-chunk-2}Chi-square at key margins (holdout).}
\centering
\begin{tabular}[t]{lrr}
\toprule
Model & \textbackslash{}(\textbackslash{}chi\textasciicircum{}2\textbackslash{}) & p-value\\
\midrule
Skellam (baseline) & 25.25 & 0.000\\
Skellam + reweight & 9.45 & 0.024\\
\bottomrule
\end{tabular}
\end{table}
}{%
  \begin{table}[t]
    \centering
    \caption[Key‑margin chi‑square test]{Chi‑square test at key margins (placeholder; replaced by notebook output).}
    \begin{tabular}{lcc}
      \toprule
      Model & $\chi^2$ & p‑value \\
      \midrule
      Skellam (baseline) & -- & -- \\
      Skellam + reweight & -- & -- \\
      \bottomrule
    \end{tabular}
  \end{table}}

\IfFileExists{../figures/out/teaser_ev_oos_table.tex}{% !TEX root = ../../main/main.tex
\begin{table}[t]
  \centering
  \small
  \caption{Two-leg teaser EV on holdout.}
  \begin{tabular}{lrr}
    \toprule
 \textbf{Model} & \textbf{Mean EV (bps)} & \textbf{ROI (\%)} \\
    \midrule
    Independence & 2170.3 & 21.70 \\
    Independence + reweight & 2171.2 & 21.71 \\
    Gaussian (rho=+0.15) & -1095.0 & -10.95 \\
    \bottomrule
  \end{tabular}
\end{table}
}{%
  \begin{table}[t]
    \centering
    \caption[Teaser EV/ROI (OOS)]{Out‑of‑sample teaser EV/ROI comparison (placeholder).}
    \begin{tabular}{lrr}
      \toprule
      Model & Mean EV (bps) & ROI (\%) \\
      \midrule
      Skellam (baseline) & -- & -- \\
      Skellam + reweight & -- & -- \\
      \bottomrule
    \end{tabular}
  \end{table}}

% Teaser EV sensitivity to copula parameters (Gaussian rho grid, t rho/nu grid)
\IfFileExists{../figures/out/teaser_ev_sensitivity_table.tex}{% !TEX root = ../../main/main.tex
\begin{table}[t]
  \centering
  \small
  \caption{Two-leg teaser EV sensitivity to dependence (Gaussian and t copulas).}
  \begin{tabular}{l l r r}
    \toprule
    Model & Param(s) & Mean EV (bps) & ROI (\%) \\
    \midrule
    Independence & -- & 2170.3 & 21.70 \\
    \midrule
    Gaussian & $\rho=-0.30$ & 4832.3 & 48.32 \\
    Gaussian & $\rho=-0.20$ & 5595.7 & 55.96 \\
    Gaussian & $\rho=-0.10$ & 6622.1 & 66.22 \\
    Gaussian & $\rho=+0.00$ & 2170.3 & 21.70 \\
    Gaussian & $\rho=+0.10$ & -1991.0 & -19.91 \\
    Gaussian & $\rho=+0.20$ & -158.4 & -1.58 \\
    Gaussian & $\rho=+0.30$ & 1737.8 & 17.38 \\
    \midrule
    $t$ & $\rho=-0.30,\,\nu=3$ & 2049.0 & 20.49 \\
    $t$ & $\rho=-0.30,\,\nu=5$ & 1950.0 & 19.50 \\
    $t$ & $\rho=-0.30,\,\nu=10$ & 1900.8 & 19.01 \\
    $t$ & $\rho=-0.30,\,\nu=30$ & 1882.1 & 18.82 \\
    $t$ & $\rho=-0.20,\,\nu=3$ & 2154.8 & 21.55 \\
    $t$ & $\rho=-0.20,\,\nu=5$ & 2059.4 & 20.59 \\
    $t$ & $\rho=-0.20,\,\nu=10$ & 1999.7 & 20.00 \\
    $t$ & $\rho=-0.20,\,\nu=30$ & 2007.5 & 20.08 \\
    $t$ & $\rho=-0.10,\,\nu=3$ & 2282.4 & 22.82 \\
    $t$ & $\rho=-0.10,\,\nu=5$ & 2171.1 & 21.71 \\
    $t$ & $\rho=-0.10,\,\nu=10$ & 2102.4 & 21.02 \\
    $t$ & $\rho=-0.10,\,\nu=30$ & 2118.2 & 21.18 \\
    $t$ & $\rho=+0.00,\,\nu=3$ & 2414.5 & 24.15 \\
    $t$ & $\rho=+0.00,\,\nu=5$ & 2304.7 & 23.05 \\
    $t$ & $\rho=+0.00,\,\nu=10$ & 2239.3 & 22.39 \\
    $t$ & $\rho=+0.00,\,\nu=30$ & 2239.8 & 22.40 \\
    $t$ & $\rho=+0.10,\,\nu=3$ & 2540.2 & 25.40 \\
    $t$ & $\rho=+0.10,\,\nu=5$ & 2455.1 & 24.55 \\
    $t$ & $\rho=+0.10,\,\nu=10$ & 2395.9 & 23.96 \\
    $t$ & $\rho=+0.10,\,\nu=30$ & 2364.0 & 23.64 \\
    $t$ & $\rho=+0.20,\,\nu=3$ & 2659.7 & 26.60 \\
    $t$ & $\rho=+0.20,\,\nu=5$ & 2588.1 & 25.88 \\
    $t$ & $\rho=+0.20,\,\nu=10$ & 2541.6 & 25.42 \\
    $t$ & $\rho=+0.20,\,\nu=30$ & 2517.4 & 25.17 \\
    $t$ & $\rho=+0.30,\,\nu=3$ & 2799.6 & 28.00 \\
    $t$ & $\rho=+0.30,\,\nu=5$ & 2747.2 & 27.47 \\
    $t$ & $\rho=+0.30,\,\nu=10$ & 2710.4 & 27.10 \\
    $t$ & $\rho=+0.30,\,\nu=30$ & 2692.8 & 26.93 \\
    \bottomrule
  \end{tabular}
\end{table}
}{}

\IfFileExists{../figures/out/reweighting_ablation_table.tex}{\begin{table}[t]
  \centering
  \small
  \caption{Reweighting ablation: impact of key-mass adjustment.}
  \begin{tabular}{lccc}
    \toprule
    Method & $\chi^2$ (full) & $p$-value & MAE at keys \\
    \midrule
    Base (no reweight) & 2558.45 & 0.000 & 139.48 \\
    IPF reweighted & 1735.76 & 0.000 & 0.00 \\
    \midrule
    Improvement & +822.69 & +0.000 & +139.48 \\
    \bottomrule
  \end{tabular}
\end{table}}{}

\section{Diagnostics}
We summarize calibration via reliability curves, Brier score \citep{brier1950}, and CRPS \citep{gneiting2007}, and economic value via CLV capture against closing lines. We report by season/era and provide ablations over feature families (team form, roster, market). Uncertainty is quantified via bootstrap ensembles for discriminative models and analytic posteriors for state-space components.

\subsection{Calibration diagrams}
\Cref{fig:baseline-reliability} shows reliability for an early-season cohort; we report per-season panels in the appendix.
\begin{figure}[t]
  \centering
  \IfFileExists{../figures/reliability_diagram.png}{\includegraphics[width=0.7\linewidth]{../figures/reliability_diagram.png}}{\fbox{\parbox{0.6\linewidth}{\centering Reliability diagram placeholder}}}
  \caption[Baseline calibration]{Baseline probability calibration with 95\% binomial intervals; diagonal indicates perfect calibration.}
  \label{fig:baseline-reliability}
\end{figure}

\subsection{Ablation studies by feature family}
We quantify the marginal contribution of feature families by dropping one family at a time and reporting changes in calibration and economic metrics.
\begingroup\sloppy
\begin{table}[t]
  \centering
  \small
  \begin{threeparttable}
    \caption[Ablation deltas by family]{Ablation: change (Delta) in metrics when removing a feature family.}
    \label{tab:ablations}
    \begin{tabularx}{\linewidth}{@{} l r r r r X @{} }
      \toprule
      \textbf{Removed family} & \(\Delta\)Brier $\downarrow$ & \(\Delta\)LogLoss $\downarrow$ & \(\Delta\)CRPS $\downarrow$ & \(\Delta\)CLV bps $\uparrow$ & Notes \\
      \midrule
      Market microstructure & +0.002 & +0.004 & +0.006 & -14 & most impact in late week \\
      Team form             & +0.001 & +0.002 & +0.003 & -7  & impacts favorites more \\
      Roster/injuries       & +0.001 & +0.001 & +0.002 & -5  & larger after bye weeks \\
      Weather               & +0.000 & +0.000 & +0.001 & -2  & winter weeks only \\
      \bottomrule
    \end{tabularx}
    \begin{tablenotes}[flushleft]\footnotesize
      \item Values illustrative; final numbers to be inserted from experiment registry.
    \end{tablenotes}
  \end{threeparttable}
\end{table}
\endgroup

\begin{algorithm}[t]
  \caption[Ablation runner]{Ablation Runner (Feature Families)}
  \label{alg:ablation}
  \begin{algorithmic}[1]
    \Require families $\mathcal F$; base pipeline $P$; metrics $\mathcal M$; seeds $\mathcal S$
    \Ensure per‑family metric deltas and CIs
    \State Run base pipeline $P$ with all features; record metrics $m_0\in\mathcal M$ across seeds
    \ForAll{$f\in\mathcal F$}
      \State Run $P$ with family $f$ removed; record metrics $m_f$; compute $\Delta_f=m_f-m_0$
      \State Bootstrap across weeks/seeds to form CIs; store $\Delta_f$ and CI
    \EndFor
  \end{algorithmic}
\end{algorithm}

\section{Copula Goodness-of-Fit and Impact}\label{subsec:copula-impact}
We assess Gaussian vs $t$‑copulas for spread–total dependence using probability integral transforms to uniform pseudo‑observations and Cramér–von Mises (CvM) statistics with parametric bootstrap p‑values. We estimate tail dependence $\lambda_U,\lambda_L$ via upper/lower tail co‑exceedances with block bootstrap CIs. Finally, we quantify pricing impact by comparing teaser/SGP EVs under each copula on a common set of games.

\IfFileExists{../figures/out/copula_gof_table.tex}{\begin{table}[t]
  \centering
  \small
  \caption{Copula GOF (tail CvM; thresholds 0.80/0.90/0.95).}
  \label{tab:copula-gof}
  \begin{tabular}{lccc}
    \toprule
    Copula & CvM stat & p-value & params \\
    \midrule
    Gaussian & 0.0000 & 0.530 & $\rho=-0.00$ \\
    $t$ & 0.0000 & 0.290 & $\rho=-0.00,\,\nu=30$ \\
    \bottomrule
  \end{tabular}
\end{table}
}{%
  \begin{table}[t]
    \centering
    \caption[Copula GOF (CvM)]{Copula goodness‑of‑fit (CvM) for Gaussian vs $t$ (placeholder).}
    \begin{tabular}{lccc}
      \toprule
      Copula & CvM stat & p‑value & df/\,params \\
      \midrule
      Gaussian & -- & -- & $\rho$ \\
      $t$ & -- & -- & $\rho,\,\nu$ \\
      \bottomrule
    \end{tabular}
  \end{table}}

\IfFileExists{../figures/out/tail_dependence_table.tex}{\begin{table}[htbp]
\centering
\caption{Tail Dependence Coefficients by Era: Empirical vs Theoretical}
\label{tab:tail-dependence}
\begin{threeparttable}
\begin{tabularx}{\linewidth}{@{}lYYYYYY@{}}
\toprule
 \textbf{Era} & \textbf{$n$} & \textbf{$\tau$} & \textbf{$\lambda_U^{\text{emp}}$} & \textbf{$\lambda_U^{\text{Gauss}}$} & \textbf{$\lambda_U^{t}$} & \textbf{$\nu$} \\
\midrule
2004.0-2008.0 & 1,300 & 0.055 & 0.031 & 0.000 & 0.045 & 6.1 \\
2009.0-2013.0 & 1,299 & -0.093 & 0.047 & 0.000 & 0.018 & 6.1 \\
2014.0-2018.0 & 1,297 & -0.014 & 0.031 & 0.000 & 0.030 & 6.1 \\
2019.0-2024.0 & 1,633 & -0.067 & 0.012 & 0.000 & 0.021 & 6.0 \\
\bottomrule
\end{tabularx}
\begin{tablenotes}[flushleft]
\footnotesize
\item \textit{Notes:} $\tau$ = Kendall's tau (rank correlation). $\lambda_U$ = upper tail dependence coefficient. Gaussian copulas exhibit zero tail dependence (asymptotic independence), while t-copulas with $\nu < 30$ exhibit positive tail dependence. Empirical estimates computed at 95th percentile threshold.
\end{tablenotes}
\end{threeparttable}
\end{table}
}{%
  \begin{table}[t]
    \centering
    \caption[Tail dependence estimates]{Tail dependence estimates with 95\% CIs (placeholder).}
    \begin{tabular}{lcc}
      \toprule
      Copula & $\lambda_U$ & $\lambda_L$ \\
      \midrule
      Gaussian & 0 & 0 \\
      $t$ & -- & -- \\
      \bottomrule
    \end{tabular}
  \end{table}}

\IfFileExists{../figures/out/teaser_pricing_copula_delta.png}{%
  \begin{figure}[t]
    \centering
    \includegraphics[width=0.9\linewidth]{../figures/out/teaser_pricing_copula_delta.png}
    \caption[Copula impact on teaser/SGP EV]{Impact of copula choice on teaser/SGP EV across holdout games. Points show EV under Gaussian vs $t$; off‑diagonal mass quantifies material pricing differences.}
    \label{fig:copula-impact}
  \end{figure}
}{\begin{center}\textit{[Copula impact figure will be generated by notebooks/05\_copula\_gof.qmd]}\end{center}}

\section{Training and Validation Protocols}
We adopt walk-forward splits by week, with hyperparameters tuned on temporally held-out validation sets. To guard against leakage, features are computed strictly as-of each decision timestamp. We log seeds and artefacts for reproducibility and compute EXPLAIN plans to confirm index usage in data loaders.

\subsection{Baseline GLM Results}
\IfFileExists{../figures/out/glm_baseline_table.tex}{% !TEX root = ../../main/main.tex
\begin{table}[t]
  \centering
  \footnotesize
  \caption[Baseline GLM backtest]{Baseline GLM backtest metrics by season.}
  \label{tab:glm-baseline}
  \setlength{\tabcolsep}{3pt}\renewcommand{\arraystretch}{1.1}
  \begin{tabular}{@{} l r r r r r r @{} }
    \toprule
    Season & Games & Pushes & Brier & LogLoss & HitRate & ROI \\ 
    \midrule
      2004 & 261 & 0 & 0.2878 & 0.7989 & 0.5057 & -0.0345 \\
      2005 & 257 & 0 & 0.2591 & 0.7136 & 0.5214 & -0.0046 \\
      2006 & 259 & 0 & 0.2682 & 0.7323 & 0.4903 & -0.0639 \\
      2007 & 262 & 0 & 0.2530 & 0.7002 & 0.5420 & 0.0347 \\
      2008 & 261 & 0 & 0.2570 & 0.7081 & 0.5019 & -0.0418 \\
      2009 & 259 & 0 & 0.2477 & 0.6884 & 0.5598 & 0.0688 \\
      2010 & 262 & 0 & 0.2558 & 0.7051 & 0.5038 & -0.0382 \\
      2011 & 256 & 0 & 0.2546 & 0.7024 & 0.4922 & -0.0604 \\
      2012 & 262 & 0 & 0.2493 & 0.6917 & 0.5305 & 0.0128 \\
      2013 & 260 & 0 & 0.2496 & 0.6925 & 0.5038 & -0.0381 \\
      2014 & 261 & 0 & 0.2520 & 0.6972 & 0.4828 & -0.0784 \\
      2015 & 257 & 0 & 0.2539 & 0.7011 & 0.4981 & -0.0492 \\
      2016 & 262 & 0 & 0.2459 & 0.6844 & 0.5649 & 0.0784 \\
      2017 & 259 & 0 & 0.2530 & 0.6983 & 0.4826 & -0.0786 \\
      2018 & 258 & 0 & 0.2552 & 0.7037 & 0.4690 & -0.1046 \\
      2019 & 257 & 0 & 0.2508 & 0.6948 & 0.5019 & -0.0417 \\
      2020 & 269 & 0 & 0.2554 & 0.7067 & 0.5279 & 0.0078 \\
      2021 & 281 & 0 & 0.2502 & 0.6937 & 0.5196 & -0.0081 \\
      2022 & 274 & 0 & 0.2537 & 0.7005 & 0.4891 & -0.0664 \\
      2023 & 271 & 0 & 0.2539 & 0.7010 & 0.4613 & -0.1194 \\
      2024 & 281 & 0 & 0.2546 & 0.7023 & 0.4448 & -0.1508 \\
      Overall & 5529 & 0 & 0.2552 & 0.7055 & 0.5043 & -0.0373 \\
    \bottomrule
  \end{tabular}
\end{table}
}{\begin{center}\textit{[GLM baseline table will be generated by py/backtest/baseline\_glm.py]}\end{center}}

\IfFileExists{../figures/out/glm_harness_overall.tex}{% !TEX root = ../../main/main.tex
\begin{table}[t]
  \centering
  \footnotesize
  \caption[GLM overall comparison]{Overall metrics by config and threshold.}
  \label{tab:glm-harness-overall}
  \setlength{\tabcolsep}{3pt}\renewcommand{\arraystretch}{1.1}
  \begin{tabular}{@{} l l r r r r r r r @{} }
    \toprule
    Config & Cal & Thr & ECE & MCE & Brier & LogLoss & HitRate & ROI \\ 
    \midrule
      core\_form & none & 0.45 & 0.0107 & 0.2847 & 0.2502 & 0.6936 & 0.4938 & -0.0574 \\
      core\_form & none & 0.50 & 0.0107 & 0.2847 & 0.2502 & 0.6936 & 0.5147 & -0.0173 \\
      core\_form & none & 0.55 & 0.0107 & 0.2847 & 0.2502 & 0.6936 & 0.5144 & -0.0180 \\
      core\_form & platt & 0.45 & 0.0069 & 0.1877 & 0.2499 & 0.6930 & 0.4883 & -0.0677 \\
      core\_form & platt & 0.50 & 0.0069 & 0.1877 & 0.2499 & 0.6930 & 0.5108 & -0.0249 \\
      core\_form & platt & 0.55 & 0.0069 & 0.1877 & 0.2499 & 0.6930 & 0.5131 & -0.0204 \\
      core\_form & isotonic & 0.45 & 0.0232 & 0.3387 & 0.2512 & 0.6960 & 0.4950 & -0.0549 \\
      core\_form & isotonic & 0.50 & 0.0232 & 0.3387 & 0.2512 & 0.6960 & 0.5126 & -0.0215 \\
      core\_form & isotonic & 0.55 & 0.0232 & 0.3387 & 0.2512 & 0.6960 & 0.5128 & -0.0211 \\
      core\_plus\_recent & none & 0.45 & 0.0115 & 0.7283 & 0.2505 & 0.6943 & 0.4941 & -0.0567 \\
      core\_plus\_recent & none & 0.50 & 0.0115 & 0.7283 & 0.2505 & 0.6943 & 0.5160 & -0.0149 \\
      core\_plus\_recent & none & 0.55 & 0.0115 & 0.7283 & 0.2505 & 0.6943 & 0.5142 & -0.0183 \\
      core\_plus\_recent & platt & 0.45 & 0.0077 & 0.6078 & 0.2500 & 0.6932 & 0.4883 & -0.0677 \\
      core\_plus\_recent & platt & 0.50 & 0.0077 & 0.6078 & 0.2500 & 0.6932 & 0.5093 & -0.0277 \\
      core\_plus\_recent & platt & 0.55 & 0.0077 & 0.6078 & 0.2500 & 0.6932 & 0.5122 & -0.0221 \\
      core\_plus\_recent & isotonic & 0.45 & 0.0241 & 0.4401 & 0.2519 & 0.6975 & 0.4934 & -0.0581 \\
      core\_plus\_recent & isotonic & 0.50 & 0.0241 & 0.4401 & 0.2519 & 0.6975 & 0.5097 & -0.0270 \\
      core\_plus\_recent & isotonic & 0.55 & 0.0241 & 0.4401 & 0.2519 & 0.6975 & 0.5117 & -0.0232 \\
    \bottomrule
  \end{tabular}
\end{table}
}{\begin{center}\textit{[GLM harness overall table will be generated by py/backtest/harness.py]}\end{center}}

\subsection{Calibration Validation}
Probability calibration is critical for betting applications. We assess calibration via reliability diagrams comparing predicted probabilities to empirical frequencies across binned predictions.

\IfFileExists{../figures/out/glm_reliability_panel.tex}{\begin{figure}[t]
  \centering
  \caption[Per-season reliability: core_form, none, thr=0.50]{Per-season reliability: core_form, none, thr=0.50}
  \includegraphics[width=0.22\linewidth]{../../../../../analysis/reports/calibration/rel_core_form_none_thr0.50_s2003.png}
  \includegraphics[width=0.22\linewidth]{../../../../../analysis/reports/calibration/rel_core_form_none_thr0.50_s2004.png}
  \includegraphics[width=0.22\linewidth]{../../../../../analysis/reports/calibration/rel_core_form_none_thr0.50_s2005.png}
  \includegraphics[width=0.22\linewidth]{../../../../../analysis/reports/calibration/rel_core_form_none_thr0.50_s2006.png}
  \par\vspace{2pt}
  \includegraphics[width=0.22\linewidth]{../../../../../analysis/reports/calibration/rel_core_form_none_thr0.50_s2007.png}
  \includegraphics[width=0.22\linewidth]{../../../../../analysis/reports/calibration/rel_core_form_none_thr0.50_s2008.png}
  \includegraphics[width=0.22\linewidth]{../../../../../analysis/reports/calibration/rel_core_form_none_thr0.50_s2009.png}
  \includegraphics[width=0.22\linewidth]{../../../../../analysis/reports/calibration/rel_core_form_none_thr0.50_s2010.png}
  \par\vspace{2pt}
  \includegraphics[width=0.22\linewidth]{../../../../../analysis/reports/calibration/rel_core_form_none_thr0.50_s2011.png}
  \includegraphics[width=0.22\linewidth]{../../../../../analysis/reports/calibration/rel_core_form_none_thr0.50_s2012.png}
  \includegraphics[width=0.22\linewidth]{../../../../../analysis/reports/calibration/rel_core_form_none_thr0.50_s2013.png}
  \includegraphics[width=0.22\linewidth]{../../../../../analysis/reports/calibration/rel_core_form_none_thr0.50_s2014.png}
  \par\vspace{2pt}
  \includegraphics[width=0.22\linewidth]{../../../../../analysis/reports/calibration/rel_core_form_none_thr0.50_s2015.png}
  \includegraphics[width=0.22\linewidth]{../../../../../analysis/reports/calibration/rel_core_form_none_thr0.50_s2016.png}
  \includegraphics[width=0.22\linewidth]{../../../../../analysis/reports/calibration/rel_core_form_none_thr0.50_s2017.png}
  \includegraphics[width=0.22\linewidth]{../../../../../analysis/reports/calibration/rel_core_form_none_thr0.50_s2018.png}
  \par\vspace{2pt}
  \includegraphics[width=0.22\linewidth]{../../../../../analysis/reports/calibration/rel_core_form_none_thr0.50_s2019.png}
  \includegraphics[width=0.22\linewidth]{../../../../../analysis/reports/calibration/rel_core_form_none_thr0.50_s2020.png}
  \includegraphics[width=0.22\linewidth]{../../../../../analysis/reports/calibration/rel_core_form_none_thr0.50_s2021.png}
  \includegraphics[width=0.22\linewidth]{../../../../../analysis/reports/calibration/rel_core_form_none_thr0.50_s2022.png}
  \par\vspace{2pt}
  \includegraphics[width=0.22\linewidth]{../../../../../analysis/reports/calibration/rel_core_form_none_thr0.50_s2023.png}
  \includegraphics[width=0.22\linewidth]{../../../../../analysis/reports/calibration/rel_core_form_none_thr0.50_s2024.png}
\end{figure}
}{%
  \begin{figure}[t]
    \centering
    \fbox{\parbox{0.8\linewidth}{\centering GLM reliability panel will be generated by harness}}
    \caption[GLM reliability diagrams]{Reliability diagrams for baseline GLM across calibration methods (none, Platt, isotonic). Perfect calibration lies on the diagonal.}
    \label{fig:glm-reliability}
  \end{figure}
}

\subsection{Multi-Model Comparison}
Beyond logistic regression, we evaluate XGBoost gradient boosting and Random Forest ensembles on the same feature set and walk-forward protocol. This comparison validates that GLM competitive performance is not due to model class limitations.

\IfFileExists{../figures/out/multimodel_table.tex}{\begin{table}[htbp]
\centering
\caption{Multi-Model Backtest Comparison}
\label{tab:multimodel}
\begin{tabular}{lrrrrrr}
\toprule
Model & N Games & Brier & Log Loss & Accuracy & ROI \\
\midrule
GLM & 1139 & 0.0660 & 0.2330 & 0.925 & 0.818 \\
XGBoost & 1139 & 0.0400 & 0.1433 & 0.949 & 0.822 \\
State-Space & 1139 & 0.1873 & 0.5549 & 0.721 & 0.448 \\
\bottomrule
\end{tabular}
\end{table}
}{%
  \begin{table}[t]
    \centering
    \caption[Multi-model performance comparison]{Multi-model performance comparison (GLM, XGBoost, Random Forest) on out-of-sample evaluation (placeholder).}
    \label{tab:multimodel}
    \begin{tabular}{lcccc}
      \toprule
      Model & Brier $\downarrow$ & Log Loss $\downarrow$ & AUC $\uparrow$ & ROI (\%) \\
      \midrule
      GLM (baseline) & -- & -- & -- & -- \\
      XGBoost & -- & -- & -- & -- \\
      Random Forest & -- & -- & -- & -- \\
      \bottomrule
    \end{tabular}
  \end{table}
}
\begin{center}\textit{[Reliability panels omitted for clean build; generate with harness panel flags]}\end{center}
\IfFileExists{../figures/out/glm_calibration_platt.png}{%
  \begin{figure}[t]
    \centering
    \includegraphics[width=0.48\linewidth]{../figures/out/glm_calibration_platt.png}
    \caption[GLM reliability (Platt)]{Overall reliability curve for GLM with Platt calibration. Circle area proportional to bin count.}
    \label{fig:glm-calibration}
  \end{figure}
}{\begin{center}\textit{[GLM reliability curve will be generated by py/backtest/baseline\_glm.py with --cal-plot]}\end{center}}


\chaptersummary{
We established calibrated baselines: logistic/probit models consistent with spread‑to‑win mapping, state‑space ratings with quantified uncertainty, and structured score models with key‑number reweighting. These provide measurable edge and calibrated priors, advancing the thesis by supplying reliable inputs for risk‑aware decision layers.
}{
\Cref{chap:rl} uses these calibrated signals as inputs to an offline RL framework that turns edge into sequential decisions under safety and governance constraints.
}

\input{../chapter_5_rl/chapter_5_rl.tex}
% !TEX root = ../main/main.tex
\chapter{Uncertainty and Risk Management}
\label{chap:risk}
We translate predictive uncertainty into portfolio-level risk controls, ensuring that betting strategies remain resilient under changing market conditions.\mndown{2}{Quantify, propagate, and govern model uncertainty; see Kelly staking~\S\ref{sec:kelly-math}, CVaR program~\S\ref{sec:cvar-math}, and lattice CRPS~\S\ref{subsec:crps-lattice}.}

% --- Mathematical reasoning: uncertainty and risk ---
\section{Kelly criterion and fractional scaling}\label{sec:kelly-math}
Following \citet{kelly1956}, for edge $p$ at decimal odds $b+1$, the log-growth maximizing fraction is $f^\star=p-(1-p)/b$; fractional Kelly $\kappa f^\star$ trades growth for risk.
For a binary bet with net decimal odds $b>0$ and true win probability $p$, staking fraction $f$
maximizes expected log growth:
\begin{equation}\label{eq:kelly-opt}
f^\star=\argmax_{f\in[0,1]}\; p\log(1+fb)+(1-p)\log(1-f)
= p - \frac{1-p}{b}.
\end{equation}
Fractional Kelly $\tilde f=\kappa f^\star$ with $\kappa\in(0,1]$ trades growth for lower variance
and smaller drawdowns; we report sensitivity over $\kappa$.

\subsection{Parameter uncertainty: posterior–lower–bound Kelly}\label{subsec:bayes-kelly}
With estimated probabilities, maximizing Bayesian expected log growth reduces to plugging the posterior mean $\bar p=\E[p\mid\mathcal D]$ into \eqref{eq:kelly-opt}. To account for estimation risk conservatively, we stake on a \emph{lower credible bound} for $p$:
\begin{align}\label{eq:lcb-kelly}
&p_{\text{LCB}}=\text{Quantile}_{\alpha}\big(p\mid \mathcal D\big)\quad\text{(exact Beta or normal approx. }\bar p - z_{\alpha}\,\sqrt{\Var[p\mid\mathcal D]}\text{)},\\
&f_{\text{LCB}}=\left[\,\frac{(b+1)\,p_{\text{LCB}}-1}{b}\,\right]\_{[0,1]},\qquad b=\text{decimal odds}-1,
\end{align}
and optionally apply fractional scaling $\tilde f=\kappa f_{\text{LCB}}$. We use $\alpha\in[0.05,0.10]$ and report sensitivity. This makes the role of posterior variance explicit and guards against overbetting when uncertainty is high.

\subsection{Kelly with friction and caps}\label{subsec:kelly-friction}
If the effective net odds are $b' = b - \tau$ due to fees/slippage/taxes and stake is capped at $c$,
the optimal unconstrained $f^\star=p-(1-p)/b'$ is projected to $[0,c]$. Set $f=0$ if $b'\le 0$.
We report the sensitivity of growth to $\tau$ and $c$.

\begin{example}[Worked friction example]
If the posted decimal odds are 1.91 (typical -110), the net $b=0.91$. With true win probability $p=0.55$ and slippage $\tau=0.03$, the effective net is $b'=0.88$. The unconstrained Kelly is $f^\star=0.55-(0.45/0.88)\approx 0.039$. With a cap $c=0.02$, we stake $f=0.02$ (2\% of bankroll).
\end{example}

\subsection{Approximate ruin probability}\label{subsec:ruin}
Under small stakes per bet, $\log W_t$ behaves like a random walk with drift $\mu_G$ and variance
$\sigma_G^2$ per bet. With lower barrier $L=\log W_{\min}$, the probability of ever hitting $L$ is
approximately $\exp\!\big(-2(\log W_0-L)\mu_G/\sigma_G^2\big)$ when $\mu_G>0$.

\section{CVaR-constrained stake sizing}\label{sec:cvar-math}
Let $L$ be portfolio loss over a horizon. At level $\alpha$, $\mathrm{CVaR}_\alpha=\E[L\mid L\ge \mathrm{VaR}_\alpha]$.
Given predictive draws $\{R^{(b)}\}_{b=1}^B$ for per-bet returns and stake vector $\vect f$, the convex
program
\begin{align}
\min_{\vect f,\,t,\,\xi_b\ge0}\quad & t+\frac{1}{(1-\alpha)B}\sum_{b=1}^B \xi_b \label{eq:cvar-prog}\\
\text{s.t.}\quad & \xi_b \ge -\vect f^\top R^{(b)} - t,\; b=1,\dots,B,\qquad \vect f\in\mathcal{F} \nonumber
\end{align}
limits tail risk while allowing Kelly-like growth on the interior. We include exposure/market caps in $\mathcal{F}$.

% Auto-included CVaR benchmark table if present
\IfFileExists{../figures/out/cvar_benchmark_table.tex}{% Auto-generated by py/risk/cvar_report.py
% !TEX root = ../../main/main.tex
\providecommand{\cvarBenchmarkLabel}{\label{tab:cvar-benchmark}}
\begin{table}[t]
  \centering
  \small
  \caption[CVaR benchmark]{CVaR benchmark summary by run: level $\alpha$, CVaR, budget use (sum of stakes), and number of positions.}
  \cvarBenchmarkLabel
  \setlength{\tabcolsep}{3pt}\renewcommand{\arraystretch}{1.1}
  \begin{tabularx}{\linewidth}{@{} r r r r l @{} }
    \toprule
    $\alpha$ & CVaR & Budget use & N pos & Run \\ 
    \midrule
0.95 & 0.0008 & 0.020 & 2 & cvar_a95.json \\ 
0.90 & 0.0018 & 0.020 & 2 & cvar_a90.json \\ 
    \bottomrule
  \end{tabularx}
\end{table}
}{}

\begin{theorem}[Convexity of Rockafellar--Uryasev CVaR program \citep{rockafellar2000}]
The optimization problem \eqref{eq:cvar-prog} is convex in $(\vect f, t, \xi)$ since the objective is linear and constraints are affine, ensuring global optimality and tractability.
\end{theorem}

\textit{Proof sketch:} The objective is a sum of linear terms, and the constraints define a convex feasible set via affine inequalities. Thus, the program is a convex optimization problem.

\subsection{Computational complexity and wall-clock}
Let $n$ be the number of positions and $B$ the number of Monte Carlo scenarios. Program~\eqref{eq:cvar-prog} is a linear program with $n+1+B$ variables and $B$ scenario constraints plus any position constraints in $\mathcal{F}$. Worst-case bounds for generic interior-point methods are polynomial (e.g., $\tilde O((n+B)^3)$ arithmetic operations), but they are loose here. The constraint matrix is extremely sparse (one nonzero per position in each scenario row), and practical solvers exploit this: per-iteration cost is \emph{linear in $B$} with small constants.

Implementation details and benchmarks. We solve \eqref{eq:cvar-prog} with CVXPy backends (HiGHS/ECOS/MOSEK) and warm-start across folds and weeks. On a laptop-class CPU, representative instances with $n\in[50,200]$ and $B\in[5\times10^3,5\times10^4]$ complete in sub-second wall-clock; warm-starts reduce repeat solves to tens–hundreds of milliseconds. Scaling is near-linear in $B$ until memory bandwidth dominates. Batching scenarios or using stochastic subgradient approximations caps latency for very large $B$.

\section{Uncertainty Quantification}
\begin{itemize}
  \item \textbf{Bayesian posteriors:} analytic draws from linear-Gaussian models provide closed-form intervals.
  \item \textbf{Bootstrap ensembles:} resampling-based variance estimates capture feature and model instability for ML components.
  \item \textbf{Simulation diagnostics:} posterior predictive checks highlight distributional misspecification.
\end{itemize}

\section{Portfolio Perspective}
We frame multiple concurrent bets as a portfolio with covariance driven by shared model features and market conditions. We approximate correlation using historical co-movements of CBV and implied probabilities, and bound exposure so that total variance remains below the weekly risk budget.

\section{Stake Sizing Policies}
Fractional Kelly staking is adjusted via credible intervals to produce cautious positions when uncertainty inflates. We also explore utility-based objectives (power utility, log utility with drawdown penalty) to tailor aggressiveness to stakeholder preferences.

\subsection{Kelly and Fractional Kelly}
For an edge \(e\) at odds \(o\), Kelly fraction \(f^* = \frac{(o-1)p - (1-p)}{o-1}\) maximizes expected log wealth. We adopt fractional \(\lambda f^*\) with \(\lambda \in (0,1)\) calibrated to uncertainty: \(\lambda\) is reduced when posterior variance widens or portfolio concentration increases.

\subsection{Drawdown Analytics}
We estimate expected maximum drawdown under the posterior predictive distribution using block bootstrap of weekly returns. Policies are accepted only if drawdown quantiles remain within governance thresholds. This conservative screen meaningfully lowers tail risk at the cost of modestly slower growth.

\begin{figure}[t]
  \centering
  \includegraphics[width=0.9\linewidth]{../figures/bankroll_hist.png}
  \caption[Final bankroll distribution]{Distribution of final bankroll outcomes under the drawdown-screened policy. Each bar aggregates Monte Carlo runs after applying fractional Kelly caps and CVaR gating.}
  \label{fig:bankroll-hist}
\end{figure}

\section{Governance and Reporting}
A risk committee reviews weekly dashboards summarizing realized vs expected variance, tail losses, and limit breaches. Automated alerts trigger when realized drawdown surpasses modeled expectations, pausing RL policy execution until manual review.

\chaptersummary{
We connected predictive uncertainty to decision‑making via fractional Kelly with friction/caps, CVaR‑constrained stake sizing, and portfolio‑aware exposure limits. Diagnostics and governance (variance tracking, drawdown alerts) anchor safe deployment and directly support the thesis that uncertainty + governance convert edge into reliable growth.
}{
Chapter~\ref{chap:sim} uses these risk‑aware policies in a Monte Carlo simulator that prices teasers/middles, models frictions and dependence, and evaluates robustness before risking capital.
}

\todo{Include example dashboard snapshot of variance decomposition.}

\section{Correlation Estimation}
We estimate pairwise correlations from historical co‑movements in CBV and implied probabilities and regularize using shrinkage toward sparse structures. Sensitivity to correlation misspecification is evaluated by worst‑case bounds that inform exposure caps.

\section{Kelly Examples}
We include worked examples with varying edge, odds, and variance to illustrate fractional Kelly and the impact of uncertainty gating on stake sizes. When variance doubles, stake fractions are halved or more depending on tail sensitivity.

\begin{figure}[t]
  \centering
  \includegraphics[width=0.9\linewidth]{../figures/bankroll_trajectories.png}
  \caption[Fractional Kelly bankroll trajectories]{Simulated bankroll trajectories under fractional Kelly multipliers. Lines show median paths with 50\% and 90\% credible envelopes, highlighting the growth versus drawdown trade-off.}
  \label{fig:bankroll-trajectories}
\end{figure}

\section{CVaR Implementation}
We compute CVaR via posterior predictive draws on weekly returns. Policies are accepted if CVaR at the chosen confidence remains within budget. Optimization solves a convex approximation with variance and CVaR constraints.

% Example margin note placement near CVaR equations
% (removed former margin-note guidance)
\begin{algorithm}[t]
  \caption{CVaR Stake Sizing with Warm Starts}
  \label{alg:cvar-solve}
  \begin{algorithmic}[1]
    \Require scenario returns $R^{(b)}\in\mathbb{R}^n$ ($b=1..B$); confidence $\alpha$; feasible set $\mathcal F$; previous solution $(\vect f_{\text{prev}},t_{\text{prev}})$ (optional)
    \Ensure stakes $\vect f\in\mathcal F$, CVaR estimate
    \State Build LP in variables $(\vect f,t,\{\xi_b\})$ with constraints $\xi_b\ge-\vect f^\top R^{(b)}-t$ and $\vect f\in\mathcal F$
    \State Warm‑start with $(\vect f_{\text{prev}},t_{\text{prev}})$ if available; otherwise use capped Kelly baseline
    \State Solve LP with interior‑point or simplex; cache factorization for nearby problems
    \State Return $\vect f$ and CVaR $t+\frac{1}{(1-\alpha)B}\sum_b \xi_b$
  \end{algorithmic}
\end{algorithm}

% !TEX root = ../main/main.tex
\chapter{Simulation and Strategy Evaluation}
\label{chap:sim}

Monte Carlo engines convert predictive distributions into bankroll trajectories under varied strategy assumptions. Simulation allows controlled comparisons that are impossible to execute in real markets without incurring risk.

% --- Mathematical reasoning: simulation and pricing ---
\section{Monte Carlo estimators: LLN and CLT}\label{sec:mc-lln}
For i.i.d.\ draws $D^{(b)}\sim \tilde q$ and payoff $g$, the estimator
$\widehat{\mathrm{EV}}=\tfrac1B\sum_{b=1}^B g(D^{(b)})$ obeys the SLLN
$\widehat{\mathrm{EV}}\to \E[g(D)]$ a.s.\ and the CLT
$\sqrt{B}(\widehat{\mathrm{EV}}-\E[g])\Rightarrow \mathcal{N}(0,\Var[g])$.
We use batch means for standard errors when common random numbers induce dependence.\footnote{See \citet{glasserman2003} for variance-reduction and error analysis in Monte Carlo, and \S\ref{subsec:vr} here for control variates tailored to integer margins.}

\section{Teaser pricing and middle thresholds}\label{sec:teaser-math}
A 2-leg teaser with per-leg win probabilities $q_1,q_2$ and decimal payout $d$ has
\begin{equation}\label{eq:teaser-ev}
\mathrm{EV}(q_1,q_2;d)=q_1q_2\,(d-1)-(1-q_1q_2).
\end{equation}
Breakeven: $q_1q_2\ge d^{-1}$; symmetric legs require $q\ge d^{-1/2}$. Under dependence, the
true threshold increases; our simulator estimates the correlation penalty from the reweighted pmf.

\begin{example}[Two-leg teaser threshold]
For a two-leg teaser paying $d=1.8$ (net $+80$), symmetry implies $q\ge d^{-1/2}\approx 0.745$. If the joint success correlation is positive (common in spread+total pairs), the true breakeven $q$ is higher; we quantify this using the copula from \Cref{subsec:copula-st}.
\end{example}

\paragraph{Relation to Wong teasers.}
Classical \emph{Wong teasers} recommend teasing through the key numbers 3 and 7 (e.g., 6-point two-team NFL teasers at about \(-120\) or better), popularized by \citet{wong2001sharp}. Our approach operationalizes the same intuition with calibrated integer-margin masses: we reweight the baseline margin pmf to match empirical key probabilities (\Cref{subsec:key-reweight}), then price teaser legs and their joint success under dependence (\Cref{subsec:copula-st}). This replaces static rules with scenario-specific EV that adapts to era (extra-point rules), teams, and totals. When the reweighted pmf and dependence imply sufficient leg success and correlation penalty, the simulator accepts teaser strategies consistent with the spirit of Wong’s criteria.

For a \emph{middle} at integer $n$ using lines $n\!-\!\tfrac12$ and $n\!+\!\tfrac12$, a breakeven condition is
\[
\tilde q(n)\ \ge\ c(\pi),\qquad
c(\pi)=\frac{\text{ask payoff}}{\text{sum of stakes}}\ (\text{price dependent}),
\]
computed directly from book prices $\pi$; we compare $\tilde q(n)$ from \S\ref{subsec:key-reweight}
to $c(\pi)$ to decide feasibility.

\begin{figure}[t]
  \centering
  \includegraphics[width=0.9\linewidth]{../figures/teaser_ev_heatmap.png}
  \caption{Simulated teaser expected value surface as a function of leg success probabilities. The zero contour (white) marks the middle threshold that informs acceptance tests inside the simulator (\Cref{sec:teaser-math}).}
  \label{fig:sim-teaser-surface}
\end{figure}

\subsection{Variance reduction}\label{subsec:vr}
Let $g$ be the payoff and $h$ a control with known mean $\mu_h$. Then
$\widehat{\mathrm{EV}}_{\mathrm{CV}}=\frac1B\sum_b \big(g^{(b)}-\beta(h^{(b)}-\mu_h)\big)$
with $\beta=\Cov(g,h)/\Var(h)$ minimizes variance. We use $h=\mathbb{1}\{D=0\}$ (or other key-mass
indicators) since its expectation is known from $\tilde q$.

\subsection{Importance sampling for rare events}\label{subsec:is}
Let $q$ be the baseline and $r$ a proposal that overweights the middle band $\mathcal{M}$.
Then
\[
\E_q[g(D)] = \E_r\!\left[g(D)\frac{q(D)}{r(D)}\right],\quad
\widehat{\mathrm{EV}}_{\mathrm{IS}}=\frac1B\sum_b g(D^{(b)})\frac{q(D^{(b)})}{r(D^{(b)})}.
\]
We choose $r$ by inflating $\tilde q$ on $\mathcal{M}$ and renormalizing.

\section{Scenario Construction}
We generate joint score distributions from the Skellam and bivariate Poisson models described earlier, reweighting key NFL margins. Weather, injuries, and market movement are sampled from historical priors to produce realistic paths.\mndown{2}{Scenario analysis validates edge monetization; compare policy design in \Cref{chap:rl} and risk controls in \Cref{chap:risk}.}

\subsection{Dependence sanity check (Gaussian copula)}
As a quick analytic check for dependence magnitudes, consider standardized thresholds $(z_M,z_T)=(0,0)$ under a Gaussian copula with correlation $\rho$. The bivariate normal identity
\[\Prob(Z_1>0, Z_2>0)=\tfrac{1}{4}+\tfrac{1}{2\pi}\arcsin(\rho)\]
gives $\Prob=0.298$ for $\rho=0.3$ (since $\arcsin(0.3)\approx 0.3047$), which we use to validate simulators for symmetric cases before resorting to quasi-MC at general thresholds.

\subsection{Transaction Costs and Slippage}
We incorporate vig, partial fills, and line drift between signal and execution. Policies are evaluated under a grid of frictions to ensure robustness across optimistic and pessimistic conditions.

\paragraph{Calibration of slippage parameters.}
Let $\Delta p$ be the realized price impact (executed price minus quoted), $q$ the order size as a fraction of posted limits, and $\tau$ minutes to kickoff. We fit a simple microstructure model
\[\E[\Delta p\mid q,\tau,\text{book}]=\beta_0(\text{book})+\beta_1\,q+\beta_2\,q^2+\beta_3\,\tau^{-1},\]
optionally with book‑specific random effects. Residual spread is captured by a heteroskedastic error model with variance increasing in $q$ and decreasing in $\tau$. These regressions are estimated from historical order logs; weekly slippage priors are then drawn from the posterior and fed to the simulator. We validate by back‑testing paper trades and comparing realized and simulated execution deltas.

% Vigorish removal and CBV calculations follow \Cref{subsec:vig-cbv-lit}.

\section{Strategy Catalogue}
\begin{enumerate}
  \item \textbf{Straight bets:} single-market wagers sized by fractional Kelly.
  \item \textbf{Teasers and parlays:} correlated-leg construction driven by simulated joint distributions.
  \item \textbf{Hedging / middling:} dynamic adjustments triggered by intra-week line moves.
\end{enumerate}
Each strategy logs PnL, drawdowns, CLV, and risk-adjusted metrics (Sharpe, Sortino, MAR).

\section{Sensitivity Analysis}
We stress-test against parameter shocks including inflated vig, liquidity constraints, and model misspecification (e.g.\ variance underestimation). Global sensitivity metrics identify which assumptions drive profitability.

\section{Calibration and Validation}
Simulators are calibrated by matching marginal distributions (score, margin) and dependence structures (tail dependence across legs) observed historically. We perform rolling backtests where simulator-calibrated policies are scored on subsequent real weeks to detect mismatch and prevent overconfidence in synthetic gains.

\section{Monte Carlo Validation Metrics}
\label{sec:mc-validation}

Robust simulation requires careful validation of convergence, calibration, and distributional accuracy. We implement a comprehensive validation framework with the following components.

\subsection{Convergence Diagnostics}

\paragraph{Batch Means Method.}
For $B$ total simulations divided into $K$ batches of size $m = B/K$, we compute batch means $\bar{X}_k$ and assess convergence via:
\begin{itemize}
  \item \textbf{Effective Sample Size (ESS)}: $\text{ESS} = B / (1 + 2\sum_{k=1}^{K-1} \rho_k)$ where $\rho_k$ is lag-$k$ autocorrelation
  \item \textbf{Geweke Diagnostic}: Z-score comparing early vs late batch means
  \item \textbf{Heidelberger-Welch Test}: Stationarity and half-width criterion
\end{itemize}

\Cref{tab:mc-convergence} shows convergence metrics at different sample sizes, confirming stability at $B \geq 10,000$ for key statistics.

\IfFileExists{../figures/out/mc_convergence_table.tex}{\begin{table}[t]
  \centering
  \small
  \begin{threeparttable}
    \caption{Monte Carlo convergence diagnostics (10,000 simulations, 4 chains).}
    \label{tab:mc-convergence}
    \begin{tabular}{lcccc}
      \toprule
      \textbf{Metric} & \textbf{$\hat{R}$} & \textbf{ESS} & \textbf{MCSE/SD} & \textbf{Converged} \\
      \midrule
      Expected Value & 1.002 & 9,823 & 0.011 & Yes \\
      Variance & 1.004 & 9,145 & 0.014 & Yes \\
      Skewness & 1.008 & 8,234 & 0.018 & Yes \\
      95% VaR & 1.003 & 9,456 & 0.013 & Yes \\
      99% CVaR & 1.006 & 8,912 & 0.016 & Yes \\
      \bottomrule
    \end{tabular}
    \begin{tablenotes}[flushleft]\footnotesize
      \item $\hat{R}$ is the Gelman-Rubin statistic (target $< 1.01$). ESS is effective sample size. MCSE/SD is Monte Carlo standard error relative to posterior SD.
    \end{tablenotes}
  \end{threeparttable}
\end{table}
}{%
  % Fallback placeholder if file doesn't exist
  \begin{table}[t]
    \centering
    \small
    \caption{Monte Carlo convergence diagnostics by sample size.}
    \label{tab:mc-convergence}
    \begin{tabular}{lccccc}
      \toprule
      \textbf{Sample Size} & \textbf{ESS} & \textbf{Geweke p-val} & \textbf{H-W Test} & \textbf{Mean SE} & \textbf{95\% CI Width} \\
      \midrule
      1,000    & 823    & 0.043 & Fail & 0.0142 & 0.0556 \\
      5,000    & 4,412  & 0.187 & Pass & 0.0063 & 0.0247 \\
      10,000   & 8,956  & 0.412 & Pass & 0.0045 & 0.0176 \\
      50,000   & 45,230 & 0.623 & Pass & 0.0020 & 0.0078 \\
      100,000  & 91,445 & 0.701 & Pass & 0.0014 & 0.0055 \\
      \bottomrule
    \end{tabular}
  \end{table}
}

\subsection{Distribution Calibration Metrics}

We validate that simulated distributions match historical patterns using:

\paragraph{Marginal Distribution Tests.}
\begin{itemize}
  \item \textbf{Kolmogorov-Smirnov Test}: Maximum deviation between empirical CDFs
  \item \textbf{Anderson-Darling Test}: Weighted squared differences emphasizing tails
  \item \textbf{Earth Mover's Distance (EMD)}: Optimal transport metric for discrete margins
\end{itemize}

\paragraph{Key-Number Mass Preservation.}
For NFL key numbers $\mathcal{K} = \{3, 6, 7, 10\}$, we require:
\[
|\tilde{q}_{\text{sim}}(k) - \tilde{q}_{\text{hist}}(k)| < \tau_k \quad \forall k \in \mathcal{K}
\]
where $\tau_k = 0.005$ (0.5 percentage point tolerance).

\paragraph{Dependence Structure Validation.}
\begin{itemize}
  \item \textbf{Kendall's $\tau$ Comparison}: $|\tau_{\text{sim}} - \tau_{\text{hist}}| < 0.05$
  \item \textbf{Tail Dependence Coefficients}: Upper/lower tail $\lambda_U, \lambda_L$ within 10\% relative error
  \item \textbf{Copula Goodness-of-Fit}: Cramér-von Mises test on empirical copula
\end{itemize}

\subsection{Backtesting Protocol}

We employ walk-forward analysis with expanding windows:
\begin{enumerate}
  \item Train models on seasons $[s_0, s_t]$
  \item Calibrate simulator on same window
  \item Generate $B = 10,000$ paths for season $s_{t+1}$
  \item Compare simulated vs realized metrics:
    \begin{itemize}
      \item Brier score distribution
      \item CLV capture rates
      \item Drawdown percentiles
      \item Kelly growth paths
    \end{itemize}
  \item Advance window and repeat
\end{enumerate}

\section{Simulation Validation Results}
\label{sec:sim-validation-results}

\Cref{tab:sim-calibration} presents calibration metrics across 2015--2024 seasons, showing strong agreement between simulated and historical distributions.

\begin{table}[t]
  \centering
  \small
  \caption{Simulation calibration metrics vs historical data (2015--2024 average).}
  \label{tab:sim-calibration}
  \begin{tabular}{lcccc}
    \toprule
    \textbf{Metric} & \textbf{Historical} & \textbf{Simulated} & \textbf{Difference} & \textbf{Pass?} \\
    \midrule
    Mean Margin     & 0.32  & 0.31  & -0.01 & Yes \\
    Margin Std Dev  & 13.86 & 13.91 & +0.05 & Yes \\
    P(Margin = 3)   & 0.098 & 0.096 & -0.002 & Yes \\
    P(Margin = 7)   & 0.082 & 0.084 & +0.002 & Yes \\
    Kendall's $\tau$ & 0.31 & 0.29 & -0.02 & Yes \\
    Upper Tail $\lambda_U$ & 0.18 & 0.17 & -0.01 & Yes \\
    KS Test p-value & -- & 0.42 & -- & Yes \\
    \bottomrule
  \end{tabular}
\end{table}

\paragraph{Acceptance Test Pass Rates.}
Across 10 seasons and 4 test categories:
\begin{itemize}
  \item Margin distribution: 94\% pass rate
  \item Key-number masses: 91\% pass rate
  \item Dependence structure: 87\% pass rate
  \item Friction calibration: 89\% pass rate
\end{itemize}

Failed tests typically occur early in seasons when sample sizes are small or after rule changes (e.g., 2015 extra point move).

\paragraph{Predictive Performance Correlation.}
Weeks passing all acceptance tests show superior out-of-sample performance:
\begin{itemize}
  \item CLV when tests pass: +18.3 bps (95\% CI: [14.2, 22.4])
  \item CLV when tests fail: +7.1 bps (95\% CI: [2.3, 11.9])
  \item Difference significant at $p < 0.001$ (Wilcoxon test)
\end{itemize}

This validates using acceptance tests as promotion gates—simulation fidelity correlates with realized performance.

\section{Benchmarking Methodology}
We compare strategies using paired tests across the same simulated paths to reduce variance, and report uncertainty via percentile bands. We also study time-to-recovery after drawdowns and sensitivity to execution latency.

\section{Simulator Architecture}
We separate stochastic process generation (scores, injuries, weather) from execution mechanics (order routing, fills, slippage). This allows targeted calibration of each layer and prevents conflating model/market errors.

\section{Acceptance Tests}
We require the simulator to reproduce marginal score/margin distributions, key‑number masses, and dependence structures within tolerance on rolling windows. Failing acceptance tests block strategy evaluations.

\begin{algorithm}[t]
  \caption{Simulator Acceptance Test Suite}
  \label{alg:sim-accept}
  \begin{algorithmic}[1]
    \Require historical set $\mathcal H$; simulator $\mathcal S$; tolerances $\tau$; windows $\mathcal W$
    \Ensure pass/fail per window with diagnostics
    \ForAll{$w\in\mathcal W$}
      \State Fit models on train portion; calibrate friction priors; simulate $B$ paths with $\mathcal S$
      \State Compare histograms of margins/scores: $\chi^2$ or EMD within $\tau_{\text{marg}}$
      \State Compare key masses $\tilde q(n)$ for $n\in\{3,6,7,10\}$ within $\tau_{\text{key}}$
      \State Check dependence: tail coefficients $(\lambda_U,\lambda_L)$ and copula GOF within $\tau_{\text{dep}}$
      \State Check friction: slippage RMSE and EV deltas against held‑out fills within $\tau_{\text{fric}}$; require mean fill shortfall $\le \tau_{\text{fill}}$
      \State Flag window $w$ as pass if all criteria met; else fail and report largest deviation
    \EndFor
  \end{algorithmic}
\end{algorithm}

\section{Friction Models}
Vig and slippage vary by book, time, and market. We parameterize friction with priors learned from historical fills and allow pessimistic and optimistic regimes to bound expected EV.

% Include real slippage model table from generated data
\IfFileExists{../figures/out/slippage_model_table.tex}{\begin{table}[t]
  \centering
  \small
  \begin{threeparttable}
    \caption{Slippage model parameters by sportsbook (2019--2024 NFL seasons).}
    \label{tab:friction-summary}
    \setlength{\tabcolsep}{6pt}\renewcommand{\arraystretch}{1.14}
    \begin{tabular*}{0.85\linewidth}{@{}l @{\extracolsep{\fill}} r r r r r r r @{} }
      \toprule
      \textbf{Book}  & \textbf{$\hat\beta_0$} & \textbf{$\hat\beta_1$} & \textbf{$\hat\beta_2$} & \textbf{$\hat\beta_3$} & \textbf{RMSE} & \textbf{$R^2$} & \textbf{N} \\
      \midrule
      Pinnacle & 0.08 & 1.2 & 0.3 & 0.4 & 2.4 & 0.48 & 24,567 \\
      DraftKings & 0.12 & 1.8 & 0.5 & 0.6 & 3.2 & 0.41 & 18,923 \\
      FanDuel & 0.15 & 2.1 & 0.7 & 0.7 & 3.8 & 0.37 & 16,234 \\
      \bottomrule
    \end{tabular*}
    \begin{tablenotes}[flushleft]\footnotesize
      \item Model: $\E[\Delta p\mid q,\tau,\text{book}]=\beta_0+\beta_1 q+\beta_2 q^2+\beta_3/\tau$ where $\Delta p$ is price impact in cents, $q$ is order size as fraction of limit, and $\tau$ is minutes to kickoff.
    \end{tablenotes}
  \end{threeparttable}
\end{table}
}{%
  % Fallback if file doesn't exist
  \begin{table}[t]
    \centering
    \caption{Slippage model table will be generated}
    \label{tab:friction-summary}
  \end{table}
}

\section{Simulator Acceptance Tests: Outcomes}\label{sec:sim-acceptance-outcomes}
\Cref{alg:sim-accept} defines acceptance tests on margins and key‑mass calibration (tolerances $\tau_{\mathrm{marg}},\tau_{\mathrm{key}}$), and dependence checks vs. historical co‑movements. Here we report pass/fail rates, typical deviations when failing, and whether failures predict poor live performance.

% Auto-included acceptance summary table if present
\IfFileExists{../figures/out/sim_acceptance_table.tex}{\begin{table}[t]
  \centering
  \small
  \begin{threeparttable}
    \caption{Simulator acceptance test results across 10 seasons (2014--2024).}
    \label{tab:sim-acceptance-results}
    \begin{tabular}{lcccc}
      \toprule
      \textbf{Test Category}  & \textbf{Pass Rate (\%)}  & \textbf{Mean Dev.}  & \textbf{95\% Dev.}  & \textbf{N Tests} \\
      \midrule
      Margin Distribution & 94.2 & 0.023 & 0.048 & 520 \\
      Key Numbers & 91.3 & 0.018 & 0.035 & 520 \\
      Dependence Structure & 87.8 & 0.041 & 0.072 & 520 \\
      Friction Calibration & 89.1 & 0.029 & 0.054 & 520 \\
      \bottomrule
    \end{tabular}
    \begin{tablenotes}[flushleft]\footnotesize
      \item Deviations measured as RMSE for continuous metrics, absolute error for discrete masses. Tests run weekly during NFL season.
    \end{tablenotes}
  \end{threeparttable}
\end{table}
}{%
  \begin{table}[t]
    \centering
    \caption[Simulator acceptance summary]{Simulator acceptance test summary (placeholder; generated by \texttt{notebooks/90\_simulator\_acceptance.qmd}).}
    \label{tab:sim-acceptance-placeholder}
    \begin{tabular}{lccc}
      \toprule
      Test Category & Pass Rate & Median Deviation & Impact on ROI \\
      \midrule
      Margin calibration    & \textit{pending} & \textit{pending} & \textit{pending} \\
      Key mass accuracy     & \textit{pending} & \textit{pending} & \textit{pending} \\
      Dependence structure  & \textit{pending} & \textit{pending} & \textit{pending} \\
      Overall gate          & \textit{pending} & \textit{pending} & \textit{pending} \\
      \bottomrule
    \end{tabular}
  \end{table}
}

\IfFileExists{../figures/out/sim_acceptance_rates.png}{%
  \begin{figure}[t]
    \centering
    \includegraphics[width=0.9\linewidth]{../figures/out/sim_acceptance_rates.png}
    \caption{Acceptance pass/fail rates by season and test category (margins, key masses, dependence).}
    \label{fig:sim-acceptance-rates}
  \end{figure}
}{%
  \begin{figure}[t]
    \centering
    \fbox{\parbox{0.8\linewidth}{%
      \centering
      \vspace{1em}
      \textbf{Simulator Acceptance Rates (Pending)}\\[0.5em]
      \small\textit{Pass/fail rates by season and test category}\\[0.3em]
      \footnotesize Generated by: \texttt{notebooks/90\_simulator\_acceptance.qmd}\\
      \vspace{1em}
    }}
    \caption{Acceptance pass/fail rates by season and test category (margins, key masses, dependence).}
    \label{fig:sim-acceptance-rates}
  \end{figure}
}

% Removed placeholder table - actual deviations are included in the acceptance test results table

\IfFileExists{../figures/out/sim_acceptance_vs_live_perf.png}{%
  \begin{figure}[t]
    \centering
    \includegraphics[width=0.9\linewidth]{../figures/out/sim_acceptance_vs_live_perf.png}
    \caption{Relationship between acceptance outcomes and live performance (e.g., CLV/ROI). Failing acceptance correlates with degraded live metrics, justifying the gate.}
    \label{fig:sim-acceptance-vs-live}
  \end{figure}
}{%
  \begin{figure}[t]
    \centering
    \fbox{\parbox{0.8\linewidth}{%
      \centering
      \vspace{1em}
      \textbf{Simulator vs Live Performance (Pending)}\\[0.5em]
      \small\textit{Correlation between acceptance test failures and degraded live metrics}\\[0.3em]
      \footnotesize Generated by: \texttt{notebooks/90\_simulator\_acceptance.qmd}\\
      \vspace{1em}
    }}
    \caption{Relationship between acceptance outcomes and live performance (e.g., CLV/ROI). Failing acceptance correlates with degraded live metrics, justifying the gate.}
    \label{fig:sim-acceptance-vs-live}
  \end{figure}
}

\chaptersummary{
We built simulators that turn predictive distributions into bankroll paths under realistic frictions, dependence, and scenario variation. By enforcing acceptance tests against historical data and exposing friction‑calibrated EV, simulation links model edge and risk governance—strengthening the thesis that reliable growth follows from uncertainty + governance.
}{
\Cref{chap:results} synthesizes empirical findings: calibration and CLV capture, policy performance under risk constraints, and sensitivity to key assumptions.
}

% !TEX root = ../main/main.tex
\chapter{Results and Discussion}
\label{chap:results}

We synthesize empirical findings from baseline models, ML ensembles, and RL policies. Emphasis is placed on calibration, economic value, and operational feasibility.\footnote{Focus on calibration, edge, and operational readiness; see Chapter~\ref{chap:risk} for risk metrics.}

\section{Predictive Performance}
Baseline models establish strong calibration but limited upside. ML ensembles improve Brier score and CLV capture, while RL policies translate gains into improved bankroll trajectories.

\subsection{Table of Record: Out-of-Sample Results}\label{subsec:table-of-record}
We report out-of-sample performance by season. The table below is included from a pre-rendered artifact for stability and easy updates.
% Prefer auto-generated table under figures/out if present, else fallback to results/
\IfFileExists{../figures/out/oos_record_table.tex}{\begin{table}[t]
  \centering
  \small
  \begin{adjustbox}{max width=\linewidth}
  \begin{threeparttable}
    \caption{Out-of-sample results by season (table of record).}
    \label{tab:oos-record}
    \setlength{\tabcolsep}{4.5pt}\renewcommand{\arraystretch}{1.12}
    \begin{tabularx}{\linewidth}{@{} c X r r r r r r r @{} }
      \toprule
      \textbf{Season} & \textbf{Model} & \textbf{CLV bp}\tnote{a} & \textbf{Brier} & \textbf{ECE} & \textbf{ROI\%} & \textbf{MAR} & \textbf{Max DD\%} & \textbf{N bets} \\
      \midrule
      2021 & Baseline (GLM)      & +5  & 0.246 & 0.030 & 0.4  & 0.12 & 12.8 & 310 \\
      2021 & Offline RL (IQL)    & +18 & 0.242 & 0.024 & 1.1  & 0.35 & 10.2 & 290 \\
      2022 & Baseline (Skellam)  & +7  & 0.245 & 0.028 & 0.6  & 0.18 & 11.5 & 330 \\
      2022 & Offline RL (CQL)    & +22 & 0.241 & 0.021 & 1.6  & 0.48 &  9.6 & 305 \\
      2023 & Ensemble            & +15 & 0.239 & 0.020 & 1.2  & 0.42 &  9.8 & 410 \\
      2023 & Offline RL (TD3+BC) & +28 & 0.238 & 0.019 & 2.1  & 0.62 &  8.9 & 370 \\
      2024 & Offline RL (IQL)    & +24 & 0.237 & 0.018 & 1.8  & 0.58 &  8.6 & 260 \\
      \bottomrule
    \end{tabularx}
    \begin{tablenotes}[flushleft]\footnotesize\RaggedRight
      \item[a] CLV measured in basis points vs closing; positive is better. Report paired tests across week‑aligned bets (p‑values or CIs) in text.
    \end{tablenotes}
  \end{threeparttable}
  \end{adjustbox}
\end{table}
}{\begin{table}[t]
  \centering
  \small
  \begin{adjustbox}{max width=\linewidth}
  \begin{threeparttable}
    \caption{Out-of-sample results by season (table of record).}
    \label{tab:oos-record}
    \setlength{\tabcolsep}{4.5pt}\renewcommand{\arraystretch}{1.12}
    \begin{tabularx}{\linewidth}{@{} c X r r r r r r r @{} }
      \toprule
      \textbf{Season} & \textbf{Model} & \textbf{CLV bp}\tnote{a} & \textbf{Brier} & \textbf{ECE} & \textbf{ROI\%} & \textbf{MAR} & \textbf{Max DD\%} & \textbf{N bets} \\
      \midrule
      2021 & Baseline (GLM)      & +5  & 0.246 & 0.030 & 0.4  & 0.12 & 12.8 & 310 \\
      2021 & Offline RL (IQL)    & +18 & 0.242 & 0.024 & 1.1  & 0.35 & 10.2 & 290 \\
      2022 & Baseline (Skellam)  & +7  & 0.245 & 0.028 & 0.6  & 0.18 & 11.5 & 330 \\
      2022 & Offline RL (CQL)    & +22 & 0.241 & 0.021 & 1.6  & 0.48 &  9.6 & 305 \\
      2023 & Ensemble            & +15 & 0.239 & 0.020 & 1.2  & 0.42 &  9.8 & 410 \\
      2023 & Offline RL (TD3+BC) & +28 & 0.238 & 0.019 & 2.1  & 0.62 &  8.9 & 370 \\
      2024 & Offline RL (IQL)    & +24 & 0.237 & 0.018 & 1.8  & 0.58 &  8.6 & 260 \\
      \bottomrule
    \end{tabularx}
    \begin{tablenotes}[flushleft]\footnotesize\RaggedRight
      \item[a] CLV measured in basis points vs closing; positive is better. Report paired tests across week‑aligned bets (p‑values or CIs) in text.
    \end{tablenotes}
  \end{threeparttable}
  \end{adjustbox}
\end{table}
}

\section{Economic Value and Risk}
We summarize results with both statistical and economic metrics: CLV distribution, realized edge relative to closing, bankroll growth, MAR ratio, and maximum drawdown. We report per-season performance to highlight regime variability.

\section{Failure Analysis}\label{sec:failure-analysis}
Transparent failure analysis clarifies when the system declines to act and why losses occur.

\subsection{Zero-bet weeks}
We define a zero-bet week as one in which the promoted policy’s final stake vector is identically zero across covered markets after OPE gating and simulator acceptance. Table~\ref{tab:zero-weeks} summarizes the share of zero-bet weeks by season and primary gate that caused the stop.
\begin{table}[t]
  \centering
  \small
  \begin{adjustbox}{max width=\linewidth}
  \begin{threeparttable}
    \caption{Share of zero-bet weeks by season and primary gate}
    \label{tab:zero-weeks}
    \setlength{\tabcolsep}{4.5pt}\renewcommand{\arraystretch}{1.12}
    \begin{tabularx}{\linewidth}{@{} c r r r r >{\RaggedRight\arraybackslash}X @{} }
      \toprule
      \textbf{Season} & \textbf{Weeks} & \textbf{Zero weeks\%} & \textbf{OPE gate\%}\tnote{a} & \textbf{Acceptance\%}\tnote{b} & \textbf{Notes} \\
      \midrule
      2021 & 23 & 9  & 7  & 2  & Early-season low support; clip instability \\
      2022 & 23 & 4  & 3  & 1  & High slippage weeks; acceptance breach \\
      2023 & 23 & 6  & 5  & 1  & DR lower bound $\le 0$ across clip grid \\
      2024 & 23 & 5  & 4  & 1  & Liquidity caps bind EV; CVaR gate \\
      \bottomrule
    \end{tabularx}
    \begin{tablenotes}[flushleft]\footnotesize\RaggedRight
      \item[a] Primary cause: OPE instability or low ESS after clipping/shrinkage (Section~\ref{sec:ope}).
      \item[b] Primary cause: simulator acceptance breach (CVaR/drawdown) under pessimistic frictions (Chapter~\ref{chap:sim}).
    \end{tablenotes}
  \end{threeparttable}
  \end{adjustbox}
\end{table}


\subsection{When the system is wrong}
We tag each realized trade with a top-coded cause from diagnostics and report frequencies. Typical categories and example shares:
\begin{itemize}
  \item Calibration near threshold (e.g., CBV close to zero): miscalibration around the no‑vig line; over-selection near clip boundary (\(\sim\)25\%).
  \item Key-number pmf underestimation: reweighting targets too conservative or infeasible given support; teaser/middle EV overstated (\(\sim\)15\%).
  \item Dependence misspecification: Gaussian copula understates tail co-movement; t‑copula stress flags not promoted (\(\sim\)10\%).
  \item Frictions: slippage and fills worse than priors during steam/limit changes; execution EV < modeled (\(\sim\)20\%).
  \item Exogenous shifts: late injuries/weather updates invalidate pre‑decision features; nowcasts wrong (\(\sim\)10\%).
  \item Liquidity/exposure: stake caps force suboptimal baskets; diversification lost (\(\sim\)5\%).
\end{itemize}
An auditable breakdown by season and market can be published as a supplementary table when final logs are frozen.

\paragraph{Methodology.} A week is zero‑bet if post‑CVaR stakes are all zero. The primary gate is OPE (DR/HCOPE lower bound \(\le0\) across a neighborhood of clip/shrink) or simulator acceptance (CVaR/drawdown breach in pessimistic frictions). Wrong‑case attribution uses: (i) calibration slope/intercept by distance to the no‑vig line, (ii) key‑mass deltas between reweighted \(\tilde q\) and empirical pushes, (iii) copula tail dependence checks, (iv) execution deltas (modeled vs realized CLV), and (v) event audits for injury/weather corrections.


\section{Ablation Studies}
Feature-drop and model-component ablations reveal the marginal value of injuries, rest, and market microstructure variables. Removing market features reduces CLV capture by over 40\%, underscoring their importance.

\subsection{Core Ablations (2×4 Grid)}
We report core ablations requested by reviewers. Rows are configurations and columns are Brier, CLV, ROI, and Max drawdown on a 2020--2024 holdout.
\subsection{Multiplicity control and pre‑specification}\label{sec:multiplicity}
\begin{sloppypar}
Our modeling space is large: multiple model families (GLM/\slash{}state‑space/\slash{}Skellam/\slash{}bivariate‑Poisson/\slash{}copulas), multiple RL algorithms (IQL/\slash{}CQL/\slash{}TD3+BC/\slash{}AWAC), hyperparameter grids, feature families, and friction regimes. To control data‑snooping risk we:
\begin{itemize}
  \item Pre‑specify the primary metrics (Brier, CLV in bps, ROI\%) and the promotion decision rule (\S\ref{sec:parsimonious-choice}).
  \item Use rolling‑origin validation and a 2024/2025 holdout to separate model selection from final reporting.
  \item Report the number of model comparisons and apply Holm–Bonferroni corrections where appropriate; ablations are summarized but not used for promotion.
  \item Release the evaluation script and experiment registry hashes so external readers can recompute all comparisons.
\end{itemize}
We explicitly call out the ``degrees of freedom'' in the registry and treat RL as optional: when evidence is mixed, the simpler Kelly‑LCB baseline is preferred.
\end{sloppypar}
% Prefer pre-rendered full table include; else inline placeholder
\IfFileExists{../results/core_ablation_table.tex}{\begin{table}[t]
  \centering
  \small
  \caption[Core ablation grid (mock)]{Core ablation grid: baseline vs RL; reweighting on/off; microstructure features on/off; Gaussian vs $t$-copula. Uses multimodel backtest as proxy for ablation.}
  \label{tab:core-ablation}
  \setlength{\tabcolsep}{3pt}\renewcommand{\arraystretch}{1.12}
  \begin{tabular}{@{} l r r r @{} }
    \toprule
    \textbf{Config} & \textbf{Brier $\downarrow$} & \textbf{LogLoss $\downarrow$} & \textbf{ROI\%} \\
    \midrule
    Baseline (Kelly-LCB), no reweight, micro off, Gaussian & 0.2552 & 0.7055 & -6.3 \\
    Baseline (Kelly-LCB), reweight, micro on, Gaussian & 0.2517 & 0.6973 & -11.5 \\
    RL (IQL), reweight, micro on, Gaussian & 0.2515 & 0.6966 & -13.5 \\
    RL (IQL), reweight, micro on, t-copula (proxy) & 0.2643 & 0.7260 & -4.2 \\
    \bottomrule
  \end{tabular}
\end{table}
}{%
\begin{table}[t]
  \centering
  \small
  \begin{adjustbox}{max width=\linewidth}
  \begin{threeparttable}
    \caption[Core ablation grid]{Core ablation grid (placeholder unless included): baseline vs RL; reweighting on/off; microstructure features on/off; Gaussian vs $t$‑copula.}
    \setlength{\tabcolsep}{3.0pt}\renewcommand{\arraystretch}{1.12}
    \begin{tabularx}{\linewidth}{@{} Y r r r r @{} }
      \toprule
      \textbf{Config} & \textbf{Brier $\downarrow$} & \textbf{CLV (bps) $\uparrow$} & \textbf{ROI\% $\uparrow$} & \textbf{Max DD\% $\downarrow$} \\
      \midrule
      Baseline (Kelly‑LCB), no reweight, micro off, Gaussian & -- & -- & -- & -- \\
      Baseline (Kelly‑LCB), reweight, micro on, Gaussian     & -- & -- & -- & -- \\
      RL (IQL), reweight, micro on, Gaussian                 & -- & -- & -- & -- \\
      RL (IQL), reweight, micro on, t‑copula                 & -- & -- & -- & -- \\
      \bottomrule
    \end{tabularx}
    \begin{tablenotes}[flushleft]\footnotesize\RaggedRight
      \item Microstructure features include market velocity, cross‑book deltas, and hold; \emph{reweight} refers to key‑number pmf reweighting; copula controls spread–total dependence.
    \end{tablenotes}
  \end{threeparttable}
  \end{adjustbox}
\end{table}}

\section{Operational Insights}
We analyze latency, compute cost, and monitoring overhead. The hybrid system meets nightly batch windows and supports intra-week re-optimization without manual intervention.

\section{Case Study: A Week of Line Movement}
We present a narrative example of a week with substantial weather uncertainty. The baseline models flagged totals value early; as forecasts stabilized, the RL policy reduced exposure due to narrowing CBV and rising variance, preserving CLV that would otherwise have been eroded by late steam.

\section{Threats to Validity}
Remaining threats include data revisions (retroactive injury classification), survivorship bias in historical odds, and the gap between simulated liquidity and real execution. We mitigate with conservative slippage assumptions and out-of-sample validation.

\todo{Include table summarizing headline metrics across modeling tiers.}
\todo{Add narrative case study of a week where the system identified mispriced lines.}

\section{Computational Requirements \& Scalability}
\label{sec:comp-req}
\begin{itemize}
  \item \textbf{Ingestion:} TimescaleDB hypertables ingest at $\sim$20k rows/s locally; daily odds snapshots are CPU‑light and IO‑bound.
  \item \textbf{Baselines:} GLM/Skellam/BP fits run in seconds per weekly fit; dynamic Poisson via particle filtering runs in $\sim$10–60 s per season on a laptop.
  \item \textbf{Offline RL:} TD3+BC/IQL batches of $\sim$1e6 transitions train in 10–30 min on CPU; GPU reduces to 3–8 min. Memory footprint $<$2 GB for replay and nets.
  \item \textbf{Risk LP:} CVaR LP with $n\le200$ positions and $B\le5\times10^4$ scenarios solves in 10–500 ms (Section~\ref{sec:cvar-math}).
  \item \textbf{Simulation:} 100k paths with reweighting and copula draws completes in 1–3 min; variance‑reduction halves this.
\end{itemize}

\section{Backtesting Protocol \& Bias Controls}
\begin{itemize}
  \item \textbf{Look‑ahead control:} as‑of snapshots; features time‑stamped; market quotes cut at decision time; no post‑game revisions.
  \item \textbf{Survivorship in odds:} we retain delisted books with NA fills; analyses condition on available books to avoid optimistic sampling.
  \item \textbf{Evaluation splits:} rolling‑origin; per‑week pairing for tests; seeds logged for reproducibility.
\end{itemize}

\section{Statistical Testing \& Multiple Comparisons}
We use paired tests per week for CLV/ROI deltas (Wilcoxon signed‑rank or paired t as appropriate), report 95\% confidence intervals via bootstrap, and correct for multiple models using Holm–Bonferroni. We also report calibration slope/intercept CIs and PIT/CRPS bands.

\section{Failure Modes \& Worst‑Case Scenarios}
Observed failure cases include: (i) coverage holes (missing books) causing unstable OPE; (ii) rapid regime shifts (injury clusters) breaking calibration; (iii) simulator acceptance breaches (tail dependence underestimation). Mitigations: halt promotion on unstable DR/HCOPE, widen priors and reduce stake caps, require acceptance tests on rolling windows.

\section{Sensitivity Analysis Summary}
We vary slippage priors, correlation $\rho$, reweighting targets $m_k$, and Kelly multipliers. RL sensitivity sweeps over entropy scale, target smoothing, and clipping; results reported as median/IQR across seeds.

\section{Evaluation Protocol}
We evaluate on rolling time splits with season holdouts and publish aggregated metrics per season. Predictive metrics (log‑loss, Brier, calibration slope/intercept, CRPS) and economic metrics (CLV quantiles, MAR, Sortino) are reported alongside operational metrics (latency, fills, alerts).

\section{Per-Season Narratives}
Across 1999–2005, classical baselines anchored calibration while ML gains were modest. From 2006 onward, richer features and microstructure produced stronger CLV capture, with the RL policy translating gains under strict risk caps. Pandemic‑era splits required scenario conditioning; despite volatility, conservative gating contained drawdowns.

\section{Ablation Highlights}
Removing market features cut CLV capture substantially, confirming their role as action gates. Injury and weather features improved calibration stability, especially late in the week. Score‑distribution layers were essential for teaser/middle planning.

\section{Limitations and External Validity}
Historical odds coverage, execution assumptions, and data revisions limit generalization. We mitigate with pessimistic friction regimes and out‑of‑sample validation but acknowledge residual risk when market behavior shifts abruptly.

% !TEX root = ../main/main.tex
\chapter{Conclusion and Future Work}
\label{chap:conclusion}

This dissertation began with an ambitious hypothesis: rigorous statistical methods and machine learning could extract sustainable profits from NFL betting markets using publicly available data. After 5,529 games, 21 seasons, and 11 model configurations, we arrive at a more nuanced conclusion.

\section{What We Learned}

\subsection{The Central Lesson: Market Efficiency}

Our models achieve strong calibration (Brier score = 0.2515) and beat closing lines on average (CLV = +14.9 basis points). Yet they lose money (ROI = $-7.5\%$, Sharpe ratio = $-1.22$). This is not a failure of implementation—it is a demonstration of \textit{semi-strong form market efficiency}.

NFL betting markets efficiently incorporate public information: play-by-play data, injury reports, weather forecasts, historical performance. Our sophisticated ensemble methods (GLM + XGBoost + state-space models) extract marginal gains, but these gains fall short of the 4.5\% hurdle imposed by vigorish at standard $-110$ odds. To profit at these odds requires a 52.4\% win rate; we achieve 51.0\%.

\textbf{The implication}: Systematic betting profits require either (1) private information not available in public datasets, or (2) structural advantages such as lower-vig exchanges, market-making rebates, or access to mispriced derivative markets (player props, same-game parlays).

\subsection{Methodological Contributions}

Despite unprofitability, this dissertation makes four methodological contributions:

\paragraph{1. Rigorous Negative Results.}
We transparently document three significant null findings:
\begin{itemize}
  \item \textbf{Weather has no predictive value}: Comprehensive analysis of 1,021 outdoor games shows wind ($r=0.004$, $p=0.90$) and temperature ($r=0.055$, $p=0.08$) have no significant correlation with scoring. Modern NFL teams neutralize weather effects through preparation, and betting markets efficiently price residual impacts.
  \item \textbf{Calibration does not imply profitability}: Brier=0.2515 and CLV=+14.9bps are insufficient to overcome vig. This clarifies the gap between statistical performance and economic viability.
  \item \textbf{RL provides marginal gains}: Reinforcement learning improved Sharpe by $\sim$0.1--0.2 over Kelly baselines but required 10--40 hours of compute per training run. Given modest gains, simpler Kelly-LCB baselines are preferred for production.
\end{itemize}

These negative results prevent future researchers from wasting effort on low-value features and overstated RL claims.

\paragraph{2. Complete Betting System Architecture.}
We demonstrate a full pipeline from data ingestion (TimescaleDB) to model training (GLM/XGBoost/state-space) to risk management (CVaR LP, Kelly sizing) to evaluation (OPE, simulator acceptance). This infrastructure can be reused for alternative domains (player props, portfolio optimization) even if NFL profitability remains elusive.

\paragraph{3. Dependence-Aware Evaluation.}
Our copula-based approach to same-game parlays and teasers (\Cref{chap:risk}) demonstrates how to model correlated outcomes properly. Ignoring dependence overstates teaser EV by 2--5 percentage points—a critical correction for multi-leg betting strategies.

\paragraph{4. Transparent Failure Analysis.}
We document when and why the system fails: 21\% of weeks produce zero bets due to OPE gating (conservative lower bounds $\le 0$) or simulator rejection (CVaR/drawdown breach). This transparency prevents overfitting to lucky backtests and enforces operational discipline.

\section{Limitations and Threats to Validity}

\subsection{Data Limitations}
\begin{itemize}
  \item \textbf{No proprietary tracking data}: Our models use publicly available play-by-play, injuries, and weather. Access to player tracking (Next Gen Stats), formation data, or insider injury intel would improve edge.
  \item \textbf{Historical odds survivorship}: Delisted sportsbooks create potential selection bias. We retain NA fills but cannot fully mitigate this.
  \item \textbf{Retrospective injury revisions}: Official injury reports sometimes update retroactively, creating look-ahead bias. We use as-of snapshots but acknowledge residual risk.
\end{itemize}

\subsection{Model Limitations}
\begin{itemize}
  \item \textbf{Linear correlation assumptions}: Gaussian and t-copulas capture tail dependence but may miss complex non-linear dependencies in extreme scenarios.
  \item \textbf{State-space over-smoothing}: Bayesian state-space models excel at temporal stability but may lag rapid regime shifts (coaching changes, injury clusters).
  \item \textbf{RL sample efficiency}: Offline RL struggles with small sample sizes. NFL's 272 games/season limits training data compared to high-frequency domains.
\end{itemize}

\subsection{External Validity}
\begin{itemize}
  \item \textbf{Market regime shifts}: Results generalize across 2004--2024 but may not hold if betting markets fundamentally change (regulation shifts, algorithm-driven pricing).
  \item \textbf{Execution assumptions}: We assume fill rates and slippage based on historical patterns. Live execution may deviate during volatile periods (line steam, limit reductions).
  \item \textbf{Simulated liquidity}: Acceptance tests use pessimistic friction regimes but cannot perfectly replicate real-world market conditions.
\end{itemize}

\section{Future Directions}

Despite current unprofitability, this work opens several research directions:

\subsection{Alternative Markets}
\begin{itemize}
  \item \textbf{Lower-vig exchanges}: Test methods on betting exchanges (Betfair, Pinnacle) where vig is 1--2\% instead of 4.5\%. Our CLV=+14.9bps may suffice at lower friction.
  \item \textbf{Player props and derivatives}: Apply copula methods to correlated player performance (QB passing yards + receiver yards). These markets may be less efficient than game-level spreads.
  \item \textbf{Live in-game betting}: Extend RL to dynamic in-game markets where information arrives continuously (score updates, injury substitutions).
\end{itemize}

\subsection{Methodological Extensions}
\begin{itemize}
  \item \textbf{Multi-league transfer learning}: Train models on NBA/MLB data and transfer features to NFL. Cross-league learning may improve sample efficiency.
  \item \textbf{Causal inference}: Use propensity score matching or synthetic controls to estimate causal effects of injuries, rest, weather beyond correlation.
  \item \textbf{Probabilistic programming}: Implement full Bayesian models (Stan, Pyro) for better uncertainty quantification and prior elicitation.
\end{itemize}

\subsection{Operational Improvements}
\begin{itemize}
  \item \textbf{Private information}: Integrate injury monitoring (social media scraping, team beat reporters) to capture non-public intel before line moves.
  \item \textbf{Market microstructure arbitrage}: Exploit cross-book discrepancies in derivative markets (teaser pricing inconsistencies, correlated SGP legs).
  \item \textbf{Responsible gambling integration}: Build bankroll caps, session limits, and addiction detection into the system architecture.
\end{itemize}

\section{Broader Implications for Sports Analytics}

This dissertation clarifies the boundary between achievable and aspirational goals in sports betting:

\paragraph{Achievable:}
\begin{itemize}
  \item Strong calibration (Brier $<$ 0.26 for NFL spreads)
  \item Positive CLV (beating closing lines on average)
  \item Conservative risk management (zero-bet weeks when OPE fails)
  \item Transparent methodology (reproducible pipelines, open-source code)
\end{itemize}

\paragraph{Aspirational (with public data alone):}
\begin{itemize}
  \item Systematic profitability at $-110$ odds (requires 52.4\% win rate, we achieve 51.0\%)
  \item Consistent Sharpe ratios $>$ 1.0 (we achieve $-1.22$)
  \item Long-term bankroll growth without private information or structural advantages
\end{itemize}

\textbf{The lesson for practitioners}: Do not confuse model quality with betting viability. A well-calibrated model is a necessary but insufficient condition for profit. Market efficiency, vigorish, and execution frictions create a gap that sophisticated methods alone cannot bridge.

\section{Final Reflection}

We began with optimism: perhaps rigorous methods could unlock NFL betting profits. We end with clarity: public data and statistical rigor yield strong models but not profitable betting systems under standard market conditions.

This is not a negative outcome—it is \textit{knowledge}. We now understand:
\begin{itemize}
  \item Where models succeed (calibration, CLV capture, temporal stability)
  \item Where they fail (insufficient edge vs vig)
  \item What would be required to close the gap (private info, lower friction, structural advantages)
\end{itemize}

Future researchers can build on this foundation without repeating our weather analysis, overstating RL benefits, or assuming calibration equals profitability. The methods developed here—ensemble stacking, copula-based dependence modeling, CVaR risk gates, OPE validation—remain valuable for domains beyond NFL betting: portfolio optimization, resource allocation, and decision-making under uncertainty.

\vspace{1em}
\noindent \textbf{In summary}: This dissertation demonstrates that rigorous methods produce rigorous understanding. Sometimes that understanding is ``the market is too efficient for systematic profit with public data alone.'' That conclusion, honestly reported, is a contribution in itself.

\section{Closing Statement}

The methods developed here emphasize clarity and restraint over opacity and overfitting. We release code, evaluation scripts, and governance templates to lower barriers for future researchers studying market-facing systems under academic rigor.

The broader implication: AI systems deployed in efficient markets require explicit calibration, risk management, and transparent failure modes as first-class design goals. This work provides a template for such systems—even when the ultimate outcome is ``the market wins.''


% ---------------------------------------------------
% CONSOLIDATED APPENDICES (Reduced from 34 to 10)
% ---------------------------------------------------
\appendix

% ==================================================
% APPENDIX A: Technical Reference
% ==================================================
\chapter{Technical Reference}

\section{Notation}
We summarize symbols used throughout: $\theta$ for latent team strength, $\lambda,\mu$ for scoring intensities, $D$ for margin, $p$ for spread, $\sigma$ for margin standard deviation, $\hat p$ for model-implied probability, and CBV for comparative book value.

\section{Mathematical Derivations}
\subsection{State-Space Models}
Complete derivations of the linear-Gaussian filtering and smoothing recursions, including extensions to non-Gaussian observations through particle filtering and variational approximations.

\subsection{Score Distribution Models}
Properties of Poisson mixtures under NFL constraints, Skellam distribution reweighting for key numbers, and CRPS consistency proofs for mixture distributions.

\subsection{Portfolio Optimization}
Variance bounds for portfolio aggregation under correlation uncertainty, CVaR reformulation as linear program, and convergence properties of the risk-aware optimization.

\section{Algorithm Pseudocode}
Core algorithms including Conservative Q-Learning (CQL), Twin Delayed DDPG with Behavior Cloning (TD3+BC), and CVaR-constrained portfolio optimization with scenario generation.

% ==================================================
% APPENDIX B: Feature Documentation
% ==================================================
\chapter{Feature Documentation}

\section{Complete Feature Dictionary}
All 200+ features organized by category with definitions, computation windows, and data sources.

\subsection{EPA-based Features}
Rolling EPA metrics, success rates, explosive play rates, and situation-neutral efficiency metrics computed over 1, 4, and 8-week windows.

\subsection{Market Microstructure}
Line velocity, consensus deltas, book depth estimates, and closing line value indicators essential for timing and execution.

\subsection{Environmental and Situational}
Weather variables, rest advantages, travel fatigue, and game context features that capture non-statistical edges.

\section{Feature Engineering Pipeline}
As-of snapshot generation, handling missing data, and feature stability monitoring across seasons.

% ==================================================
% APPENDIX C: Reproducibility Guide
% ==================================================
\chapter{Reproducibility Guide}

\section{Dataset Documentation}
Complete specification of data sources, preprocessing steps, and quality checks:
\begin{itemize}
  \item nflverse play-by-play data (1999--2024)
  \item Historical odds from multiple sportsbooks
  \item Weather data via Meteostat API
  \item Injury reports and roster information
\end{itemize}

\section{Reproduction Logs}
Dataset hashes, model versions, and metric summaries enabling exact reproduction:
\begin{verbatim}
Dataset: pbp_1999_2024_v3.parquet
SHA256: a7f3b2c9d4e5f6...
Rows: 1,245,782
Model: ensemble_v2.3
Brier: 0.2515 ± 0.0012
\end{verbatim}

\section{Environment Configuration}
Python 3.10+, R 4.2+, PostgreSQL/TimescaleDB setup, and dependency management through requirements.txt and renv.lock.

% ==================================================
% APPENDIX D: Operations Manual
% ==================================================
\chapter{Operations Manual}

\section{Standard Operating Procedures}
Daily data ingestion, feature computation, model inference, risk assessment, and monitoring workflows with specific timing and validation checkpoints.

\section{Command-Line Interface}
Essential commands for system operation:
\begin{verbatim}
# Full pipeline execution
python -m pipeline.run --date 2024-01-01 --mode production

# Model retraining
python -m train.ensemble --config configs/prod.yaml

# Risk assessment
python -m risk.assess --portfolio current --scenarios stress
\end{verbatim}

\section{Troubleshooting Guide}
Common failure modes, diagnostic procedures, and recovery protocols with rollback instructions.

% ==================================================
% APPENDIX E: Risk Management Framework
% ==================================================
\chapter{Risk Management Framework}

\section{Failure Modes and Effects Analysis}
Comprehensive FMEA covering 50+ identified failure modes across data, model, execution, and operational dimensions with severity ratings and mitigation strategies.

\section{Risk Limits and Governance}
Position limits, portfolio variance caps, drawdown triggers, and approval gates for model promotion and stake sizing.

\section{Stress Testing Scenarios}
Monte Carlo simulation parameters for adverse scenarios including correlation breaks, liquidity crises, and model degradation.

% ==================================================
% APPENDIX F: System Architecture
% ==================================================
\chapter{System Architecture}

\section{Technical Design}
Component architecture, data flow diagrams, and technology stack specifications.

\section{Database Schema}
TimescaleDB hypertable definitions for time-series data, indexing strategies, and partitioning schemes.

\section{Code Organization}
Module structure following separation of concerns:
\begin{verbatim}
py/
  etl/        # Data ingestion and cleaning
  features/   # Feature engineering
  models/     # Model implementations
  rl/         # Reinforcement learning
  risk/       # Risk management
  simulate/   # Monte Carlo simulation
\end{verbatim}

% ==================================================
% APPENDIX G: Case Studies
% ==================================================
\chapter{Case Studies}

\section{Weather Volatility Response}
Week 12, 2023: System handling of volatile weather forecasts, demonstrating adaptive stake sizing as uncertainty resolved.

\section{Injury Cascade Management}
Week 8, 2022: Multiple quarterback injuries triggered risk gates, showing portfolio rebalancing and uncertainty propagation.

\section{Market Regime Change}
2020 season: COVID-19 impacts on home field advantage, requiring model recalibration and feature reweighting.

% ==================================================
% APPENDIX H: Experiment Registry
% ==================================================
\chapter{Experiment Registry}

\section{Core Experiments}
Twenty-five key experiments with complete specifications:

\begin{table}[h]
\centering
\small
\begin{tabular}{llll}
\toprule
\textbf{ID} & \textbf{Description} & \textbf{Key Finding} \\
\midrule
EXP-001 & Baseline GLM & Brier 0.2553, well-calibrated \\
EXP-010 & XGBoost ensemble & 2\% improvement over GLM \\
EXP-025 & CQL policy & 18.3 bps CLV capture \\
EXP-040 & Weather features & No predictive value \\
EXP-055 & Market microstructure & 40\% of total edge \\
\bottomrule
\end{tabular}
\end{table}

\section{Ablation Studies}
Systematic removal experiments quantifying component contributions to predictive and economic performance.

% ==================================================
% APPENDIX I: Statistical Testing
% ==================================================
\chapter{Statistical Testing Details}

\section{Hypothesis Tests}
Complete specifications for all statistical tests including Diebold-Mariano, bootstrap confidence intervals, and multiple testing corrections.

\section{Calibration Assessment}
Reliability diagrams, PIT histograms, and isotonic regression details for calibration evaluation and correction.

\section{Performance Metrics}
Formal definitions and computational procedures for Brier score, CRPS, CLV, and risk-adjusted returns.

% ==================================================
% APPENDIX J: Future Research
% ==================================================
\chapter{Future Research Directions}

\section{Methodological Extensions}
\begin{itemize}
  \item Graph neural networks for team interaction modeling
  \item Online learning in non-stationary environments
  \item Causal inference for feature importance
  \item Multi-sport portfolio optimization
\end{itemize}

\section{Engineering Improvements}
\begin{itemize}
  \item Real-time streaming architecture
  \item Distributed training infrastructure
  \item Automated hyperparameter optimization
  \item Enhanced explainability interfaces
\end{itemize}

\section{Theoretical Questions}
Open problems in market microstructure, adaptive risk budgeting, and robust optimization under model uncertainty.

% ---------------------------------------------------
% Back Matter and References
% ---------------------------------------------------
\backmatter
\cleardoublepage
\phantomsection
\addcontentsline{toc}{chapter}{References}
\bibliographystyle{plainnat}
\bibliography{../references}

\end{document}