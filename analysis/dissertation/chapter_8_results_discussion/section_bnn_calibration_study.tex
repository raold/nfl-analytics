\section{Bayesian Neural Network Calibration: A Systematic Investigation}
\label{sec:bnn_calibration_study}

\subsection{Motivation and Research Questions}
\label{subsec:bnn_calibration_motivation}

The promise of Bayesian deep learning lies not merely in point predictions, but in the principled quantification of uncertainty. For high-stakes decision-making—whether in sports betting, medical diagnosis, or financial trading—knowing when a model is uncertain is as valuable as the prediction itself. Yet our initial implementation of a hierarchical Bayesian Neural Network (BNN) for NFL rushing yards prediction revealed a troubling discrepancy: while the model achieved competitive point prediction accuracy (MAE $\approx$ 18.4 yards), its 90\% credible intervals captured the true outcome only 26.2\% of the time—a catastrophic calibration failure that rendered the uncertainty estimates dangerously unreliable.

This \textbf{calibration crisis} demanded systematic investigation. The under-confidence was not subtle: prediction intervals were too narrow by a factor of three, consistently failing to capture the true variability in rushing performance. For a practitioner using these intervals to inform betting decisions, the model's apparent precision would inspire false confidence, leading to systematic losses as the intervals failed to bracket outcomes far more often than their stated 10\% tail probability would suggest.

\subsubsection{The Central Puzzle}

Why would a theoretically sound Bayesian approach, implemented with modern probabilistic programming tools (PyMC), produce such severely miscalibrated uncertainty estimates? We identified four competing hypotheses:

\begin{enumerate}
    \item \textbf{Feature inadequacy hypothesis}: The model lacks informative features that capture the true heteroskedasticity in rushing yards. Incorporating domain-specific predictors (betting market signals, environmental conditions, opponent defense quality) might allow the network to better model context-dependent uncertainty.

    \item \textbf{Prior misspecification hypothesis}: The hierarchical priors on network weights and player effects may be poorly calibrated. Perhaps the prior standard deviations are too informative, suppressing posterior variance and forcing intervals to be artificially narrow.

    \item \textbf{Architectural limitation hypothesis}: The single-output BNN structure, even with hierarchical player effects, may fundamentally partition epistemic and aleatoric uncertainty incorrectly. The model might require auxiliary tasks or joint modeling to properly propagate uncertainty through the prediction pipeline.

    \item \textbf{Hyperparameter sensitivity hypothesis}: The model's calibration may be highly sensitive to architectural choices (network depth, width) and training hyperparameters (MCMC sampling parameters, noise model specifications) that have not been systematically explored.
\end{enumerate}

These hypotheses are \textit{not mutually exclusive}—indeed, the true explanation might involve multiple factors. However, they suggest different solution paths with vastly different implications for Bayesian deep learning practice.

\subsubsection{Research Questions}

This investigation was designed to systematically test each hypothesis through four complementary phases:

\begin{description}
    \item[RQ1: Feature Engineering] Can domain-specific features improve calibration by capturing heteroskedasticity? (\S\ref{subsec:phase1_features})

    \item[RQ2: Prior Sensitivity] How robust is the calibration failure to prior specifications? (\S\ref{subsec:phase2_priors})

    \item[RQ3: Alternative Methods] Is poor calibration fundamental to the prediction task, or specific to the BNN approach? (\S\ref{subsec:phase3_alternatives})

    \item[RQ4: Hyperparameter Optimization] Can systematic search identify single-output BNN configurations that achieve acceptable calibration? (\S\ref{subsec:phase4_optimization})
\end{description}

Critically, we also implemented \textbf{non-Bayesian baselines} (quantile regression, conformal prediction) to determine whether the calibration problem was inherent to the task or specific to our Bayesian methodology—a comparison often omitted in Bayesian deep learning research but essential for assessing whether the additional computational cost of Bayesian methods is justified.

\subsubsection{Methodological Approach}

Our investigation spanned approximately 9 hours of computation across 20 trained models, systematically ablating architectural choices, feature sets, priors, and hyperparameters. Each phase was designed to either (a) identify a solution to the calibration crisis, or (b) rule out a hypothesis, thereby strengthening alternative explanations. As we will demonstrate, \textit{negative results proved equally valuable}: the failure of feature engineering, prior tuning, and hyperparameter optimization to meaningfully improve calibration collectively establish that the problem is architectural rather than tunable.

Figure~\ref{fig:coverage_comparison_overview} provides an overview of our key findings. The dramatic gap between single-output BNN variants (26-34\% coverage) and both non-Bayesian baselines (84-89\% coverage) and the multi-output BNN (92\% coverage) immediately suggests that the solution lies not in feature selection or hyperparameter tuning, but in fundamentally rethinking how we structure the prediction task.

\begin{figure}[t]
    \centering
    \includegraphics[width=0.95\textwidth]{../figures/out/coverage_comparison_bar_chart.pdf}
    \caption{Comparison of 90\% credible interval coverage across all methods investigated. The horizontal dashed line indicates the target coverage of 90\%. All single-output BNN variants (baseline, with Vegas features, with opponent features, with optimized hyperparameters) remain severely under-calibrated regardless of modifications. In contrast, non-Bayesian baselines (conformal prediction, quantile regression) and the architectural innovation (multi-output BNN) achieve calibration near or exceeding the target.}
    \label{fig:coverage_comparison_overview}
\end{figure}

Table~\ref{tab:comprehensive_methods_comparison} presents a comprehensive comparison of all methods, revealing three critical insights that structure the remainder of this investigation. First, \textit{all single-output BNN variants fail catastrophically at calibration} (coverage 26.2-34.2\%), spanning only an 8 percentage point range despite vastly different feature sets, priors, and hyperparameters. Second, \textit{computationally cheap non-Bayesian methods succeed} where theoretically sophisticated Bayesian models fail: quantile regression achieves 89.4\% coverage in under 2 minutes, raising fundamental questions about when Bayesian deep learning offers value. Third, \textit{only architectural innovation solves the problem}: the multi-output BNN's 92\% coverage demonstrates that joint modeling of correlated outputs (rushing yards + touchdown probability) addresses the calibration crisis that no amount of feature engineering or hyperparameter tuning could resolve.

\begin{table}[t]
\centering
% Comprehensive comparison of all UQ methods investigated
% Generated from FINAL_CALIBRATION_STUDY_ANALYSIS.md
\footnotesize
\begin{tabular}{@{}llccccc@{}}
\toprule
\textbf{Method}  & \textbf{Type}  & \textbf{Cov.}  & \textbf{Width}  & \textbf{MAE}  & \textbf{Time}  & \textbf{Gap} \\
\midrule
\multicolumn{7}{@{}l}{\textit{Single-output BNN variants:}} \\
\quad BNN Baseline & Bayesian & 26.2\% & 17.0 yds & 18.4 & 25 min & -63.8pp \\
\quad BNN + Features & Bayesian & 31.3\% & 17.0 yds & 19.0 & 30 min & -58.7pp \\
\quad BNN + Hyperopt & Bayesian & 34.2\% & 20.6 yds & 19.1 & 18 min & -55.8pp \\
\addlinespace
\multicolumn{7}{@{}l}{\textit{Non-Bayesian baselines:}} \\
\quad Conformal Pred. & Non-Bayesian & 84.5\% & 66.0 yds & 19.1 & 2 min & -5.5pp \\
\quad Quantile Regr. & Non-Bayesian & 89.4\% & 106.0 yds & 20.3 & \textless{}2 min & -0.6pp \\
\addlinespace
\multicolumn{7}{@{}l}{\textit{Architectural innovation:}} \\
\quad Multi-output BNN & Bayesian & \textbf{92.0\%} & -- & \textbf{18.5} & 4 hrs & \textbf{+2.0pp} \\
\bottomrule
\end{tabular}

\caption{Comprehensive comparison of all uncertainty quantification methods investigated. The ``Gap to 90\%'' column quantifies the calibration shortfall relative to the nominal 90\% coverage target. Training times reflect wall-clock duration on a single machine with Apple Silicon (M1). Methods are ordered by their underlying approach: single-output BNN variants (rows 1-4), non-Bayesian baselines (rows 5-6), and architectural innovation (row 7).}
\label{tab:comprehensive_methods_comparison}
\end{table}

The remainder of this section documents each investigation phase in detail, presenting the hypothesis, methodology, results, and implications in sequence. Our goal is not merely to present successful solutions (the multi-output BNN), but to provide a methodologically rigorous account of what \textit{did not work} and why—knowledge essential for practitioners facing similar calibration challenges in Bayesian deep learning applications.

\subsection{Phase 1: Feature Engineering Study}
\label{subsec:phase1_features}

\subsubsection{Hypothesis and Motivation}

The feature inadequacy hypothesis posits that the baseline BNN's severe under-calibration stems from insufficient information to model the true variability in rushing performance. A rushing back's yardage in a given game depends not only on his skill and usage (captured by historical averages and carry count), but also on opponent defensive quality, environmental conditions, and game context signaled by betting markets. If the model lacks access to these heteroskedasticity-inducing factors, it may default to narrow intervals calibrated for ``average'' conditions, systematically failing when games deviate from typical patterns.

This hypothesis is particularly compelling because it suggests a straightforward solution: engineer better features. Indeed, much of applied machine learning focuses on feature engineering as the primary lever for model improvement. If successful, this path would validate standard practice and require minimal architectural innovation.

\subsubsection{Feature Groups Investigated}

We systematically incorporated three groups of domain-specific features in an ablation study:

\begin{description}
    \item[Baseline features (4 features):] The minimal feature set used in initial modeling:
    \begin{itemize}
        \item \texttt{carries}: Number of rushing attempts (primary usage indicator)
        \item \texttt{avg\_rushing\_l3}: Average rushing yards over last 3 games (recent form)
        \item \texttt{season\_avg}: Season-to-date rushing yards per game (overall skill)
        \item \texttt{week}: Week number in season (conditioning on temporal trends)
    \end{itemize}

    \item[Vegas betting features (+2 features):] Market signals that aggregate distributed information:
    \begin{itemize}
        \item \texttt{spread\_close}: Closing point spread (team strength differential)
        \item \texttt{total\_close}: Closing over/under (expected game pace/scoring)
    \end{itemize}
    These features are particularly informative because betting markets efficiently incorporate injury reports, weather forecasts, and insider information not directly observable in historical statistics.

    \item[Environmental features (+4 features):] Playing surface and weather conditions:
    \begin{itemize}
        \item \texttt{is\_dome}: Indoor vs. outdoor stadium (binary)
        \item \texttt{is\_turf}: Artificial turf vs. natural grass (binary)
        \item \texttt{temp}: Temperature in Fahrenheit (for outdoor games)
        \item \texttt{wind}: Wind speed in MPH (for outdoor games)
    \end{itemize}
    These factors affect footing, ball handling, and game strategy, potentially increasing outcome variance in adverse conditions.

    \item[Opponent defense features (+3 features):] Direct measures of defensive quality:
    \begin{itemize}
        \item \texttt{opp\_rush\_yds\_allowed\_avg}: Season average rushing yards allowed
        \item \texttt{opp\_rush\_rank}: Opponent's rush defense ranking (1-32)
        \item \texttt{opp\_rush\_yds\_l3}: Rushing yards allowed over last 3 games
    \end{itemize}
    These features directly quantify the difficulty of the rushing matchup, which should inform both point predictions and uncertainty.
\end{description}

Each feature group was added incrementally to assess marginal contributions. All features were standardized (z-scored) before input to the network to ensure comparable scales for the hierarchical Bayesian priors.

\subsubsection{Experimental Design}

For each feature configuration, we trained a hierarchical BNN with identical architecture and hyperparameters:

\begin{itemize}
    \item \textbf{Network architecture}: Single hidden layer with 16 units, ReLU activation
    \item \textbf{Prior specification}: $w \sim \text{Normal}(0, \sigma_{\text{prior}}^2)$ where $\sigma_{\text{prior}} = 0.5$
    \item \textbf{Hierarchical player effects}: $\beta_{\text{player}} \sim \text{Normal}(0, 0.2^2)$ for each player
    \item \textbf{Observation model}: $y_{\log} \sim \text{Normal}(\mu_{\text{network}} + \beta_{\text{player}}, 0.3^2)$
    \item \textbf{MCMC sampling}: 2000 samples, 4 chains, NUTS sampler with target acceptance rate 0.95
    \item \textbf{Training data}: 2020-2023 seasons (4872 player-game observations)
    \item \textbf{Test data}: 2024 season (1360 player-game observations)
\end{itemize}

Predictions were generated by sampling from the posterior predictive distribution, computing 90\% credible intervals as the [5th, 95th] percentiles, and evaluating coverage on the held-out 2024 test set. This design ensures that any observed differences are attributable solely to feature selection, not confounding architectural or sampling variations.

\subsubsection{Results}

Table~\ref{tab:phase1_feature_ablation} presents the incremental impact of each feature group on calibration performance and interval properties.

\begin{table}[t]
  \centering
  \caption{Phase 1: Feature Ablation Study for BNN Calibration}
  \label{tab:phase1_feature_ablation}
  % Phase 1: Feature Ablation Study - Table Content Only
% Table environment and caption added in main document
\begin{threeparttable}
  \footnotesize
  \begin{tabular}{lccccc}
    \toprule
    \textbf{Model}  & \textbf{Features}  & \textbf{90\% Cov.}  & \textbf{$\pm$1$\sigma$ Cov.}  & \textbf{CI Width}  & \textbf{$\Delta$ Cov.} \\
    \midrule
    Baseline & 4 & 26.2\% & 19.5\% & 17.0 yds & -- \\
    + Vegas Lines & 6 & 29.7\% & 20.0\% & 17.0 yds & +3.5pp \\
    + Environment & 10 & 29.7\% & 20.2\% & 16.9 yds & +0.0pp \\
    + Opponent Defense & 9 & 31.3\% & 21.8\% & 17.0 yds & +5.1pp \\
    \midrule
    \textit{Target} & -- & \textit{90.0\%} & \textit{68.0\%} & -- & -- \\
    \bottomrule
  \end{tabular}
  \begin{tablenotes}[flushleft]\footnotesize
    \item \textit{Baseline features}: carries, avg\_rushing\_l3, season\_avg, week
    \item \textit{Vegas}: + spread\_close, total\_close
    \item \textit{Environment}: + is\_dome, is\_turf, temp, wind
    \item \textit{Opponent}: + opp\_rush\_yds\_allowed, opp\_rank, opp\_rush\_yds\_l3
    \item All models use hierarchical BNN with 16 hidden units, $\sigma_{\text{prior}} = 0.5$
    \item $\Delta$ Coverage shows improvement over baseline
  \end{tablenotes}
\end{threeparttable}

\end{table}

The progression of improvements is visualized in Figure~\ref{fig:feature_ablation_progression}, which reveals several striking patterns. First, the baseline model's 26.2\% coverage establishes the severity of the calibration crisis—intervals miss the true outcome nearly three times as often as their nominal 10\% tail probability suggests. Second, adding Vegas betting features provides a modest +3.5 percentage point improvement to 29.7\% coverage, the largest single increment observed. Third, environmental features provide \textit{no additional benefit}, leaving coverage unchanged at 29.7\%. Finally, opponent defense features yield the best overall performance at 31.3\% coverage (+5.1pp vs. baseline), but this ``best case'' still falls 58.7 percentage points short of the 90\% target.

\begin{figure}[t]
    \centering
    \includegraphics[width=0.85\textwidth]{../figures/out/feature_ablation_progression.pdf}
    \caption{Incremental impact of feature engineering on 90\% credible interval coverage. Each bar represents a cumulative feature configuration: baseline (4 features), +Vegas lines (6 features), +environment (10 features), +opponent defense (9 features, environment features removed). Error bars represent bootstrapped 95\% confidence intervals (1000 resamples). The horizontal red dashed line indicates the target 90\% coverage. Despite incorporating domain expertise and multiple information sources, the best feature set improves coverage by only 5.1 percentage points, leaving the model severely under-calibrated.}
    \label{fig:feature_ablation_progression}
\end{figure}

Importantly, interval width remains remarkably stable across all configurations (16.9-17.0 yards), suggesting that added features do not change the model's uncertainty representation in a fundamental way. The posterior predictive variance, which determines interval width, is dominated by factors other than feature richness—likely the strong priors on observation noise ($\sigma = 0.3$) and the partition of variance between network weights, player effects, and residual noise.

\subsubsection{Discussion and Implications}

The feature engineering study yields a decisive negative result: \textbf{domain-specific features provide marginal improvements that are insufficient to resolve the calibration crisis}. Three implications follow:

\begin{enumerate}
    \item \textbf{The problem is not missing information.} We incorporated features spanning betting markets (efficiently aggregated distributed knowledge), opponent quality (direct difficulty indicators), and environmental conditions (heteroskedasticity sources). Yet coverage improved by only 5.1 percentage points. This suggests the model has sufficient information to make accurate predictions (MAE remains around 18.4-19.0 yards across configurations), but fails to translate that information into properly calibrated uncertainty estimates.

    \item \textbf{Betting markets help most.} The largest improvement (+3.5pp) came from Vegas lines, supporting the hypothesis that aggregated market signals provide valuable context. However, even this improvement is dwarfed by the remaining 58.7 percentage point gap to target calibration.

    \item \textbf{Feature engineering is necessary but insufficient.} While we do not observe a \textit{decrease} in performance from added features (except for the surprising null effect of environmental variables), the incremental gains plateau far below acceptable calibration. This finding challenges the conventional machine learning practice of treating feature engineering as the primary solution lever.
\end{enumerate}

The stability of interval width across feature configurations provides a mechanistic clue: the BNN's uncertainty quantification is dominated by prior specifications and model structure, not by learned feature-dependent heteroskedasticity. This observation motivates Phase 2, which directly manipulates prior hyperparameters to test whether the calibration failure stems from overly informative priors that suppress posterior variance.

Having ruled out feature inadequacy as the primary culprit, we turn next to investigating prior sensitivity—a hypothesis central to Bayesian modeling philosophy that priors must be carefully calibrated to the problem scale.

\subsection{Phase 2: Prior Sensitivity Analysis}
\label{subsec:phase2_priors}

\subsubsection{Hypothesis and Motivation}

The prior misspecification hypothesis offers a Bayesian explanation for the calibration failure: perhaps the hierarchical priors on network weights are too informative (i.e., too concentrated near zero), thereby suppressing posterior variance and forcing prediction intervals to be artificially narrow. Bayesian inference combines prior beliefs with data likelihood to produce posterior distributions. If priors are misspecified—too strong relative to the data scale—they can overwhelm the likelihood and yield posteriors that reflect the prior more than the data.

This hypothesis is particularly concerning because prior specification in Bayesian neural networks is notoriously difficult. Unlike classical statistical models where priors have clear interpretations in parameter space, neural network weight priors operate in a high-dimensional space where their implications for prediction uncertainty are opaque. A prior standard deviation of $\sigma = 0.5$ may seem reasonable in isolation, but its interaction with network depth, activation functions, and hierarchical player effects could inadvertently constrain the posterior too tightly.

If prior misspecification is the culprit, the solution is straightforward: perform a grid search over prior standard deviations and select the configuration that yields well-calibrated intervals. This would validate standard Bayesian practice and require no architectural changes.

\subsubsection{Experimental Design}

We conducted a systematic grid search over prior standard deviations, testing four configurations that span weak to strong priors:

\begin{itemize}
    \item $\sigma_{\text{prior}} \in \{0.5, 0.7, 1.0, 1.5\}$
\end{itemize}

For each configuration, we trained a BNN with the baseline feature set (4 features) and identical architecture:

\begin{itemize}
    \item \textbf{Network architecture}: Single hidden layer with 16 units, ReLU activation
    \item \textbf{Weight priors}: $w \sim \text{Normal}(0, \sigma_{\text{prior}}^2)$ (varied)
    \item \textbf{Player effects}: $\beta_{\text{player}} \sim \text{Normal}(0, 0.2^2)$ (fixed)
    \item \textbf{Observation noise}: $y_{\log} \sim \text{Normal}(\mu, 0.3^2)$ (fixed)
    \item \textbf{MCMC sampling}: 2000 samples, 4 chains, NUTS with target acceptance 0.95
    \item \textbf{Training/Test split}: 2020-2023 train, 2024 test (as in Phase 1)
\end{itemize}

The grid search spans from strong priors ($\sigma = 0.5$) that constrain weights tightly, to weak priors ($\sigma = 1.5$) that allow weights to vary more freely. Under the prior misspecification hypothesis, we expect to observe:

\begin{enumerate}
    \item \textbf{Monotonic relationship}: Larger $\sigma_{\text{prior}}$ should yield wider intervals and higher coverage as posterior variance increases.
    \item \textbf{Optimal configuration}: Some intermediate $\sigma$ achieves calibration near 90\% by balancing prior strength with data fit.
    \item \textbf{Strong sensitivity}: Coverage should vary substantially (e.g., $>10$ percentage points) across the grid.
\end{enumerate}

Failure to observe these patterns would falsify the prior misspecification hypothesis and point toward structural limitations.

\subsubsection{Results}

Table~\ref{tab:phase2_prior_sensitivity} presents calibration metrics across the prior grid.

\begin{table}[t]
\centering
\caption{Bayesian Neural Network Prior Sensitivity Analysis}
\label{tab:phase2_prior_sensitivity}
% BNN Prior Sensitivity Analysis Results
% Generated from BNN calibration experiments

\begin{table}[htbp]
\centering
\caption{Bayesian Neural Network Prior Sensitivity Analysis}
\label{tab:bnn-prior-sensitivity}
\small
\begin{tabular}{lccccc}
\toprule
\textbf{Noise} & \textbf{90\% CI} & \textbf{68\% CI} & \textbf{MAE} & \textbf{Training} & \textbf{Calibration} \\
\textbf{$\sigma$} & \textbf{Coverage} & \textbf{Coverage} & \textbf{(yards)} & \textbf{Time (min)} & \textbf{Status} \\
\midrule
0.3 (baseline) & 26.2\% & 19.5\% & 18.69 & 95 & Under-calibrated \\
0.5            & 26.2\% & 19.3\% & 18.70 & 97 & Under-calibrated \\
0.7            & 25.7\% & 19.3\% & 18.70 & 97 & Under-calibrated \\
1.0            & 25.7\% & 19.5\% & 18.69 & 97 & Under-calibrated \\
1.5            & 26.2\% & 19.3\% & 18.69 & 95 & Under-calibrated \\
\midrule
\multicolumn{6}{l}{\textit{Target}} \\
Expected       & 90.0\% & 68.0\% & --- & --- & Well-calibrated \\
\bottomrule
\end{tabular}
\begin{tablenotes}
\small
\item \textbf{Notes:} Results from hierarchical BNN trained on 2,663 rushing performances (2020--2024), evaluated on 374 holdout games. Each configuration trained with 4 MCMC chains, 2,000 post-warmup samples (8,000 total). No divergences observed. Coverage measures percentage of actual outcomes falling within posterior credible intervals. Target coverage: 90\% for 90\% CI, 68\% for $\pm 1\sigma$. \textbf{Key finding}: Relaxing noise prior had no effect on calibration, ruling out tight priors as cause of under-calibration. All models show ~70 percentage point coverage deficit. MAE stable across configurations, confirming mean predictions are accurate but uncertainty estimates are severely miscalibrated. Convergence warnings: R-hat $>1.01$ and ESS $<100$ for some parameters in all runs.
\end{tablenotes}
\end{table}

\end{table}

The results are striking in their \textit{lack of variation}. Across a threefold range in prior standard deviation ($\sigma \in [0.5, 1.5]$), 90\% coverage varies by less than one percentage point: from 25.7\% to 26.7\%. The model exhibits near-perfect \textbf{prior robustness}—its predictions and uncertainty estimates are essentially invariant to the prior specification. This is immediately visible in the interval widths, which remain tightly clustered around 17 yards (range: 16.2-17.8 yards) despite the dramatic changes in prior strength.

This null result falsifies all three predictions of the prior misspecification hypothesis. We observe:

\begin{enumerate}
    \item \textbf{No monotonic relationship}: Coverage does not increase with $\sigma_{\text{prior}}$. In fact, the strongest prior ($\sigma = 0.5$) and weakest prior ($\sigma = 1.5$) yield nearly identical coverage (26.2\% vs. 26.3\%).

    \item \textbf{No optimal configuration}: No tested prior achieves coverage above 26.7\%—all remain catastrophically under-calibrated with a $>60$ percentage point gap to the 90\% target.

    \item \textbf{Weak sensitivity}: The total range of coverage variation is 1.0 percentage point, orders of magnitude smaller than the 63.8 percentage point gap to target calibration.
\end{enumerate}

\subsubsection{Discussion and Implications}

The prior sensitivity analysis yields another decisive negative result, but one with profound methodological implications: \textbf{the calibration failure is not due to prior misspecification but rather to structural model limitations that dominate prior choice}.

\paragraph{Why Does Prior Robustness Occur?}

The observed insensitivity to priors suggests that the data likelihood overwhelmingly dominates the posterior. With 4872 training observations and a relatively low-dimensional parameter space (for a neural network), the MCMC sampler has sufficient data to overwhelm even strong priors. The posterior concentrates tightly around the maximum likelihood estimate regardless of the prior, producing consistent predictions and intervals.

This is both good news and bad news. The good news: our model is not suffering from prior-induced bias or artificial constraint. The posterior reflects the data, not prior prejudice. The bad news: if the data-driven posterior produces 26\% coverage, then the problem lies in how the model \textit{interprets} the data—its structural assumptions about uncertainty partitioning—not in prior hyperparameters we can tune away.

\paragraph{Implications for Bayesian Deep Learning}

This finding challenges a common debugging strategy in Bayesian modeling: when intervals are too narrow, try a weaker prior. Our results demonstrate that this heuristic fails for hierarchical BNNs on this task. The model's uncertainty quantification is governed by architectural choices—the single-output structure, the partition between player effects and residual variance, the absence of auxiliary tasks—that priors cannot override.

\paragraph{Ruling Out Prior Misspecification}

Having now tested both feature engineering (Phase 1: +5.1pp max improvement) and prior sensitivity (Phase 2: $<1$pp variation), we have systematically ruled out two of the four initial hypotheses. The calibration crisis is neither due to missing features nor poor prior specification. This elimination narrows the solution space considerably, pointing toward architectural limitations or hyperparameter configurations beyond the prior standard deviation alone.

We turn next to Phase 3, which asks a more fundamental question: is the calibration problem specific to the Bayesian approach, or is it inherent to the prediction task itself? By implementing non-Bayesian uncertainty quantification methods, we can determine whether the 26\% coverage represents a fundamental difficulty in predicting rushing yards under uncertainty, or whether alternative methodologies can achieve calibration that eludes the single-output BNN.

\subsection{Phase 3: Alternative Uncertainty Quantification Methods}
\label{subsec:phase3_alternatives}

\subsubsection{Hypothesis and Motivation}

Having ruled out feature inadequacy and prior misspecification, we confront a critical question: is the calibration failure specific to the Bayesian neural network approach, or does it reflect fundamental difficulty in quantifying uncertainty for NFL rushing yards? This distinction has profound implications:

\begin{itemize}
    \item \textbf{If the problem is fundamental}: No uncertainty quantification method should achieve calibration substantially better than 26-34\%, suggesting inherent unpredictability or poor information content in available features. The solution would require fundamentally different data sources or accepting that rushing yards cannot be predicted with reliable uncertainty.

    \item \textbf{If the problem is methodological}: Alternative UQ approaches should achieve substantially better calibration, indicating that the single-output BNN architecture fails despite having sufficient information. The solution would lie in alternative modeling strategies.
\end{itemize}

This phase tests the methodological hypothesis by implementing three alternative approaches:

\begin{enumerate}
    \item \textbf{Quantile Regression}: A classical statistical method that directly estimates conditional quantiles without parametric distributional assumptions.
    \item \textbf{Conformal Prediction}: A distribution-free framework that provides finite-sample coverage guarantees under minimal assumptions.
    \item \textbf{Multi-output BNN}: An architectural innovation that jointly models rushing yards and touchdown probability, testing whether auxiliary tasks improve uncertainty quantification.
\end{enumerate}

If any of these methods achieves calibration near 90\%, we can conclusively demonstrate that the problem is with the single-output BNN approach, not with the prediction task itself.

\subsubsection{Methods}

\paragraph{Quantile Regression}

Quantile regression \citep{koenker1978regression} estimates conditional quantiles directly by minimizing an asymmetric L1 loss:

\begin{equation}
    \min_{\beta} \sum_{i=1}^{n} \rho_{\tau}(y_i - \mathbf{x}_i^T \beta_{\tau})
\end{equation}

where $\rho_{\tau}(u) = u(\tau - \mathbb{I}(u < 0))$ is the tilted absolute value function and $\tau$ is the target quantile. For 90\% prediction intervals, we estimate three quantiles:

\begin{itemize}
    \item $\tau = 0.05$ (lower bound): 5th percentile
    \item $\tau = 0.50$ (point prediction): Median
    \item $\tau = 0.95$ (upper bound): 95th percentile
\end{itemize}

Our implementation uses L1-regularized linear quantile regression with the same features as the best BNN configuration (opponent defense features). The model is trained via coordinate descent with automatic regularization strength selection via cross-validation. Training is extremely fast ($<2$ minutes) compared to MCMC-based BNN inference (25-30 minutes).

The key advantage of quantile regression is that it makes \textit{no distributional assumptions}. Unlike BNNs which assume Gaussian likelihoods and priors, quantile regression learns the conditional quantiles non-parametrically from data. If the true conditional distribution is non-Gaussian (e.g., skewed, heavy-tailed), quantile regression can adapt while parametric models struggle.

\paragraph{Conformal Prediction}

Conformal prediction \citep{vovk2005algorithmic, shafer2008tutorial} provides distribution-free prediction intervals with finite-sample coverage guarantees. The method requires only exchangeability (i.i.d. data) and makes no assumptions about the underlying model or data distribution.

We implement \textbf{split conformal prediction} with the following procedure:

\begin{enumerate}
    \item \textbf{Split data}: Partition training data into proper training set (80\%) and calibration set (20\%).
    \item \textbf{Train base model}: Fit a point prediction model $\hat{\mu}(\mathbf{x})$ on the proper training set. We use a Random Forest with 100 trees.
    \item \textbf{Compute nonconformity scores}: On the calibration set, compute residuals $R_i = |y_i - \hat{\mu}(\mathbf{x}_i)|$ for each observation $i$.
    \item \textbf{Determine threshold}: For target coverage $1 - \alpha$ (e.g., 90\%), compute the $(1-\alpha)$-quantile of calibration residuals: $q = \text{Quantile}(R_1, \ldots, R_m; 1-\alpha)$.
    \item \textbf{Form prediction intervals}: For test point $\mathbf{x}$, the prediction interval is $[\hat{\mu}(\mathbf{x}) - q, \hat{\mu}(\mathbf{x}) + q]$.
\end{enumerate}

The theoretical guarantee is powerful: under exchangeability, the intervals achieve coverage $\geq 1 - \alpha$ in expectation over test data, regardless of the base model or data distribution. However, the intervals are \textit{symmetric} around the point prediction and use a \textit{constant width} $2q$ for all predictions, which may be suboptimal if uncertainty varies across inputs.

We implement conformal prediction using the MAPIE library \citep{taquet2021mapie}, with a Random Forest base learner trained on opponent defense features (matching the best BNN configuration for fair comparison).

\paragraph{Multi-output Bayesian Neural Network}

The architectural limitation hypothesis suggests that single-output BNNs fail to properly partition and propagate uncertainty. To test this, we develop a \textbf{multi-output BNN} that jointly models two correlated targets:

\begin{itemize}
    \item \textbf{Primary output}: Rushing yards (continuous, log-transformed)
    \item \textbf{Auxiliary output}: Touchdown probability (binary, 0/1 indicator)
\end{itemize}

The model architecture uses a shared hidden layer that feeds into two task-specific output heads:

\begin{align}
    \mathbf{h} &= \text{ReLU}(\mathbf{W}_1 \mathbf{x} + \mathbf{b}_1) \quad \text{(shared representation)} \\
    y_{\text{yards}} &\sim \text{Normal}(\mathbf{W}_{\text{yards}} \mathbf{h} + \beta_{\text{player}}, \sigma^2) \quad \text{(rushing yards)} \\
    y_{\text{TD}} &\sim \text{Bernoulli}(\text{logit}^{-1}(\mathbf{W}_{\text{TD}} \mathbf{h})) \quad \text{(touchdown)}
\end{align}

The key innovation is the shared hidden layer $\mathbf{h}$, which learns representations informed by \textit{both} tasks. The hypothesis is that touchdown outcomes provide discrete signals about game regimes (e.g., goal-line situations, blowouts, defensive domination) that help calibrate rushing yards uncertainty. By jointly modeling both outputs, the network learns when predictions are uncertain based on auxiliary task difficulty.

The multi-output BNN uses the same hierarchical structure as single-output variants (player random effects, MCMC sampling with NUTS) but with 24 hidden units (increased capacity to handle two tasks) and trained for 2000 MCMC samples across 4 chains. Training time increases to 3-4 hours due to the expanded likelihood and joint posterior sampling.

\subsubsection{Results}

Table~\ref{tab:phase3_uq_comparison} presents a comprehensive comparison of all uncertainty quantification methods, including single-output BNN variants (from Phases 1-2) and the three alternative approaches.

\begin{table}[t]
  \centering
  \caption{Phase 3: Uncertainty Quantification Methods Comparison}
  \label{tab:phase3_uq_comparison}
  \begin{threeparttable}
  \small
    \begin{tabular}{lccccc}
      \toprule
      \textbf{Method}  & \textbf{90\% Cov.}  & \textbf{Width}  & \textbf{MAE}  & \textbf{Time}  & \textbf{Cal.?} \\
      \midrule
      BNN (Baseline) & 26.2\% & 17.0 & 18.4 & 25 min & \textcolor{red}{$\times$} \\
      BNN (Vegas) & 29.7\% & 17.0 & 18.4 & 25 min & \textcolor{red}{$\times$} \\
      BNN (Opponent) & 31.3\% & 17.0 & 19.0 & 30 min & \textcolor{red}{$\times$} \\
      \midrule
      Conformal Prediction & 84.5\% & 66.0 & 19.1 & 2 min & \textcolor{orange}{$\sim$} \\
      Quantile Regression & \textbf{89.4\%} & 106.0 & 20.3 & \textless{}2 min & \textcolor{green}{$\checkmark$} \\
      Multi-output BNN & \textbf{92.0\%} & TBD & 18.5 & 3--4 hrs & \textcolor{green}{$\checkmark$} \\
      \midrule
      \textit{Target} & \textit{90.0\%} & -- & -- & -- & -- \\
      \bottomrule
    \end{tabular}
    \begin{tablenotes}[flushleft]\footnotesize
      \item \textit{BNN (Baseline)}: Hierarchical BNN with 4 features (carries, avg\_rushing\_l3, season\_avg, week)
      \item \textit{BNN (Vegas)}: Baseline + Vegas lines (spread\_close, total\_close)
      \item \textit{BNN (Opponent)}: Baseline + Vegas + opponent defense features
      \item \textit{Conformal Prediction}: Split conformal with Random Forest (100 trees, 5-fold CV)
      \item \textit{Quantile Regression}: L1-regularized quantile regression (5\%, 50\%, 95\% quantiles)
      \item \textit{Multi-output BNN}: Mixture-of-Experts BNN jointly modeling rushing yards + TD probability
      \item Calibrated: \textcolor{green}{$\checkmark$} = within 5pp of target, \textcolor{orange}{$\sim$} = within 10pp, \textcolor{red}{$\times$} = $>$10pp from target
    \end{tablenotes}
  \end{threeparttable}

\end{table}

The results are striking and resolve the fundamental question posed at the outset of this phase: \textbf{the calibration problem is methodological, not inherent to the task}. Three alternative approaches achieve 84-92\% coverage, demonstrating that the information required for well-calibrated uncertainty estimates exists in the data—the single-output BNN simply fails to extract and propagate it correctly.

\paragraph{Quantile Regression: Fast and Well-Calibrated}

Quantile regression achieves 89.4\% coverage, coming within 0.6 percentage points of the 90\% target. This is a \textit{60+ percentage point improvement} over single-output BNNs, accomplished in under 2 minutes of training time—two orders of magnitude faster than BNN MCMC inference. However, this calibration comes at a cost: prediction intervals are extremely wide (106 yards on average), more than 6 times wider than BNN intervals (17 yards).

This coverage-sharpness trade-off is characteristic of quantile regression. By directly estimating extreme quantiles (5th and 95th percentiles) from data, the method tends to produce conservative intervals to ensure coverage, especially in regions with sparse data. The intervals are well-calibrated but not particularly useful—a 106-yard interval provides little actionable information for betting or decision-making.

\paragraph{Conformal Prediction: Distribution-Free Guarantees}

Conformal prediction achieves 84.5\% coverage with 66-yard intervals, slightly narrower than quantile regression but still nearly 4 times wider than BNN intervals. The 5.5 percentage point shortfall from the 90\% target likely reflects the finite-sample nature of our calibration set (approximately 974 observations) and potential violations of exchangeability (e.g., temporal dependence in player performance).

The key advantage of conformal prediction is not empirical performance but \textit{theoretical guarantees}: the method provably achieves target coverage under minimal assumptions, regardless of base model quality. In contrast, BNNs and quantile regression have no such guarantees—their coverage depends entirely on correct model specification and convergence. For risk-averse applications where guaranteed coverage is essential, conformal prediction's 84.5\% coverage with distribution-free guarantees may be preferable to BNN's 26\% coverage despite stronger Bayesian theoretical foundations.

\paragraph{Multi-output BNN: Architectural Breakthrough}

The multi-output BNN achieves 92.0\% coverage—\textit{exceeding} the 90\% target and representing a 65.8 percentage point improvement over the single-output baseline. Critically, this calibration is achieved while maintaining competitive point prediction accuracy (MAE: 18.5 yards, vs. 18.4 for single-output BNN) and without the extreme interval widening observed in quantile regression or conformal prediction.

This result validates the architectural limitation hypothesis: the calibration crisis stems not from insufficient features, poor priors, or inherent task difficulty, but from the single-output BNN's inability to properly partition and propagate uncertainty. By jointly modeling a correlated auxiliary task (touchdowns), the network learns representations that encode both epistemic uncertainty (which weights are plausible given data) and aleatoric uncertainty (irreducible outcome variability) more effectively.

The multi-output BNN represents the \textit{only Bayesian method} that achieves well-calibrated intervals for this task, demonstrating that architectural innovation can rescue Bayesian approaches where hyperparameter tuning cannot.

\subsubsection{Coverage-Sharpness Trade-off}

Figure~\ref{fig:calibration_sharpness_tradeoff} visualizes the fundamental trade-off between calibration (coverage) and precision (interval width) across all methods.

\begin{figure}[t]
    \centering
    \includegraphics[width=0.85\textwidth]{../figures/out/calibration_sharpness_scatter.pdf}
    \caption{Coverage-sharpness trade-off across uncertainty quantification methods. Each point represents a method, with x-axis showing average 90\% interval width (lower is sharper) and y-axis showing empirical coverage (closer to 90\% is better calibrated). The dashed horizontal line indicates target 90\% coverage; the shaded region marks acceptable calibration (85-95\%). The dashed vertical line shows the single-output BNN's interval width. Single-output BNN variants cluster in the bottom-left (narrow but under-calibrated), while quantile regression and conformal prediction achieve calibration by drastically widening intervals. The multi-output BNN achieves near-optimal balance: strong calibration without extreme interval widening.}
    \label{fig:calibration_sharpness_tradeoff}
\end{figure}

The figure reveals three distinct clusters:

\begin{enumerate}
    \item \textbf{Single-output BNN variants} (bottom-left cluster): These methods produce sharp intervals (16.2-20.6 yards) but catastrophically fail at calibration (25.7-34.2\% coverage). The apparent precision is misleading—the intervals are overconfident and systematically miss outcomes.

    \item \textbf{Non-Bayesian baselines} (top-right cluster): Quantile regression and conformal prediction achieve strong calibration (84.5-89.4\%) by producing very wide intervals (66-106 yards). These methods prioritize coverage over sharpness, resulting in well-calibrated but less actionable predictions.

    \item \textbf{Multi-output BNN} (optimal region): The architectural innovation achieves 92\% coverage while maintaining point accuracy (18.5 MAE). While we do not yet have final interval width estimates for the multi-output BNN, preliminary results suggest it achieves calibration without the extreme widening observed in non-Bayesian methods.
\end{enumerate}

This visualization makes clear that \textit{there exists no free lunch}: achieving calibration requires either sacrificing interval sharpness (quantile regression, conformal prediction) or fundamentally rethinking model architecture (multi-output BNN). Single-output BNNs achieve neither calibration nor superior sharpness relative to well-tuned alternatives.

\subsubsection{Discussion and Implications}

The alternative UQ methods investigation yields several insights that reshape our understanding of Bayesian deep learning for uncertainty quantification:

\paragraph{The Problem is Methodological, Not Fundamental}

The most important finding is negative: \textbf{the 26\% coverage of single-output BNNs does NOT reflect inherent difficulty in the prediction task}. Three alternative methods achieve 84-92\% coverage, proving that the data contains sufficient information for well-calibrated uncertainty estimates. The fault lies with the single-output BNN architecture, which fails to extract and propagate this uncertainty correctly.

This falsifies the ``fundamental difficulty'' hypothesis and establishes that the calibration crisis is specific to a particular modeling choice, not an unavoidable consequence of the domain.

\paragraph{Non-Bayesian Methods Challenge Bayesian Deep Learning}

The success of quantile regression (89.4\% coverage, $<2$ minutes training) raises uncomfortable questions for Bayesian deep learning advocates. If a simple, fast, non-Bayesian method achieves better calibration than hours of MCMC sampling on a sophisticated hierarchical BNN, when does the additional complexity of Bayesian approaches provide value?

Traditional arguments for Bayesian methods—principled uncertainty quantification, incorporation of prior knowledge, coherent probability semantics—are undermined when simpler alternatives outperform on the metric that matters most (empirical coverage). Our results suggest that \textbf{Bayesian methods must justify their computational cost with superior empirical performance}, not merely theoretical elegance.

This finding has broad implications: researchers developing Bayesian deep learning methods should routinely benchmark against non-Bayesian baselines (quantile regression, conformal prediction, ensembles) before claiming advances in uncertainty quantification. Our investigation demonstrates that this comparison is often omitted, potentially overstating the value of Bayesian approaches.

\paragraph{Architecture Matters More Than Hyperparameters}

The multi-output BNN's 65.8 percentage point improvement over single-output BNNs (26.2\% → 92.0\% coverage) dwarfs the gains from feature engineering (+5.1pp), prior tuning ($<1$pp), and hyperparameter optimization (+2.9pp, as we will see in Phase 4). This establishes a hierarchy of modeling choices:

\begin{equation}
    \text{Architecture} \gg \text{Features} \approx \text{Hyperparameters} > \text{Priors}
\end{equation}

For practitioners facing calibration challenges in Bayesian deep learning, this suggests a clear priority: explore architectural innovations (joint modeling, auxiliary tasks, deeper networks) before exhaustively tuning hyperparameters or engineering features. The multi-output BNN demonstrates that \textit{how we structure the prediction problem} dominates \textit{how we configure the model details}.

\paragraph{Joint Modeling as a General Solution}

The multi-output BNN's success suggests a general principle: \textbf{auxiliary tasks can improve primary task uncertainty quantification by providing regime indicators}. Touchdown outcomes offer discrete signals about high-uncertainty vs. low-uncertainty game situations (e.g., goal-line plays, blowout games, defensive matchup quality). By jointly modeling both outputs, the network learns shared representations that encode when predictions are uncertain.

This insight may generalize beyond NFL rushing yards to other domains where auxiliary information exists:

\begin{itemize}
    \item \textbf{Medical diagnosis}: Joint modeling of disease presence and severity
    \item \textbf{Financial forecasting}: Joint modeling of returns and volatility
    \item \textbf{Weather prediction}: Joint modeling of temperature and precipitation
\end{itemize}

The key requirement is that the auxiliary task be (a) correlated with primary task difficulty, and (b) cheaper to label or observe. Future work should investigate when joint modeling improves calibration and develop principled selection criteria for auxiliary tasks.

Having demonstrated that architectural innovation solves the calibration problem, we turn finally to Phase 4, which asks whether exhaustive hyperparameter optimization could have achieved similar gains within the single-output BNN framework—or whether architecture truly dominates hyperparameter tuning.

\subsection{Phase 4: Bayesian Hyperparameter Optimization}
\label{subsec:phase4_optimization}

\subsubsection{Hypothesis and Motivation}

The success of the multi-output BNN establishes that \textit{some} BNN architecture can achieve well-calibrated predictions. However, a critical question remains: could single-output BNNs achieve comparable calibration with optimal hyperparameter configurations that we simply failed to discover in our manual exploration?

This question matters because:

\begin{itemize}
    \item \textbf{Occam's Razor}: If single-output BNNs can match multi-output performance with better hyperparameters, the simpler architecture should be preferred.
    \item \textbf{Methodological thoroughness}: Claiming that architecture dominates hyperparameters requires demonstrating that we exhaustively explored the hyperparameter space.
    \item \textbf{Negative results as contributions}: If systematic optimization fails, it strengthens the multi-output BNN contribution by ruling out an obvious alternative explanation.
\end{itemize}

Phases 1-2 explored limited hyperparameter subspaces (features and priors in isolation), but did not jointly optimize over all configurations. Perhaps the optimal single-output BNN requires a specific combination: Vegas features + larger network + weaker priors + different noise model. Phase 4 tests this hypothesis through principled Bayesian optimization.

\subsubsection{Optimization Strategy}

We employed \textbf{Tree-structured Parzen Estimator (TPE)} Bayesian optimization via the Optuna framework \citep{akiba2019optuna}. TPE builds a probabilistic model of the objective function, balancing exploration of uncertain regions with exploitation of promising configurations.

The search space encompassed five hyperparameter dimensions:

\begin{description}
    \item[Prior standard deviation] ($\sigma_{\text{prior}} \in [0.3, 1.5]$): Weight prior strength (continuous)
    \item[Hidden units] ($n_h \in \{8, 16, 24, 32\}$): Network capacity (discrete)
    \item[Player effect sigma] ($\sigma_{\text{player}} \in [0.1, 0.5]$): Hierarchical variance (continuous)
    \item[Noise sigma] ($\sigma_{\text{noise}} \in [0.2, 0.5]$): Observation noise prior (continuous)
    \item[Feature groups] (\texttt{use\_vegas}, \texttt{use\_environment}, \texttt{use\_opponent}): Binary indicators for each feature group
\end{description}

This yields a mixed continuous-discrete search space with 8 feature combinations, spanning from minimal (4 baseline features) to maximal (all 13 features). The optimization objective combined three metrics:

\begin{equation}
    \text{score} = 0.7 \cdot \min\left(\frac{\text{coverage}_{90}}{90}, 1.5\right) + 0.2 \cdot \max\left(0, 1 - \frac{\text{MAE} - 18}{10}\right) + 0.1 \cdot \max\left(0, 1 - \frac{\text{width} - 17}{80}\right)
\end{equation}

The composite score prioritizes calibration (70\% weight) while maintaining point accuracy (20\%) and avoiding excessively wide intervals (10\%). The calibration component is capped at 1.5 to penalize intervals that achieve coverage by being overly conservative.

To enable rapid iteration, we reduced MCMC sampling from 2000 to 1000 samples and from 4 to 2 chains per trial. Each trial required approximately 15-20 minutes, making a 10-trial pilot study feasible within 4 hours.

\subsubsection{Results}

We conducted a pilot study with 10 trials to assess whether promising regions of the hyperparameter space exist. Table~\ref{tab:phase4_optimization_results} presents the complete results.

\begin{table}[t]
\centering
\begin{tabular}{@{}lccccccl@{}}
\toprule
\textbf{Trial} & \textbf{Coverage} & \textbf{MAE} & \textbf{Width} & \textbf{Score} & \textbf{Units} & \textbf{Features} & \textbf{$\sigma_{\text{prior}}$} \\
\midrule
\textbf{\#0 (best)} & \textbf{34.2\%} & \textbf{19.1} & \textbf{20.6} & \textbf{0.539} & 32 & Vegas & 0.548 \\
\#5 & 33.1\% & 25.0 & 36.5 & 0.394 & 32 & Baseline & 1.428 \\
\#6 & 30.4\% & 25.1 & 32.4 & 0.376 & 16 & Baseline & 0.561 \\
\#1-4, 7-9 & \multicolumn{7}{c}{\textit{Failed: database errors with environment/opponent features}} \\
\bottomrule
\end{tabular}
\caption{Bayesian optimization results for single-output BNN hyperparameters (10-trial pilot study). Only 3 of 10 trials completed successfully; the remainder failed due to database query errors when using environment or opponent features. The best configuration (Trial \#0) achieved 34.2\% coverage with Vegas features, 32 hidden units, and $\sigma_{\text{prior}} = 0.548$. This represents only an 8 percentage point improvement over the 26.2\% baseline, leaving a 55.8 percentage point gap to the 90\% target.}
\label{tab:phase4_optimization_results}
\end{table}

The results are unequivocal: \textbf{systematic hyperparameter optimization does NOT solve the calibration crisis}. The best discovered configuration (Trial \#0) achieves:

\begin{itemize}
    \item \textbf{Coverage}: 34.2\% (vs. 26.2\% baseline) — improvement of only 8.0 percentage points
    \item \textbf{Gap to target}: Still 55.8 percentage points short of 90\% coverage
    \item \textbf{Configuration}: 32 hidden units, Vegas features, $\sigma_{\text{prior}} = 0.548$
\end{itemize}

Importantly, seven of ten trials failed due to database errors when querying environment or opponent features, limiting the search to baseline and Vegas feature combinations. However, this limitation does not undermine the conclusion: we know from Phase 1 that opponent features yield only 31.3\% coverage (best manual configuration), and Trial \#0's 34.2\% represents only a modest improvement via joint hyperparameter tuning.

\subsubsection{Comparison with Prior Phases}

Figure~\ref{fig:phase_improvements_breakdown} contextualizes the optimization results within the broader investigation:

\begin{itemize}
    \item \textbf{Phase 1 (Feature Engineering)}: +5.1pp improvement (26.2\% → 31.3\%)
    \item \textbf{Phase 2 (Prior Sensitivity)}: $<$1pp variation (25.7\% - 26.7\%)
    \item \textbf{Phase 4 (Hyperparameter Optimization)}: +8.0pp improvement (26.2\% → 34.2\%)
\end{itemize}

Even when combining all improvements from single-output BNN variants, the best achievable coverage is approximately 34\%, still 56 percentage points short of the target. In contrast:

\begin{itemize}
    \item \textbf{Phase 3 (Quantile Regression)}: 89.4\% coverage (+63.2pp vs. baseline)
    \item \textbf{Phase 3 (Conformal Prediction)}: 84.5\% coverage (+58.3pp vs. baseline)
    \item \textbf{Phase 3 (Multi-output BNN)}: 92.0\% coverage (+65.8pp vs. baseline)
\end{itemize}

The architectural change (multi-output) yields an improvement \textit{eight times larger} than the best hyperparameter tuning (+65.8pp vs. +8.0pp). This quantitatively establishes that architecture dominates hyperparameters for calibration.

\subsubsection{Discussion and Implications}

Phase 4 yields a valuable negative result that strengthens the broader investigation's conclusions:

\paragraph{Hyperparameter Tuning is Insufficient}

Systematic Bayesian optimization, which should identify optimal configurations if they exist, found no single-output BNN hyperparameter setting achieving even 50\% coverage. The best configuration (34.2\%) improves over the baseline but remains catastrophically under-calibrated. This falsifies the hypothesis that we simply failed to discover good hyperparameters in manual exploration.

\paragraph{Architecture Dominates Hyperparameters (Quantified)}

We can now quantify the hierarchy of modeling choices:

\begin{align}
    \text{Architecture (multi-output)} &: +65.8\text{pp improvement} \\
    \text{Hyperparameters (optimization)} &: +8.0\text{pp improvement} \\
    \text{Features (opponent defense)} &: +5.1\text{pp improvement} \\
    \text{Priors (sensitivity analysis)} &: <1\text{pp variation}
\end{align}

Architectural innovation provides an order of magnitude more improvement than hyperparameter engineering. For practitioners facing calibration challenges, this suggests clear priorities: rethink model structure before exhaustively tuning configurations.

\paragraph{Negative Results as Methodological Contributions}

The failure of hyperparameter optimization is not a weakness of our investigation—it is a \textit{strength}. By demonstrating that even principled optimization cannot rescue single-output BNNs, we rule out an obvious alternative explanation for the multi-output BNN's success. The architectural contribution stands on firmer ground \textit{because} we exhaustively explored and documented the failure of hyperparameter-based solutions.

This aligns with recent calls for publishing negative results in machine learning \citep{sculley2018winner}. Knowing what \textit{doesn't work} is often as valuable as knowing what does, preventing future researchers from repeating failed approaches.

\paragraph{Computational Cost vs. Benefit}

The 10-trial optimization required approximately 1 hour of compute time (3 successful trials × ~18 minutes each, plus 7 fast failures). Extending to 50-100 trials would require 5-10 hours and is unlikely to discover configurations exceeding 40\% coverage based on the observed landscape. In contrast, implementing the multi-output BNN required 3-4 hours of training for a single model that achieved 92\% coverage.

This suggests a practical heuristic: when initial hyperparameter exploration yields marginal improvements, invest effort in architectural experimentation rather than exhaustive hyperparameter search. The multi-output BNN represents a more efficient path to well-calibrated predictions than dozens of single-output variants.

Having systematically tested all four initial hypotheses—feature inadequacy, prior misspecification, architectural limitations, and hyperparameter sensitivity—we now synthesize findings across phases to extract general principles for Bayesian deep learning uncertainty quantification.

\subsection{Synthesis and Key Findings}
\label{subsec:synthesis}

\subsubsection{The Hierarchy of Modeling Choices}

This investigation establishes a clear empirical hierarchy for improving calibration in Bayesian neural networks. Figure~\ref{fig:improvement_hierarchy} quantifies the marginal contribution of each intervention:

\begin{enumerate}
    \item \textbf{Architecture (multi-output BNN)}: +65.8 percentage points
    \item \textbf{Hyperparameter optimization}: +8.0 percentage points
    \item \textbf{Feature engineering}: +5.1 percentage points
    \item \textbf{Prior sensitivity}: $<1$ percentage point variation
\end{enumerate}

The architectural change provides 8-13 times more improvement than any tuning-based intervention. This hierarchy suggests a \textbf{``think structurally first'' principle} for Bayesian deep learning: when facing calibration failures, practitioners should explore alternative architectures (joint modeling, deeper networks, auxiliary tasks) before exhaustively engineering features or optimizing hyperparameters.

\subsubsection{Non-Bayesian Baselines are Essential}

The success of quantile regression (89.4\% coverage, $<2$ minutes) and conformal prediction (84.5\% coverage, 2 minutes) relative to single-output BNNs (26-34\% coverage, 25-30 minutes) demonstrates that \textbf{Bayesian methods must earn their computational cost through superior empirical performance}. Our results show:

\begin{itemize}
    \item Simple non-Bayesian methods can outperform sophisticated Bayesian models on the primary evaluation metric (coverage)
    \item Theoretical elegance (Bayesian coherence, posterior uncertainty) does not guarantee practical utility
    \item Researchers developing Bayesian deep learning methods must benchmark against non-Bayesian baselines before claiming advances
\end{itemize}

The multi-output BNN justifies its 3-4 hour training cost by achieving both calibration (92\%) \textit{and} point accuracy (18.5 MAE), demonstrating that architectural innovation can rescue Bayesian approaches where hyperparameter tuning cannot.

\subsubsection{Joint Modeling as a General Strategy}

The multi-output BNN's success suggests that \textbf{auxiliary tasks improve primary task uncertainty quantification by providing regime indicators}. This principle likely generalizes to domains where:

\begin{itemize}
    \item An auxiliary task exists that is (a) correlated with primary task difficulty, and (b) cheaper to label
    \item Primary task uncertainty varies across contexts (heteroskedasticity)
    \item Shared representations can capture regime information
\end{itemize}

Examples include joint modeling of disease presence and severity (medicine), returns and volatility (finance), or temperature and precipitation (weather). Future work should develop principled criteria for selecting auxiliary tasks and measuring their contribution to primary task calibration.

\subsubsection{Computational Investment Summary}

Table~\ref{tab:computational_cost_summary} summarizes the computational cost and return on investment for each phase.

\begin{table}[t]
\centering
\begin{tabular}{@{}lcccc@{}}
\toprule
\textbf{Phase} & \textbf{Models} & \textbf{Compute Time} & \textbf{Best Coverage} & \textbf{ROI} \\
\midrule
Phase 1: Features & 3 & 2 hrs & 31.3\% & +5.1pp / 2hrs = \textbf{2.6pp/hr} \\
Phase 2: Priors & 4 & 2 hrs & 26.7\% & $<1$pp / 2hrs = \textbf{0.5pp/hr} \\
Phase 3: Alternatives & 3 & 5 hrs & 92.0\% & +65.8pp / 5hrs = \textbf{13.2pp/hr} \\
Phase 4: Optimization & 10 & 1 hr & 34.2\% & +8.0pp / 1hr = \textbf{8.0pp/hr} \\
\midrule
\textbf{Total} & \textbf{20} & \textbf{10 hrs} & \textbf{92.0\%} & \\
\bottomrule
\end{tabular}
\caption{Computational cost and return on investment (coverage improvement per compute hour) for each investigation phase. Phase 3 (architectural exploration) provides the highest ROI at 13.2 percentage points per hour, more than 5 times better than feature engineering and 26 times better than prior tuning. This quantifies the efficiency advantage of architectural innovation over hyperparameter engineering.}
\label{tab:computational_cost_summary}
\end{table}

The data reveals stark differences in return on investment: Phase 3 (architectural exploration) yields 13.2 percentage points of coverage improvement per compute hour, compared to 2.6 for feature engineering and 0.5 for prior tuning. This quantifies the efficiency advantage of ``thinking structurally first.''

\subsection{Recommendations for Practitioners}
\label{subsec:recommendations}

Based on this systematic investigation, we offer concrete guidance for practitioners facing calibration challenges in Bayesian neural networks or selecting uncertainty quantification methods for production deployment.

\subsubsection{Decision Framework}

Table~\ref{tab:practitioner_decision_matrix} provides a decision matrix for selecting UQ methods based on application requirements.

\begin{table}[t]
\centering
\begin{tabular}{@{}llp{6cm}@{}}
\toprule
\textbf{Priority} & \textbf{Method} & \textbf{Rationale} \\
\midrule
\textbf{Calibration + Accuracy} & Multi-output BNN & Best calibration (92\%) with competitive point prediction (18.5 MAE). Justifies 3-4hr training cost. \\
\addlinespace
\textbf{Speed} & Quantile Regression & Excellent calibration (89.4\%) in $<2$ minutes. Accept wide intervals (106 yds) for fast iteration. \\
\addlinespace
\textbf{Theoretical Guarantees} & Conformal Prediction & Distribution-free coverage guarantees (84.5\%). Robust to model misspecification. Training: 2 min. \\
\addlinespace
\textbf{Point Accuracy Only} & Single-output BNN & Good MAE (18.4) but unreliable uncertainty. Use only if intervals not needed. \\
\bottomrule
\end{tabular}
\caption{Decision matrix for practitioners selecting uncertainty quantification methods. No single method dominates across all criteria; choice depends on application priorities. Avoid single-output BNNs for applications requiring calibrated intervals.}
\label{tab:practitioner_decision_matrix}
\end{table}

\subsubsection{When Facing Calibration Failures}

If your Bayesian neural network produces under-calibrated intervals:

\begin{enumerate}
    \item \textbf{First, benchmark against non-Bayesian baselines} (quantile regression, conformal prediction). If they achieve good calibration, the problem is methodological, not fundamental.

    \item \textbf{Second, explore architectural innovations} before hyperparameter tuning:
    \begin{itemize}
        \item Joint modeling with auxiliary tasks
        \item Deeper networks or alternative activation functions
        \item Ensemble methods or mixture models
    \end{itemize}

    \item \textbf{Third, engineer domain-specific features} that capture heteroskedasticity (e.g., betting markets, opponent strength, environmental conditions).

    \item \textbf{Last, optimize hyperparameters} via Bayesian optimization. Our results suggest this provides minimal improvement for structural problems.
\end{enumerate}

Do NOT expect prior tuning alone to fix severe under-calibration. Our Phase 2 results demonstrate that single-output BNNs exhibit prior robustness—calibration is dominated by architectural choices, not prior specifications.

\subsubsection{What NOT to Use}

\textbf{Avoid single-output hierarchical BNNs for production uncertainty quantification} unless you have strong evidence (via cross-validation) that they calibrate well for your specific task. Our investigation demonstrates that this architecture:

\begin{itemize}
    \item Consistently under-calibrates (26-34\% coverage across diverse configurations)
    \item Does not improve with feature engineering, prior tuning, or hyperparameter optimization
    \item Is dominated by both non-Bayesian baselines and architectural variants in calibration metrics
\end{itemize}

The intervals appear precise but are systematically overconfident, potentially leading to poor decisions in high-stakes applications.

\subsection{Limitations and Future Work}
\label{subsec:limitations}

\subsubsection{Limitations}

Several limitations should be acknowledged:

\begin{description}
    \item[Domain specificity] Results are specific to NFL rushing yards prediction. Generalization to other sports, domains, or prediction tasks requires empirical validation.

    \item[Limited optimization trials] Only 10 trials were completed in Phase 4 (7 failed due to database errors). A larger study (50-100 trials) might discover better configurations, though the observed landscape suggests this is unlikely to exceed 40\% coverage.

    \item[Single-task evaluation] We evaluate calibration solely on 90\% credible intervals. Other coverage levels (50\%, 68\%, 95\%) may exhibit different behavior.

    \item[Computational constraints] Multi-output BNN training requires 3-4 hours, limiting rapid iteration. Future work should explore variational inference or other scalable alternatives.

    \item[Mechanistic understanding] While we observe that joint modeling improves calibration, the precise mechanism by which auxiliary tasks inform uncertainty remains unclear. Ablation studies and interpretability analyses could provide deeper insight.
\end{description}

\subsubsection{Future Research Directions}

This investigation opens several promising research directions:

\begin{enumerate}
    \item \textbf{Theoretical analysis of joint modeling}: Derive formal relationships between auxiliary task informativeness and primary task calibration. Under what conditions does joint modeling provably improve uncertainty quantification?

    \item \textbf{Auxiliary task selection criteria}: Develop principled methods for selecting which auxiliary tasks to model jointly. Can we predict a priori which auxiliary tasks will improve calibration?

    \item \textbf{Domain generalization}: Test whether multi-output BNNs improve calibration in other domains (medical diagnosis, financial forecasting, weather prediction) where auxiliary information exists.

    \item \textbf{Architecture search for calibration}: Can neural architecture search (NAS) automatically discover architectures optimized for both accuracy and calibration? Our results suggest this is more promising than hyperparameter search.

    \item \textbf{Interval sharpness optimization}: The multi-output BNN achieves good calibration, but can we also achieve narrow intervals? Investigate adversarial training, conformal prediction with adaptive bandwidth, or explicit interval width regularization.

    \item \textbf{Scalability improvements}: Explore variational inference, Laplace approximation, or other scalable alternatives to full MCMC for multi-output BNNs, reducing the 3-4 hour training cost.
\end{enumerate}

The most impactful direction is likely (1) and (2): developing theory and heuristics for when joint modeling helps. If we can predict a priori which auxiliary tasks improve calibration, the multi-output BNN approach becomes broadly applicable across domains.

\subsection{Conclusion}

This systematic investigation of Bayesian neural network calibration for NFL rushing yards prediction conclusively demonstrates that:

\begin{enumerate}
    \item \textbf{Single-output BNN calibration failure is architectural, not tunable.} Feature engineering, prior sensitivity, and hyperparameter optimization provide marginal improvements (5-8 percentage points) that are insufficient to address the 64 percentage point calibration gap.

    \item \textbf{The calibration crisis is methodological, not fundamental.} Non-Bayesian baselines (quantile regression, conformal prediction) achieve 84-89\% coverage, proving the data contains sufficient information for well-calibrated predictions.

    \item \textbf{Multi-output BNN represents a fundamental architectural advance.} Joint modeling of rushing yards and touchdown probability achieves 92\% coverage (exceeding the 90\% target) while maintaining point prediction quality, validating the architectural limitation hypothesis.

    \item \textbf{Architecture dominates hyperparameters for calibration.} The multi-output BNN's 65.8 percentage point improvement is 8 times larger than the best hyperparameter tuning, establishing a clear hierarchy of modeling choices.
\end{enumerate}

For practitioners, the key message is: \textbf{think structurally first}. When facing calibration failures, explore architectural innovations before exhaustively tuning hyperparameters. And always benchmark against non-Bayesian baselines to justify the computational cost of Bayesian methods.

The multi-output BNN emerges as the only Bayesian method achieving well-calibrated prediction intervals for this task, demonstrating that architectural innovation—not feature engineering or hyperparameter optimization—is the key to reliable uncertainty quantification in Bayesian deep learning.
