%------------------------------------------------------------
%  Tufte-style handout layout for readable measure + margin notes
%  Compile with: pdflatex -> bibtex (if using .bib) -> pdflatex -> pdflatex
%------------------------------------------------------------
\PassOptionsToPackage{numbers,sort&compress}{natbib}
\documentclass[letterpaper,sfsidenotes]{tufte-handout}

% Font & micro-typography (you can switch to XeLaTeX/LuaLaTeX if you prefer)
\usepackage[T1]{fontenc}
\usepackage[utf8]{inputenc}
\usepackage{microtype}

% Math & tables
\usepackage{amsmath,amssymb}
\usepackage{booktabs}

% Graphics (supports margin figures)
\usepackage{graphicx}
\setkeys{Gin}{width=\linewidth,totalheight=\textheight,keepaspectratio}

% Hyperlinks
\usepackage{hyperref}
\hypersetup{
  colorlinks=true,
  linkcolor=blue,
  citecolor=blue,
  urlcolor=blue
}

% If you prefer BibTeX + natbib, uncomment these lines and use \citet/\citep
% \usepackage[numbers,sort&compress]{natbib}
% \bibliographystyle{plainnat}

% --- Quality-of-life macros for margin notes & structure ---
% Convenience: margin-exposition blocks (bold label + text)
\newcommand{\marginexplain}[2][]{%
  \marginpar[\footnotesize \textbf{#1}~#2]{\footnotesize \textbf{#1}~#2}
}

% A “new thought” drop lead (Tufte style)
% Provided by `tufte-handout`; ensured here in case derivative classes unset it.
\providecommand{\newthought}[1]{\textsc{#1}}

% Full-width floats (for wide tables/figures)
% Wrap large tables/figures with \begin{fullwidth} ... \end{fullwidth}

% Title
\title{Discrete Margin Modeling and Beyond:\\
A Readable, Margin-Note Edition for Football Betting Analysis}
\author{Richard Oldham}
\date{\today}

\begin{document}
\maketitle

% Optional: a brief note in the margin about line length / readability
\marginexplain[Readable line length]{Traditional research layouts often run
  90–100+ characters per line. This handout keeps the main text near
  60–75 characters per line (a widely-cited ergonomic range), improving
  scanning speed and reducing eye strain.}

% Abstract in main column; quick context in the margin
\begin{abstract}
We present a Tufte-style edition of the paper to improve readability via a narrow
main text block and rich margin notes for exposition, references, and context.
All technical content (discrete margin modeling, teaser/alt-line pricing, ML pipelines,
Bayesian hierarchical modeling, and the ensembling framework) is unchanged;
the presentation is optimized for human factors.
\end{abstract}

\section{How to Use This Layout}
\newthought{Sidenotes vs.\ footnotes.}
All footnotes are now margin \emph{sidenotes}.
Write \verb|\footnote{...}| as usual; it will render in the outer margin with ample room
for exposition, derivations, or further reading.
\marginexplain[Tip]{Use the margin to carry intuition, examples, and short proofs so the main
argument flows. Keep main text tight; let the margin teach.}

\subsection{Margin figures and tables}
Use \verb|\begin{marginfigure}...\end{marginfigure}| for small figures in the margin,
and \verb|\begin{fullwidth}...\end{fullwidth}| for wide tables/figures that need extra space.

\begin{marginfigure}
  \includegraphics{example-image-a}%
  \caption{A margin figure (e.g., a key-number histogram for NFL margins).
  Use small, high-contrast visuals here.}
\end{marginfigure}

\begin{fullwidth}
\begin{table}
  \centering
  \caption{Example full-width table for alt-line ladders or teaser payout menus.}
  \begin{tabular}{@{}lccc@{}}
    \toprule
    Market & Offered Odds & Fair Odds & EV \\
    \midrule
    Two-leg 6-pt Teaser & $-120$ & $-112$ & $+3.6\%$ \\
    Alt-Spread Parlay   & $-125$ & $-118$ & $+3.0\%$ \\
    \bottomrule
  \end{tabular}
\end{table}
\end{fullwidth}

\subsection{Recommended packages and compile path}
This template uses \texttt{tufte-handout}. It works with pdf\LaTeX.
If you switch to Xe\LaTeX/Lua\LaTeX you can load system fonts (e.g., \texttt{fontspec})—
but keep microtypography (\texttt{microtype}) on for best spacing.

\section{Drop-in Replacement for Your Paper}
Replace your previous \verb|article| document with this one by copying your content
into the sections below. All math, theorems, and figures work unchanged. Your bibliography
can remain as \verb|thebibliography| or you may switch to BibTeX/biblatex.

% ----------- BEGIN YOUR PAPER CONTENT -----------
\section{Introduction}
% Paste your original Introduction here.
% Example margin context:
\sidenote{Key-number intuition: NFL margins cluster at 3 and 7; the value of crossing
these thresholds drives teaser/alt-line EV.}
Point-spread betting in American football often revolves around key numbers (3, 7).
% ... (rest of your original text)

\section{Alternative Strategies Similar to Wong Teasers}
% Paste your Section 2 here
% Add margin commentary as exposition:
\marginexplain[Market microstructure]{Modern books dynamically price teasers.
Replicate teaser moves with alt-lines when the ladder is softer than the teaser menu.}
% ... (rest of your original text)

\section{Machine-Learning Pipelines to Simulate NFL and NCAA Outcomes}
% Paste your Section 3 here
% ... (rest of your original text)

\section{Bayesian Hierarchical Modeling for Dynamic Team Strength}
% Paste your Section 4 here
% ... (rest of your original text)

\section{Ensembling Heterogeneous Predictive Models for NFL Betting Decisions}
% Paste the new ensembling section here (the one we just wrote).
% Margin note example:
\sidenote{Blend at the \emph{probability} level via linear pools, BMA, or
log-score-optimal stacking; gate weights by context (totals/weather/season stage).}
% ... (rest of your ensembling text)

\section{Integrated Workflow and Strategy Deployment}
% Paste your Section 6 (was 5 before adding ensembles)
% ... (rest of your original text)

\section{Conclusion}
% Paste your conclusion
% ... (rest of your original text)
% ----------- END YOUR PAPER CONTENT -----------

% Bibliography: either your existing thebibliography or BibTeX
% (A) Keep your manual bibliography:
\begin{thebibliography}{99}
\bibitem{Wong2001} Wong, Stanford. \emph{Sharp Sports Betting}. Pi Yee Press, 2001.
% ... (paste the rest of your entries)
\end{thebibliography}

% (B) Or switch to BibTeX + natbib:
% \bibliography{refs}

\end{document}
